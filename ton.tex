\documentclass[12pt,oneside]{article}
\usepackage{cmap}
\usepackage[T2A]{fontenc}
\usepackage[utf8]{inputenc}
\usepackage[english,russian]{babel}
\usepackage{indentfirst}
\usepackage{comment}
%\usepackage{euler}
\usepackage{amssymb, amsmath, amsfonts, stmaryrd}
\usepackage[mathscr]{euscript}
\usepackage{mathrsfs}
\usepackage{theorem}
\usepackage[english]{babel}
\usepackage{bm}
\usepackage[all]{xy}
%\usepackage{chngcntr}
%\CompileMatrices
\usepackage[bookmarks=false,pdfauthor={Nikolai Durov},pdftitle={Telegram Open Network}]{hyperref}
\usepackage{fancyhdr}
\usepackage{caption}
%
\setlength{\headheight}{15.2pt}
\pagestyle{fancy}
\renewcommand{\headrulewidth}{0.5pt}
%
\def\makepoint#1{\medbreak\noindent{\bf #1.\ }}
\def\zeropoint{\setcounter{subsection}{-1}}
\def\zerosubpoint{\setcounter{subsubsection}{-1}}
\def\nxpoint{\refstepcounter{subsection}%
  \smallbreak\makepoint{\thesubsection}}
\def\nxsubpoint{\refstepcounter{subsubsection}%
  \smallbreak\makepoint{\thesubsubsection}}
\def\nxsubsubpoint{\refstepcounter{paragraph}%
  \makepoint{\paragraph}}
%\setcounter{secnumdepth}{4}
%\counterwithin{paragraph}{subsubsection}
\def\refpoint#1{{\rm\textbf{\ref{#1}}}}
\let\ptref=\refpoint
\def\embt(#1.){\textbf{#1.}}
\def\embtx(#1){\textbf{#1}}
\long\def\nodo#1{}
%
%\def\markbothsame#1{\markboth{#1}{#1}}
\fancyhf{}
\fancyfoot[C]{\thepage}
\def\markbothsame#1{\fancyhead[C]{#1}}
%\def\mysection#1{\section{#1}\fancyhead[C]{\textsc{Chapter \textbf{\thesection.} #1}}}
%\def\mysubsection#1{\subsection{#1}\fancyhead[C]{\small{\textsc{\textrm{\thesubsection.} #1}}}}
%\def\myappendix#1{\section{#1}\fancyhead[C]{\textsc{Appendix \textbf{\thesection.} #1}}}
\def\mysection#1{\section{#1}\fancyhead[C]{\textsc{Глава \textbf{\thesection.} #1}}}
\def\mysubsection#1{\subsection{#1}\fancyhead[C]{\small{\textsc{\textrm{\thesubsection.} #1}}}}
\def\myappendix#1{\section{#1}\fancyhead[C]{\textsc{Приложение \textbf{\thesection.} #1}}}
%\includecomment{English}
\includecomment{Russian}
\excludecomment{English}
%\excludecomment{Russian}
%
\let\tp=\textit
\let\vr=\textit
\def\workchainid{\vr{workchain\_id\/}}
\def\shardpfx{\vr{shard\_prefix}}
\def\accountid{\vr{account\_id\/}}
\def\currencyid{\vr{currency\_id\/}}
\def\uint{\tp{uint}}
\def\opsc#1{\operatorname{\textsc{#1}}}
\def\blkseqno{\opsc{blk-seqno}}
\def\blkprev{\opsc{blk-prev}}
\def\blkhash{\opsc{blk-hash}}
\def\Hash{\opsc{Hash}}
\def\Sha{\opsc{sha256}}
\def\height{\opsc{height}}
\def\len{\opsc{len}}
\def\leaf{\opsc{Leaf}}
\def\node{\opsc{Node}}
\def\root{\opsc{Root}}
\def\emptyroot{\opsc{EmptyRoot}}
\def\code{\opsc{code}}
\def\Ping{\opsc{Ping}}
\def\Store{\opsc{Store}}
\def\FindNode{\opsc{Find\_Node}}
\def\FindValue{\opsc{Find\_Value}}
\def\Bytes{\tp{Bytes}}
\def\Transaction{\tp{Transaction}}
\def\Account{\tp{Account}}
\def\State{\tp{State}}
\def\Maybe{\opsc{Maybe}}
\def\List{\opsc{List}}
\def\Block{\tp{Block}}
\def\Blockchain{\tp{Blockchain}}
\def\isValidBc{\tp{isValidBc}}
\def\evtrans{\vr{ev\_trans}}
\def\evblock{\vr{ev\_block}}
\def\Hashmap{\tp{Hashmap}}
\def\Type{\tp{Type}}
\def\nat{\tp{nat\/}}
\def\hget{\vr{hget\/}}
\def\bbB{{\mathbb{B}}}
\def\st#1{{\mathbf{#1}}}
%
\hfuzz=0.8pt

\title{Telegram Open Network}
\author{Dr.\ Nikolai Durov}% a.k.a. K.O.T.
\begin{document}

%\pagestyle{myheadings}
\maketitle

\begin{English}
    \begin{abstract}
        The aim of this text is to provide a first description of the
        Telegram Open Network (TON) and related blockchain, peer-to-peer,
        distributed storage and service hosting technologies. To reduce the
        size of this document to reasonable proportions, we focus mainly on
        the unique and defining features of the TON platform that are
        important for it to achieve its stated goals.
    \end{abstract}
\end{English}

\begin{Russian}
    \begin{abstract}
        Цель этого текста состоит в том, чтобы дать первое описание
        Telegram Open Network (TON) и связанных с ней блокчейном, одноранговой сетью,
        технологий распределенного хранения и хостинга сервисов. Чтобы уменьшить
        размер этого документа до разумного объема, мы сосредоточимся в основном на
        уникальных и определяющих функциях платформы TON, которые
        важны для достижения поставленных целей.
    \end{abstract}
\end{Russian}

\begin{English}
    \section*{Introduction}
    \markbothsame{Introduction}
\end{English}

\begin{Russian}
    \section*{Введение}
    \markbothsame{Введение}
\end{Russian}

\begin{English}
    The {\em Telegram Open Network (TON)} is a fast, secure and scalable
    blockchain and network project, capable of handling millions of
    transactions per second if necessary, and both user-friendly and
    service provider-friendly. We aim for it to be able to host all
    reasonable applications currently proposed and conceived. One might
    think about TON as a huge distributed supercomputer, or rather a huge
    ``superserver'', intended to host and provide a variety of services.
\end{English}

\begin{Russian}
    {\em Telegram Open Network (TON)} --- это быстрый, безопасный и масштабируемый
    блокчейн и сетевой проект, способный обрабатывать миллионы
    транзакций в секунду, если это необходимо, а также удобный для пользователей и
    разработчиков сервисов. Мы стремимся к тому, чтобы он мог поддерживать все
    практические приложения, предлагаемые и задуманные в настоящее время. Можно
    представить TON как огромный распределенный суперкомпьютер или, скорее, огромный
    ``суперсервер'', предназначенный для размещения и обеспечения функционирования различных сервисов.
\end{Russian}

\begin{English}
    This text is not intended to be the ultimate reference with respect to
    all implementation details. Some particulars are likely to change
    during the development and testing phases.
\end{English}

\begin{Russian}
    Данный документ не предназначен для использования в качестве
    полного справочника по всем деталям реализации. Некоторые детали могут
    измениться на этапах разработки и тестирования.
\end{Russian}
\clearpage
\tableofcontents

\clearpage
\begin{English}
    \mysection{Brief Description of TON Components}
\end{English}

\begin{Russian}
    \mysection{Краткое описание компонентов TON}
\end{Russian}
\label{sect:ton.components}

\begin{English}
    The {\em Telegram Open Network (TON)} is a combination of the
    following components:
\end{English}

\begin{Russian}
    {\em Telegram Open Network (TON)} представляет собой комбинацию следующих компонентов:
\end{Russian}

\begin{English}
    \begin{itemize}
    \item A flexible multi-blockchain platform ({\em TON Blockchain};
      cf.\ Chapter~\ptref{sect:blockchain}), capable of processing
      millions of transactions per second, with Turing-complete smart
      contracts, upgradable formal blockchain specifications,
      multi-cryptocurrency value transfer, support for micropayment
      channels and off-chain payment networks. {\em TON Blockchain\/}
      presents some new and unique features, such as the ``self-healing''
      vertical block\-chain mechanism (cf.~\ptref{sp:inv.sh.blk.corr}) and
      Instant Hypercube Routing (cf.~\ptref{sp:instant.hypercube}), which
      enable it to be fast, reliable, scalable and self-consistent at the
      same time.
    \item A peer-to-peer network ({\em TON P2P Network}, or just {\em TON
      Network}; cf.\ Chapter~\ptref{sect:network}), used for accessing the
      TON Block\-chain, sending transaction candidates, and receiving
      updates about only those parts of the blockchain a client is
      interested in (e.g., those related to the client's accounts and
      smart contracts), but also able to support arbitrary distributed
      services, blockchain-related or not.
    \item A distributed file storage technology {\em (TON Storage;}
      cf.~\ptref{sp:ex.ton.storage}), accessible through {\em TON
        Network}, used by the TON Blockchain to store archive copies of
      blocks and status data (snapshots), but also available for storing
      arbitrary files for users or other services running on the platform,
      with torrent-like access technology.
    \item A network proxy/anonymizer layer {\em (TON Proxy;}
      cf.~\ptref{sp:ex.ton.proxy} and~\ptref{sp:tunnels}), similar to the
      $I^2P$ (Invisible Internet Project), used to hide the identity and
      IP addresses of {\em TON Network\/} nodes if necessary (e.g., nodes
      committing transactions from accounts with large amounts of
      cryptocurrency, or high-stake blockchain validator nodes who wish to
      hide their exact IP address and geographical location as a measure
      against DDoS attacks).
    \item A Kademlia-like distributed hash table ({\em TON DHT};
      cf.~\ptref{sect:kademlia}), used as a ``torrent tracker'' for {\em
        TON Storage} (cf.~\ptref{sp:distr.torr.tr}), as an ``input tunnel
      locator'' for {\em TON Proxy\/} (cf.~\ptref{sp:loc.abs.addr}), and
      as a service locator for {\em TON Services}
      (cf.~\ptref{sp:loc.serv}).
    \item A platform for arbitrary services ({\em TON Services};
      cf.\ Chapter~\ptref{sect:services}), residing in and available
      through {\em TON Network\/} and {\em TON Proxy}, with formalized
      interfaces (cf.~\ptref{sp:pub.int.smartc}) enabling browser-like or
      smartphone application interaction. These formal interfaces and
      persistent service entry points can be published in the TON
      Blockchain (cf.~\ptref{sp:ui.ton.dns}); actual nodes providing
      service at any given moment can be looked up through the {\em TON
        DHT\/} starting from information published in the TON Blockchain
      (cf.~\ptref{sp:loc.serv}). Services may create smart contracts in
      the TON Blockchain to offer some guarantees to their clients
      (cf.~\ptref{sp:mixed.serv}).
    \item {\em TON DNS\/} (cf.~\ptref{sp:ton.dns}), a service for
      assigning human-readable names to accounts, smart contracts,
      services and network nodes.
    \item {\em TON Payments\/} (cf.\ Chapter~\ptref{sect:payments}), a
      platform for micropayments, micropayment channels and a micropayment
      channel network. It can be used for fast off-chain value transfers,
      and for paying for services powered by {\em TON Services}.
    \item TON will allow easy integration with third-party messaging and
      social networking applications, thus making blockchain technologies
      and distributed services finally available and accessible to
      ordinary users (cf.~\ptref{sp:ton.www}), rather than just to a
      handful of early cryptocurrency adopters. We will provide an example
      of such an integration in another of our projects, the Telegram
      Messenger (cf.~\ptref{sp:telegram.integr}).
    \end{itemize}
\end{English}

\begin{Russian}
    \begin{itemize}
    \item Гибкая мультиблокчейн-платформа ({\em TON Blockchain};
        см.\ Главу~\ptref{sect:blockchain}), способная обрабатывать
        миллионы транзакций в секунду, с полными по Тьюрингу
        смарт-контрактами, обновляемыми формальными спецификациями
        блокчейна, мультивалютной передачей стоимости, поддержкой каналов
        для микроплатежей и внешних платежных сетей. {\em TON Blockchain\/}
        представляет некоторые новые и уникальные функции, такие как
        ``самовосстанавливающийся'' механизм вертикальной цепочки блоков
        (см.~\ptref{sp:inv.sh.blk.corr}) и мгновенная маршрутизация в
        гиперкубе (см.\ Instant Hypercube Routing ~\ptref{sp:instant.hypercube}),
        которые обеспечивают быстроту, надежность, масштабируемость и
        самосогласованность одновременно.
    \item Одноранговая сеть ({\em TON P2P Network} или просто {\em TON Network};
        см.\ Главу~\ptref{sect:network}), используемая для доступа к цепочке блоков TON,
        отправки кандидатов на транзакции и получения обновлений только о тех частях
        блокчейна, которые интересуют клиента (например, те, которые связаны с
        учетными записями клиента и смарт-контрактами), но также способная поддерживать
        произвольные распределенные сервисы, независимо от того, связаны они с
        блокчейном или нет.
    \item Технология распределенного хранения файлов {\em (TON Storage;}
        см.~\ptref{sp:ex.ton.storage}), доступная через {\em TON Network},
        используемая блокчейном TON для хранения архивных копий блоков и
        данных о состоянии (snapshots), а также доступная для хранения
        произвольных файлов пользователей или других сервисов, работающих
        на платформе, с технологией доступа, подобной торрент-сети.
    \item  Уровень сетевого прокси/анонимайзера {\em (TON Proxy;}
        см.~\ptref{sp:ex.ton.proxy} и~\ptref{sp:tunnels}), аналогичный
        $I^2P$ (Invisible Internet Project), используемый для сокрытия
        личности и IP-адресов узлов {\em TON Network\/}, если это
        необходимо (например, узлы, совершающие транзакции со счетов
        с большим количеством криптовалюты, или узлы-валидаторы блокчейна
        с высокими ставками, которые желают скрыть свои точные IP-адреса
        и географическое положение в качестве меры защиты от DDoS-атак).
    \item Распределенная хэш-таблица, работающая по протоколу Kademlia
        ({\em TON DHT}; см.~\ptref{sect:kademlia}), используемая в качестве
        ``торрент-трекера'' для {\em TON Storage} (см.~\ ptref{sp:distr.torr.tr}),
        ``входного локатора туннеля'' для {\em TON Proxy\/} (см.~\ptref{sp:loc.abs.addr})
        и в качестве локатора сервисов для {\em TON Services} (см.~\ptref{sp:loc.serv}).
    \item Платформа для произвольных сервисов ({\em TON Services};
        см.\ Главу~\ptref{sect:services}), находящихся и доступных через
        {\em TON Network\/} и {\em TON Proxy}, с формализованными интерфейсами
        (cf.~\ptref{sp:pub.int.smartc}), обеспечивающими взаимодействие с
        браузером или приложением для смартфона. Эти формальные интерфейсы
        и постоянные точки входа в сервисы могут быть опубликованы в блокчейне
        TON (см. ~\ptref{sp:ui.ton.dns}); актуальные узлы, предоставляющие
        услуги в любой момент, можно найти через {\em TON DHT\/}, начиная с
        информации, опубликованной в блокчейне TON (см. ~\ptref{sp:loc.serv}).
        Сервисы могут создавать смарт-контракты в блокчейне TON, чтобы предлагать
        своим клиентам некоторые гарантии (см.~\ptref{sp:mixed.serv}).
    \item {\em TON DNS\/} (см.~\ptref{sp:ton.dns}), служба для присвоения
        удобочитаемых имен учетным записям, смарт-контрактам, сервисам и сетевым узлам.
    \item {\em TON Payments\/} (см.\ Главу~\ptref{sect:payments}), платформа
        для микроплатежей, каналов микроплатежей и сети каналов микроплатежей.
        Его можно использовать для быстрых переводов криптовалюты вне сети и
        для оплаты услуг, предоставляемых {\em TON Services}.
    \item TON будет обеспечивать легкую интеграцию со сторонними приложениями
        для обмена сообщениями и социальными сетями, тем самым сделав
        блокчейн-технологии и распределенные сервисы наконец-то полезными и
        доступными для обычных пользователей (см.~\ptref{sp:ton.www}),
        а не только для маленькой группы первых пользователей криптовалюты.
        Мы приведем пример такой интеграции в другом нашем проекте,
        Telegram Messenger (cf.~\ptref{sp:telegram.integr})
    \end{itemize}
\end{Russian}

\begin{English}
    While the TON Blockchain is the core of the TON project, and the other
    components might be considered as playing a supportive role for the
    blockchain, they turn out to have useful and interesting functionality
    by themselves. Combined, they allow the platform to host more
    versatile applications than it would be possible by just using the TON
    Blockchain (cf.~\ptref{sp:blockchain.facebook}
    and~\ptref{sect:ton.service.impl}).
\end{English}

\begin{Russian}
    Несмотря на то, что блокчейн TON является ядром проекта TON, а другие
    компоненты могут рассматриваться как вспомогательные для блокчейна,
    они сами по себе обладают полезными и интересными функциями.
    В совокупности они позволяют платформе размещать более универсальные
    приложения, чем это было бы возможно, просто используя блокчейн TON (см. ~\ptref{sp:blockchain.facebook} и~\ptref{sect:ton.service.impl}).
\end{Russian}

\clearpage
\begin{English}
    \mysection{TON Blockchain}
\end{English}

\begin{Russian}
    \mysection{Блокчейн TON}
\end{Russian}

\label{sect:blockchain}

We start with a description of the Telegram Open Network (TON)
Blockchain, the core component of the project. Our approach here is
``top-down'': we give a general description of the whole first, and
then provide more detail on each component.

For simplicity, we speak here about {\em the\/} TON Blockchain, even
though in principle several instances of this blockchain protocol may
be running independently (for example, as a result of hard forks). We
consider only one of them.

\mysubsection{TON Blockchain as a Collection of 2-Blockchains}

The TON Blockchain is actually a {\em collection\/} of blockchains
(even a collection of {\em blockchains of blockchains}, or {\em
  2-blockchains}---this point will be clarified later
in~\ptref{sp:inv.sh.blk.corr}), because no single blockchain project
is capable of achieving our goal of processing millions of
transactions per second, as opposed to the now-standard dozens of
transactions per second.

\nxsubpoint\label{sp:list.blkch.typ}
\embt(List of blockchain types.) The blockchains in this collection
are:
\begin{itemize}
\item The unique {\em master blockchain}, or {\em masterchain\/} for
  short, containing general information about the protocol and the
  current values of its parameters, the set of validators and their
  stakes, the set of currently active workchains and their ``shards'',
  and, most importantly, the set of hashes of the most recent blocks
  of all workchains and shardchains.
\item Several (up to $2^{32}$) {\em working blockchains}, or {\em
  workchains\/} for short, which are actually the ``workhorses'',
  containing the value-transfer and smart-contract
  transactions. Different workchains may have different ``rules'',
  meaning different formats of account addresses, different formats of
  transactions, different virtual machines (VMs) for smart contracts,
  different basic cryptocurrencies and so on. However, they all must
  satisfy certain basic interoperability criteria to make interaction
  between different work\-chains possible and relatively simple. In
  this respect, the TON Blockchain is {\em heterogeneous\/}
  (cf.~\ptref{sp:blkch.hom.het}), similarly to the EOS
  (cf.~\ptref{sp:discuss.EOS}) and PolkaDot
  (cf.~\ptref{sp:discuss.PolkaDot}) projects.
\item Each workchain is in turn subdivided into up to $2^{60}$ {\em
  shard blockchains}, or {\em shardchains\/} for short, having the
  same rules and block format as the workchain itself, but responsible
  only for a subset of accounts, depending on several first (most
  significant) bits of the account address. In other words, a form of
  sharding is built into the system
  (cf.~\ptref{sp:shard.supp}). Because all these shardchains share a
  common block format and rules, the TON Blockchain is {\em
    homogeneous\/} in this respect (cf.~\ptref{sp:blkch.hom.het}),
  similarly to what has been discussed in one of Ethereum scaling
  proposals.\footnote{\url{https://github.com/ethereum/wiki/wiki/Sharding-FAQ}}
\item Each block in a shardchain (and in the masterchain) is actually
  not just a block, but a small blockchain. Normally, this ``block
  blockchain'' or ``vertical blockchain'' consists of exactly one
  block, and then we might think this is just the corresponding block
  of the shardchain (also called ``horizontal block\-chain'' in this
  situation). However, if it becomes necessary to fix incorrect
  shardchain blocks, a new block is committed into the ``vertical
  block\-chain'', containing either the replacement for the invalid
  ``horizontal blockchain'' block, or a ``block difference'',
  containing only a description of those parts of the previous version
  of this block that need to be changed. This is a TON-specific
  mechanism to replace detected invalid blocks without making a true
  fork of all shardchains involved; it will be explained in more
  detail in~\ptref{sp:inv.sh.blk.corr}. For now, we just remark that
  each shardchain (and the masterchain) is not a conventional
  blockchain, but a {\em blockchain of blockchains}, or {\em
    2D-blockchain}, or just a {\em 2-blockchain}.
\end{itemize}

\nxsubpoint\label{sp:ISP} \embt(Infinite Sharding Paradigm.)  Almost
all blockchain sharding proposals are ``top-down'': one first imagines
a single blockchain, and then discusses how to split it into several
interacting shardchains to improve performance and achieve
scalability.

The TON approach to sharding is ``bottom-up'', explained as follows.

Imagine that sharding has been taken to its extreme, so that exactly
one account or smart contract remains in each shardchain. Then we have
a huge number of ``account-chains'', each describing the state and
state transitions of only one account, and sending value-bearing
messages to each other to transfer value and information.

Of course, it is impractical to have hundreds of millions of blockchains, with updates (i.e., new blocks) usually appearing quite rarely in each of them. In order to implement them more efficiently, we group these ``account-chains'' into ``shardchains'', so that each block of the shardchain is essentially a collection of blocks of account-chains that have been assigned to this shard. Thus the ``account-chains'' have only a purely virtual or logical existence inside the ``shardchains''.

We call this perspective the {\em Infinite Sharding Paradigm}. It explains many of the design decisions for the TON Blockchain.

\nxsubpoint\label{sp:msg.IHR} \embt(Messages. Instant Hypercube Routing.)
The Infinite Sharding Para\-digm instructs us to regard each account
(or smart contract) as if it were in its own shardchain by
itself. Then the only way one account might affect the state of
another is by sending a {\em message\/} to it (this is a special
instance of the so-called Actor model, with accounts as Actors;
cf.~\ptref{sp:actors}). Therefore, a system of messages between
accounts (and shardchains, because the source and destination accounts
are, generally speaking, located in different shardchains) is of
paramount importance to a scalable system such as the TON
Blockchain. In fact, a novel feature of the TON Blockchain, called
{\em Instant Hypercube Routing\/} (cf.~\ptref{sp:instant.hypercube}),
enables it to deliver and process a message created in a block of one
shardchain into the very next block of the destination shardchain,
{\em regardless of the total number of shardchains in the system.}

\nxsubpoint \embt(Quantity of masterchains, workchains and
shardchains.) A TON Blockchain contains exactly one
masterchain. However, the system can potentially accommodate up to
$2^{32}$ workchains, each subdivided into up to $2^{60}$ shardchains.

\nxsubpoint \embt(Workchains can be virtual blockchains, not true
blockchains.) Because a workchain is usually subdivided into
shardchains, the existence of the workchain is ``virtual'', meaning
that it is not a true blockchain in the sense of the general
definition provided in~\ptref{sp:gen.blkch.def} below, but just a
collection of shardchains. When only one shardchain corresponds to a
workchain, this unique shardchain may be identified with the
workchain, which in this case becomes a ``true'' blockchain, at least
for some time, thus gaining a superficial similarity to customary
single-blockchain design. However, the Infinite Sharding Paradigm
(cf.~\ptref{sp:ISP}) tells us that this similarity is indeed
superficial: it is just a coincidence that the potentially huge number
of ``account-chains'' can temporarily be grouped into one blockchain.

\nxsubpoint \embt(Identification of workchains.)  Each workchain is
identified by its {\em number\/} or {\em workchain identifier\/}
($\workchainid:\uint_{32}$), which is simply an unsigned 32-bit
integer. Workchains are created by special transactions in the
masterchain, defining the (previously unused) workchain identifier and
the formal description of the workchain, sufficient at least for the
interaction of this workchain with other workchains and for
superficial verification of this workchain's blocks.

\nxsubpoint \embt(Creation and activation of new workchains.)  The
creation of a new workchain may be initiated by essentially any member
of the community, ready to pay the (high) masterchain transaction fees
required to publish the formal specification of a new
workchain. However, in order for the new workchain to become active, a
two-thirds consensus of validators is required, because they will need
to upgrade their software to process blocks of the new workchain, and
signal their readiness to work with the new workchain by special
masterchain transactions. The party interested in the activation of
the new workchain might provide some incentive for the validators to
support the new workchain by means of some rewards distributed by a
smart contract.

\nxsubpoint\label{sp:shard.ident} \embt(Identification of
shardchains.)  Each shardchain is identified by a couple
$(w,s)=(\workchainid, \shardpfx)$, where $\workchainid:\uint_{32}$
identifies the corresponding workchain, and
$\shardpfx:\st2^{0\ldots60}$ is a bit string of length at most 60,
defining the subset of accounts for which this shardchain is
responsible. Namely, all accounts with $\accountid$ starting with
$\shardpfx$ (i.e., having $\shardpfx$ as most significant bits) will
be assigned to this shardchain.

\nxsubpoint \embt(Identification of account-chains.)  Recall that
account-chains have only a virtual existence
(cf.~\ptref{sp:ISP}). However, they have a natural
identifier---namely, $(\workchainid,\accountid)$---because any
account-chain contains information about the state and updates of
exactly one account (either a simple account or smart contract---the
distinction is unimportant here).

\nxsubpoint\label{sp:dyn.split.merge} \embt(Dynamic splitting and
merging of shardchains; cf.~\ptref{sect:split.merge}.)  A less
sophisticated system might use {\em static sharding}---for example, by
using the top eight bits of the $\accountid$ to select one of 256
pre-defined shards.

An important feature of the TON Blockchain is that it implements {\em
  dynamic sharding}, meaning that the number of shards is not
fixed. Instead, shard $(w,s)$ can be automatically subdivided into
shards $(w,s.0)$ and $(w,s.1)$ if some formal conditions are met
(essentially, if the transaction load on the original shard is high
enough for a prolonged period of time). Conversely, if the load stays
too low for some period of time, the shards $(w,s.0)$ and $(w,s.1)$
can be automatically merged back into shard $(w,s)$.

Initially, only one shard $(w,\emptyset)$ is created for workchain
$w$. Later, it is subdivided into more shards, if and when this becomes necessary (cf.~\ptref{sp:split.necess} and~\ptref{sp:merge.necess}).

\nxsubpoint\label{sp:basic.workchain} \embt(Basic workchain or
Workchain Zero.)  While up to $2^{32}$ workchains can be defined with
their specific rules and transactions, we initially define only one,
with $\workchainid=0$. This workchain, called Workchain Zero or the
basic workchain, is the one used to work with {\em TON smart
  contracts\/} and transfer {\em TON coins}, also known as {\em
  Grams\/} (cf.\ Appendix~\ref{app:coins}). Most applications are
likely to require only Workchain Zero. Shardchains of the basic
workchain will be called {\em basic shardchains}.

\nxsubpoint \embt(Block generation intervals.)  We expect a new block
to be generated in each shardchain and the masterchain approximately
once every five seconds. This will lead to reasonably small
transaction confirmation times. New blocks of all shardchains are
generated approximately simultaneously; a new block of the masterchain
is generated approximately one second later, because it must contain
the hashes of the latest blocks of all shardchains.

\nxsubpoint\label{sp:sc.hash.mc} \embt(Using the masterchain to make
workchains and shardchains tightly coupled.)  Once the hash of a block
of a shardchain is incorporated into a block of the masterchain, that
shardchain block and all its ancestors are considered ``canonical'',
meaning that they can be referenced from the subsequent blocks of all
shardchains as something fixed and immutable. In fact, each new
shardchain block contains a hash of the most recent masterchain block,
and all shardchain blocks referenced from that masterchain block are
considered immutable by the new block.

Essentially, this means that a transaction or a message committed in a
shardchain block may be safely used in the very next blocks of the
other shardchains, without needing to wait for, say, twenty
confirmations (i.e., twenty blocks generated after the original block
in the same blockchain) before forwarding a message or taking other
actions based on a previous transaction, as is common in most proposed
``loosely-coupled'' systems (cf.~\ptref{sp:blkch.interact}), such as
EOS. This ability to use transactions and messages in other
shardchains a mere five seconds after being committed is one of the
reasons we believe our ``tightly-coupled'' system, the first of its
kind, will be able to deliver unprecedented performance
(cf.~\ptref{sp:shard.supp} and~\ptref{sp:blkch.interact}).

\nxsubpoint \embt(Masterchain block hash as a global state.)
According to~\ptref{sp:sc.hash.mc}, the hash of the last masterchain
block completely determines the overall state of the system from the
perspective of an external observer. One does not need to monitor the
state of all shardchains separately.

\nxsubpoint \embt(Generation of new blocks by validators;
cf.~\ptref{sect:validators}.)  The TON Blockchain uses a
Proof-of-Stake (PoS) approach for generating new blocks in the
shardchains and the masterchain. This means that there is a set of,
say, up to a few hundred {\em validators}---special nodes that have
deposited {\em stakes\/} (large amounts of TON coins) by a special
masterchain transaction to be eligible for new block generation and
validation.

Then a smaller subset of validators is assigned to each shard $(w,s)$
in a deterministic pseudorandom way, changing approximately every 1024
blocks. This subset of validators suggests and reaches consensus on
what the next shardchain block would be, by collecting suitable
proposed transactions from the clients into new valid block
candidates. For each block, there is a pseudorandomly chosen order on
the validators to determine whose block candidate has the highest
priority to be committed at each turn.

Validators and other nodes check the validity of the proposed block
candidates; if a validator signs an invalid block candidate, it may be
automatically punished by losing part or all of its stake, or by being
suspended from the set of validators for some time. After that, the
validators should reach consensus on the choice of the next block,
essentially by an efficient variant of the BFT (Byzantine Fault
Tolerant; cf.~\ptref{sp:dpos.bft}) consensus protocol, similar to
PBFT~\cite{PBFT} or Honey Badger BFT~\cite{HoneyBadger}. If consensus
is reached, a new block is created, and validators divide between
themselves the transaction fees for the transactions included, plus
some newly-created (``minted'') coins.

Each validator can be elected to participate in several validator
subsets; in this case, it is expected to run all validation and
consensus algorithms in parallel.

After all new shardchain blocks are generated or a timeout is passed,
a new masterchain block is generated, including the hashes of the
latest blocks of all shardchains. This is done by BFT consensus of
{\em all\/} validators.\footnote{Actually, two-thirds by stake is
  enough to achieve consensus, but an effort is made to collect as
  many signatures as possible.}

More detail on the TON PoS approach and its economical model is
provided in section~\ptref{sect:validators}.

\nxsubpoint \embt(Forks of the masterchain.)  A complication that
arises from our tightly-coupled approach is that switching to a
different fork in the masterchain will almost necessarily require
switching to another fork in at least some of the shardchains. On the
other hand, as long as there are no forks in the masterchain, no forks
in the shardchain are even possible, because no blocks in the
alternative forks of the shardchains can become ``canonical'' by
having their hashes incorporated into a masterchain block.

The general rule is that {\em if masterchain block $B'$ is a
  predecessor of $B$, $B'$ includes hash $\Hash(B'_{w,s})$ of
  $(w,s)$-shardchain block $B'_{w,s}$, and $B$ includes hash
  $\Hash(B_{w,s})$, then $B'_{w,s}$ {\bf must} be a predecessor of
  $B_{w,s}$; otherwise, the masterchain block $B$ is invalid.}

We expect masterchain forks to be rare, next to non-existent, because
in the BFT paradigm adopted by the TON Blockchain they can happen only
in the case of incorrect behavior by a {\em majority\/} of validators
(cf.~\ptref{sp:validators} and~\ptref{sp:new.master.blk}), which would
imply significant stake losses by the offenders. Therefore, no true
forks in the shardchains should be expected. Instead, if an invalid
shardchain block is detected, it will be corrected by means of the
``vertical blockchain'' mechanism of the 2-blockchain
(cf.~\ptref{sp:inv.sh.blk.corr}), which can achieve this goal without
forking the ``horizontal blockchain'' (i.e., the shardchain). The same
mechanism can be used to fix non-fatal mistakes in the masterchain
blocks as well.

\nxsubpoint\label{sp:inv.sh.blk.corr} \embt(Correcting invalid
shardchain blocks.)  Normally, only valid shardchain blocks will be
committed, because validators assigned to the shardchain must reach a
two-thirds Byzantine consensus before a new block can be
committed. However, the system must allow for detection of previously
committed invalid blocks and their correction.

Of course, once an invalid shardchain block is found---either by a
validator (not necessarily assigned to this shardchain) or by a
``fisherman'' (any node of the system that made a certain deposit to
be able to raise questions about block validity;
cf.~\ptref{sp:fish})---the invalidity claim and its proof are
committed into the masterchain, and the validators that have signed
the invalid block are punished by losing part of their stake and/or
being temporarily suspended from the set of validators (the latter
measure is important for the case of an attacker stealing the private
signing keys of an otherwise benign validator).

However, this is not sufficient, because the overall state of the
system (TON Block\-chain) turns out to be invalid because of the
invalid shardchain block previously committed. This invalid block must
be replaced by a newer valid version.

Most systems would achieve this by ``rolling back'' to the last block
before the invalid one in this shardchain and the last blocks
unaffected by messages propagated from the invalid block in each of
the other shardchains, and creating a new fork from these blocks. This
approach has the disadvantage that a large number of otherwise correct
and committed transactions are suddenly rolled back, and it is unclear
whether they will be included later at all.

The TON Blockchain solves this problem by making each ``block'' of
each shardchain and of the masterchain (``horizontal blockchains'') a
small blockchain (``vertical blockchain'') by itself, containing
different versions of this ``block'', or their
``differences''. Normally, the vertical blockchain consists of exactly
one block, and the shardchain looks like a classical
blockchain. However, once the invalidity of a block is confirmed and
committed into a masterchain block, the ``vertical blockchain'' of the
invalid block is allowed to grow by a new block in the vertical
direction, replacing or editing the invalid block. The new block is
generated by the current validator subset for the shardchain in
question.

The rules for a new ``vertical'' block to be valid are quite
strict. In particular, if a virtual ``account-chain block''
(cf.~\ptref{sp:ISP}) contained in the invalid block is valid by
itself, it must be left unchanged by the new vertical block.

Once a new ``vertical'' block is committed on top of the invalid
block, its hash is published in a new masterchain block (or rather in
a new ``vertical'' block, lying above the original masterchain block
where the hash of the invalid shardchain block was originally
published), and the changes are propagated further to any shardchain
blocks referring to the previous version of this block (e.g., those
having received messages from the incorrect block). This is fixed by
committing new ``vertical'' blocks in vertical blockchains for all
blocks previously referring to the ``incorrect'' block; new vertical
blocks will refer to the most recent (corrected) versions
instead. Again, strict rules forbid changing account-chains that are
not really affected (i.e., that receive the same messages as in the
previous version). In this way, fixing an incorrect block generates
``ripples'' that are ultimately propagated towards the most recent
blocks of all affected shardchains; these changes are reflected in new
``vertical'' masterchain blocks as well.

Once the ``history rewriting'' ripples reach the most recent blocks,
the new shardchain blocks are generated in one version only, being
successors of the newest block versions only. This means that they
will contain references to the correct (most recent) vertical blocks
from the very beginning.

The masterchain state implicitly defines a map transforming the hash
of the first block of each ``vertical'' blockchain into the hash of
its latest version. This enables a client to identify and locate any
vertical blockchain by the hash of its very first (and usually the
only) block.

\nxsubpoint \embt(TON coins and multi-currency workchains.)  The TON
Block\-chain supports up to $2^{32}$ different ``cryptocurrencies'',
``coins'', or ``tokens'', distinguished by a 32-bit $\currencyid$. New
cryptocurrencies can be added by special transactions in the
masterchain. Each workchain has a basic cryptocurrency, and can have
several additional cryptocurrencies.

There is one special cryptocurrency with $\currencyid=0$, namely, the
{\em TON coin}, also known as the {\em Gram\/}
(cf.\ Appendix~\ref{app:coins}). It is the basic cryptocurrency of
Workchain Zero. It is also used for transaction fees and validator
stakes.

In principle, other workchains may collect transaction fees in other
tokens. In this case, some smart contract for automated conversion of
these transaction fees into Grams should be provided.

\nxsubpoint \embt(Messaging and value transfer.)  Shardchains
belonging to the same or different workchains may send {\em
  messages\/} to each other. While the exact form of the messages
allowed depends on the receiving workchain and receiving account
(smart contract), there are some common fields making inter-workchain
messaging possible. In particular, each message may have some {\em
  value} attached, in the form of a certain amount of Grams (TON
coins) and/or other registered cryptocurrencies, provided they are
declared as acceptable cryptocurrencies by the receiving workchain.

The simplest form of such messaging is a value transfer from one
(usually not a smart-contract) account to another.

\nxsubpoint\label{sp:tonvm} \embt(TON Virtual Machine.)  The {\em TON
  Virtual Machine}, also abbreviated as {\em TON VM\/} or {\em TVM\/},
is the virtual machine used to execute smart-contract code in the
masterchain and in the basic workchain. Other workchains may use other
virtual machines alongside or instead of the TVM.

Here we list some of its features. They are discussed further
in~\ptref{sp:pec.tvm}, \ptref{sp:tvm.cells} and elsewhere.

\begin{itemize}
\item TVM represents all data as a collection of {\em (TVM) cells\/}
  (cf.~\ptref{sp:tvm.cells}). Each cell contains up to 128 data bytes
  and up to 4 references to other cells. As a consequence of the
  ``everything is a bag of cells'' philosophy
  (cf.~\ptref{sp:everything.is.BoC}), this enables TVM to work with
  all data related to the TON Blockchain, including blocks and
  blockchain global state if necessary.
\item TVM can work with values of arbitrary algebraic data types
  (cf.~\ptref{sp:pec.tvm}), represented as trees or directed acyclic
  graphs of TVM cells. However, it is agnostic towards the existence
  of algebraic data types; it just works with cells.
\item TVM has built-in support for hashmaps (cf.~\ptref{sp:patricia}).
\item TVM is a stack machine. Its stack keeps either 64-bit integers
  or cell references.
\item 64-bit, 128-bit and 256-bit arithmetic is supported. All $n$-bit
  arithmetic operations come in three flavors: for unsigned integers,
  for signed integers and for integers modulo $2^n$ (no automatic
  overflow checks in the latter case).
\item TVM has unsigned and signed integer conversion from $n$-bit to
  $m$-bit, for all $0\leq m,n\leq 256$, with overflow checks.
\item All arithmetic operations perform overflow checks by default,
  greatly simplifying the development of smart contracts.
\item TVM has ``multiply-then-shift'' and ``shift-then-divide''
  arithmetic operations with intermediate values computed in a larger
  integer type; this simplifies implementing fixed-point arithmetic.
\item TVM offers support for bit strings and byte strings.
\item Support for 256-bit Elliptic Curve Cryptography (ECC) for some
  predefined curves, including Curve25519, is present.
\item Support for Weil pairings on some elliptic curves, useful for
  fast implementation of zk-SNARKs, is also present.
\item Support for popular hash functions, including $\Sha$, is
  present.
\item TVM can work with Merkle proofs
  (cf.~\ptref{sp:ton.smart.pc.supp}).
\item TVM offers support for ``large'' or ``global'' smart
  contracts. Such smart contracts must be aware of sharding
  (cf.~\ptref{sp:loc.glob.smct} and \ptref{sp:tvm.data.shard}). Usual
  (local) smart contracts can be sharding-agnostic.
\item TVM supports closures.
\item A ``spineless tagless $G$-machine'' \cite{STGM} can be easily
  implemented inside TVM.
\end{itemize}
Several high-level languages can be designed for TVM, in addition to
the ``TVM assembly''. All these languages will have static types and
will support algebraic data types.  We envision the following
possibilities:
\begin{itemize}
\item A Java-like imperative language, with each smart contract
  resembling a separate class.
\item A lazy functional language (think of Haskell).
\item An eager functional language (think of ML).
\end{itemize}

\nxsubpoint\label{sp:config.param} \embt(Configurable parameters.)  An
important feature of the TON Block\-chain is that many of its
parameters are {\em configurable}. This means that they are part of
the masterchain state, and can be changed by certain special
proposal/vote/result transactions in the masterchain, without any need
for hard forks. Changing such parameters will require collecting
two-thirds of validator votes and more than half of the votes of all
other participants who would care to take part in the voting process
in favor of the proposal.

\mysubsection{Generalities on Blockchains}

\nxsubpoint\label{sp:gen.blkch.def} \embt(General blockchain
definition.)  In general, any {\em (true) blockchain\/} is a sequence
of {\em blocks}, each block $B$ containing a reference $\blkprev(B)$
to the previous block (usually by including the hash of the previous
block into the header of the current block), and a list of {\em
  transactions}. Each transaction describes some transformation of the
{\em global blockchain state}; the transactions listed in a block are
applied sequentially to compute the new state starting from the old
state, which is the resulting state after the evaluation of the
previous block.

\nxsubpoint \embt(Relevance for the TON Blockchain.)  Recall that the
            {\em TON Block\-chain\/} is not a true blockchain, but a
            collection of 2-blockchains (i.e., of blockchains of
            blockchains; cf.~\ptref{sp:list.blkch.typ}), so the above
            is not directly applicable to it. However, we start with
            these generalities on true blockchains to use them as
            building blocks for our more sophisticated constructions.

\nxsubpoint \embt(Blockchain instance and blockchain type.)  One often
uses the word {\em blockchain\/} to denote both a general {\em
  blockchain type\/} and its specific {\em blockchain instances},
defined as sequences of blocks satisfying certain conditions. For
example, \ptref{sp:gen.blkch.def} refers to blockchain instances.

In this way, a blockchain type is usually a ``subtype'' of the type
$\Block^*$ of lists (i.e., finite sequences) of blocks, consisting of
those sequences of blocks that satisfy certain compatibility and
validity conditions:
\begin{equation}
  \Blockchain \subset \Block^*
\end{equation}

A better way to define $\Blockchain$ would be to say that
$\Blockchain$ is a {\em dependent couple type}, consisting of couples
$(\bbB,v)$, with first component $\bbB:\Block^*$ being of type
$\Block^*$ (i.e., a list of blocks), and the second component
$v:\isValidBc(\bbB)$ being a proof or a witness of the validity of
$\bbB$. In this way,
\begin{equation}
  \Blockchain\equiv\Sigma_{(\bbB:\Block^*)}\isValidBc(\bbB)
\end{equation}
We use here the notation for dependent sums of types borrowed from~\cite{HoTT}.

\nxsubpoint \embt(Dependent type theory, Coq and TL.)  Note that we
are using (Martin-L\"of) dependent type theory here, similar to that
used in the Coq\footnote{\url{https://coq.inria.fr}} proof
assistant. A simplified version of dependent type theory is also used
in {\em TL (Type
  Language)},\footnote{\url{https://core.telegram.org/mtproto/TL}}
which will be used in the formal specification of the TON Blockchain
to describe the serialization of all data structures and the layouts
of blocks, transactions, and the like.

In fact, dependent type theory gives a useful formalization of what a
proof is, and such formal proofs (or their serializations) might
become handy when one needs to provide proof of invalidity for some
block, for example.

\nxsubpoint\label{sp:TL} \embt(TL, or the Type Language.)  Since TL
(Type Language) will be used in the formal specifications of TON
blocks, transactions, and network datagrams, it warrants a brief
discussion.

TL is a language suitable for description of dependent algebraic {\em
  types}, which are allowed to have numeric (natural) and type
parameters. Each type is described by means of several {\em
  constructors}. Each constructor has a (human-readable) identifier
and a {\em name,} which is a bit string (32-bit integer by
default). Apart from that, the definition of a constructor contains a
list of fields along with their types.

A collection of constructor and type definitions is called a {\em
  TL-scheme}. It is usually kept in one or several files with the
suffix \texttt{.tl}.

An important feature of TL-schemes is that they determine an
unambiguous way of serializing and deserializing values (or objects)
of algebraic types defined. Namely, when a value needs to be
serialized into a stream of bytes, first the name of the constructor
used for this value is serialized. Recursively computed serializations
of each field follow.

The description of a previous version of TL, suitable for serializing
arbitrary objects into sequences of 32-bit integers, is available at
\url{https://core.telegram.org/mtproto/TL}. A new version of TL,
called {\em TL-B}, is being developed for the purpose of describing
the serialization of objects used by the TON Project. This new version
can serialize objects into streams of bytes and even bits (not just
32-bit integers), and offers support for serialization into a tree of
TVM cells (cf.~\ptref{sp:tvm.cells}). A description of TL-B will be a
part of the formal specification of the TON Blockchain.

\nxsubpoint\label{sp:blk.transf} \embt(Blocks and transactions as
state transformation operators.)  Normally, any blockchain (type)
$\Blockchain$ has an associated global state (type) $\State$, and a
transaction (type) $\Transaction$. The semantics of a blockchain are
to a large extent determined by the transaction application function:
\begin{equation}
  \evtrans':\Transaction\times\State\to\State^?
\end{equation}
Here $X^?$ denotes $\Maybe X$, the result of applying the $\Maybe$
monad to type $X$. This is similar to our use of $X^*$ for $\List
X$. Essentially, a value of type $X^?$ is either a value of type $X$
or a special value $\bot$ indicating the absence of an actual value
(think about a null pointer). In our case, we use $\State^?$ instead
of $\State$ as the result type because a transaction may be invalid if
invoked from certain original states (think about attempting to
withdraw from an account more money than it is actually there).

We might prefer a curried version of $\evtrans'$:
\begin{equation}
  \evtrans:\Transaction\to\State\to\State^?
\end{equation}

Because a block is essentially a list of transactions, the block
evaluation function
\begin{equation}
  \evblock:\Block\to\State\to\State^?
\end{equation}
can be derived from $\evtrans$. It takes a block $B:\Block$ and the
previous blockchain state $s:\State$ (which might include the hash of
the previous block) and computes the next blockchain state
$s'=\evblock(B)(s):\State$, which is either a true state or a special
value $\bot$ indicating that the next state cannot be computed (i.e.,
that the block is invalid if evaluated from the starting state
given---for example, the block includes a transaction trying to debit
an empty account.)

\nxsubpoint \embt(Block sequence numbers.)  Each block $B$ in the
blockchain can be referred to by its {\em sequence number}
$\blkseqno(B)$, starting from zero for the very first block, and
incremented by one whenever passing to the next block. More formally,
\begin{equation}
  \blkseqno(B)=\blkseqno\bigl(\blkprev(B)\bigr)+1
\end{equation}
Notice that the sequence number does not identify a block uniquely in
the presence of {\em forks}.

\nxsubpoint \embt(Block hashes.)  Another way of referring to a block
$B$ is by its hash $\blkhash(B)$, which is actually the hash of the
            {\em header\/} of block $B$ (however, the header of the
            block usually contains hashes that depend on all content
            of block $B$). Assuming that there are no collisions for
            the hash function used (or at least that they are very
            improbable), a block is uniquely identified by its hash.

\nxsubpoint \embt(Hash assumption.)  During formal analysis of
blockchain algorithms, we assume that there are no collisions for the
$k$-bit hash function $\Hash:\Bytes^*\to\st2^{k}$ used:
\begin{equation}\label{eq:hash.coll}
  \Hash(s)=\Hash(s')\Rightarrow s=s'\quad\text{for any $s$,
    $s'\in\Bytes^*$}
\end{equation}
Here $\Bytes=\{0\ldots255\}=\st2^8$ is the type of bytes, or the set
of all byte values, and $\Bytes^*$ is the type or set of arbitrary
(finite) lists of bytes; while $\st2=\{0,1\}$ is the bit type, and
$\st2^k$ is the set (or actually the type) of all $k$-bit sequences
(i.e., of $k$-bit numbers).

Of course, \eqref{eq:hash.coll} is impossible mathematically, because
a map from an infinite set to a finite set cannot be injective. A more
rigorous assumption would be
\begin{equation}\label{eq:hash.coll.prec}
  \forall s, s': s\neq s', P\bigl(\Hash(s)=\Hash(s')\bigr)=2^{-k}
\end{equation}
However, this is not so convenient for the proofs. If
\eqref{eq:hash.coll.prec} is used at most $N$ times in a proof with
$2^{-k}N<\epsilon$ for some small $\epsilon$ (say,
$\epsilon=10^{-18}$), we can reason as if \eqref{eq:hash.coll} were
true, provided we accept a failure probability $\epsilon$ (i.e., the
final conclusions will be true with probability at least
$1-\epsilon$).

Final remark: in order to make the probability statement
of~\eqref{eq:hash.coll.prec} really rigorous, one must introduce a
probability distribution on the set $\Bytes^*$ of all byte
sequences. A way of doing this is by assuming all byte sequences of
the same length $l$ equiprobable, and setting the probability of
observing a sequence of length $l$ equal to $p^l-p^{l+1}$ for some
$p\to1-$. Then \eqref{eq:hash.coll.prec} should be understood as a
limit of conditional probability $P\bigl(\Hash(s)=\Hash(s')|s\neq
s'\bigr)$ when $p$ tends to one from below.

\nxsubpoint\label{sp:hash.change} \embt(Hash used for the TON
Blockchain.)  We are using the 256-bit $\Sha$ hash for the TON
Blockchain for the time being. If it turns out to be weaker than
expected, it can be replaced by another hash function in the
future. The choice of the hash function is a configurable parameter of
the protocol, so it can be changed without hard forks as explained
in~\ptref{sp:config.param}.

\mysubsection{Blockchain State, Accounts and Hashmaps}

We have noted above that any blockchain defines a certain global
state, and each block and each transaction defines a transformation of
this global state. Here we describe the global state used by TON
blockchains.

\nxsubpoint \embt(Account IDs.)  The basic account IDs used by TON
blockchains---or at least by its masterchain and Workchain Zero---are
256-bit integers, assumed to be public keys for 256-bit Elliptic Curve
Cryptography (ECC) for a specific elliptic curve. In this way,
\begin{equation}
  \accountid:\Account=\uint_{256}=\st2^{256}
\end{equation}
Here $\Account$ is the account {\em type}, while $\accountid:\Account$
is a specific variable of type $\Account$.

Other workchains can use other account ID formats, 256-bit or
otherwise. For example, one can use Bitcoin-style account IDs, equal
to $\Sha$ of an ECC public key.

However, the bit length $l$ of an account ID must be fixed during the
creation of the workchain (in the masterchain), and it must be at
least 64, because the first 64 bits of $\accountid$ are used for
sharding and message routing.

\nxsubpoint \embt(Main component: {\em Hashmaps}.)  The principal
component of the TON blockchain state is a {\em hashmap}. In some
cases we consider (partially defined) ``maps''
$h:\st2^n\dashrightarrow\st2^m$. More generally, we might be
interested in hashmaps $h:\st2^n\dashrightarrow X$ for a composite
type $X$. However, the source (or index) type is almost always
$\st2^n$.

Sometimes, we have a ``default value'' $\vr{empty}:X$, and the hashmap
$h:\st2^n\to X$ is ``initialized'' by its ``default value''
$i\mapsto\vr{empty}$.

\nxsubpoint \embt(Example: TON account balances.)  An important
example is given by TON account balances. It is a hashmap
\begin{equation}
  \vr{balance}:\Account\to\uint_{128}
\end{equation}
mapping $\Account=\st2^{256}$ into a Gram (TON coin) balance of type
$\uint_{128}=\st2^{128}$. This hashmap has a default value of zero,
meaning that initially (before the first block is processed) the
balance of all accounts is zero.

\nxsubpoint \embt(Example: smart-contract persistent storage.)
Another example is given by smart-contract persistent storage, which
can be (very approximately) represented as a hashmap
\begin{equation}
  \vr{storage}:\st2^{256}\dashrightarrow\st2^{256}
\end{equation}
This hashmap also has a default value of zero, meaning that
uninitialized cells of persistent storage are assumed to be zero.

\nxsubpoint \embt(Example: persistent storage of all smart contracts.)
Because we have more than one smart contract, distinguished by
$\accountid$, each having its separate persistent storage, we must
actually have a hashmap
\begin{equation}
  \vr{Storage}:\Account\dashrightarrow(\st2^{256}\dashrightarrow\st2^{256})
\end{equation}
mapping $\accountid$ of a smart contract into its persistent storage.

\nxsubpoint \embt(Hashmap type.)  The hashmap is not just an abstract
(partially defined) function $\st2^n\dashrightarrow X$; it has a
specific representation. Therefore, we suppose that we have a special
hashmap type
\begin{equation}
  \Hashmap (n,X):\Type
\end{equation}
corresponding to a data structure encoding a (partial) map
$\st2^n\dashrightarrow X$.  We can also write
\begin{equation}
  \Hashmap (n:\nat) (X:\Type) : \Type
\end{equation}
or
\begin{equation}
  \Hashmap:\nat\to\Type\to\Type
\end{equation}
We can always transform $h:\Hashmap(n,X)$ into a map
$\hget(h):\st2^n\to X^?$. Henceforth, we usually write $h[i]$ instead
of $\hget(h)(i)$:
\begin{equation}
  h[i]:\equiv\hget(h)(i):X^?\quad\text{for any $i:\st2^n$,
    $h:\Hashmap(n,X)$}
\end{equation}

\nxsubpoint\label{sp:patricia} \embt(Definition of hashmap type as a
Patricia tree.)  Logically, one might define $\Hashmap(n,X)$ as an
(incomplete) binary tree of depth $n$ with edge labels $0$ and $1$ and
with values of type $X$ in the leaves. Another way to describe the
same structure would be as a {\em (bitwise) trie\/} for binary strings
of length equal to $n$.

In practice, we prefer to use a compact representation of this trie,
by compressing each vertex having only one child with its parent. The
resulting representation is known as a {\em Patricia tree\/} or a {\em
  binary radix tree\/}. Each intermediate vertex now has exactly two
children, labeled by two non-empty binary strings, beginning with zero
for the left child and with one for the right child.

In other words, there are two types of (non-root) nodes in a Patricia
tree:
\begin{itemize}
\item $\leaf(x)$, containing value $x$ of type $X$.
\item $\node(l,s_l,r,s_r)$, where $l$ is the (reference to the) left
  child or subtree, $s_l$ is the bitstring labeling the edge
  connecting this vertex to its left child (always beginning with 0),
  $r$ is the right subtree, and $s_r$ is the bitstring labeling the
  edge to the right child (always beginning with 1).
\end{itemize}
A third type of node, to be used only once at the root of the Patricia
tree, is also necessary:
\begin{itemize}
\item $\root(n,s_0,t)$, where $n$ is the common length of index
  bitstrings of $\Hashmap(n,X)$, $s_0$ is the common prefix of all
  index bitstrings, and $t$ is a reference to a $\leaf$ or a $\node$.
\end{itemize}
If we want to allow the Patricia tree to be empty, a fourth type of
(root) node would be used:
\begin{itemize}
\item $\emptyroot(n)$, where $n$ is the common length of all index
  bitstrings.
\end{itemize}

We define the height of a Patricia tree by
\begin{align}
  \height(\leaf(x))&=0\\ \height\bigl(\node(l,s_l,r,s_r)\bigr)&=\height(l)+\len(s_l)=\height(r)+\len(s_r)\\ \height\bigl(\root(n,s_0,t)\bigr)&=\len(s_0)+\height(t)=n
\end{align}
The last two expressions in each of the last two formulas must be
equal. We use Patricia trees of height $n$ to represent values of type
$\Hashmap(n,X)$.

If there are $N$ leaves in the tree (i.e., our hashmap contains $N$
values), then there are exactly $N-1$ intermediate vertices. Inserting
a new value always involves splitting an existing edge by inserting a
new vertex in the middle and adding a new leaf as the other child of
this new vertex. Deleting a value from a hashmap does the opposite: a
leaf and its parent are deleted, and the parent's parent and its other
child become directly linked.

\nxsubpoint\label{sp:merkle.patr.hash} \embt(Merkle-Patricia trees.)
When working with blockchains, we want to be able to compare Patricia
trees (i.e., hash maps) and their subtrees, by reducing them to a
single hash value. The classical way of achieving this is given by the
Merkle tree. Essentially, we want to describe a way of hashing objects
$h$ of type $\Hashmap(n,X)$ with the aid of a hash function $\Hash$
defined for binary strings, provided we know how to compute hashes
$\Hash(x)$ of objects $x:X$ (e.g., by applying the hash function
$\Hash$ to a binary serialization of object~$x$).

One might define $\Hash(h)$ recursively as follows:
\begin{align}\label{eq:hash.leaf}
  \Hash\bigl(\leaf(x)\bigr):=&\Hash(x)\\
  \label{eq:hash.node}
  \Hash\bigl(\node(l,s_l,r,s_r)\bigr):=&\Hash\bigl(\Hash(l).\Hash(r).\code(s_l).\code(s_r)\bigr)\\ \Hash\bigl(\root(n,s_0,t)\bigr):=&\Hash\bigl(\code(n).\code(s_0).\Hash(t)\bigr)
\end{align}
Here $s.t$ denotes the concatenation of (bit) strings $s$ and $t$, and
$\code(s)$ is a prefix code for all bit strings $s$. For example, one
might encode 0 by 10, 1 by 11, and the end of the string by 0.%
\footnote{One can show that this encoding is optimal for approximately
  half of all edge labels of a Patricia tree with random or
  consecutive indices. Remaining edge labels are likely to be long
  (i.e., almost 256 bits long). Therefore, a nearly optimal encoding
  for edge labels is to use the above code with prefix 0 for ``short''
  bit strings, and encode 1, then nine bits containing length $l=|s|$
  of bitstring~$s$, and then the $l$ bits of $s$ for ``long''
  bitstrings (with $l\geq10$).}

We will see later (cf.~\ptref{sp:pec.tvm} and \ptref{sp:tvm.cells})
that this is a (slightly tweaked) version of recursively defined
hashes for values of arbitrary (dependent) algebraic types.

\nxsubpoint \embt(Recomputing Merkle tree hashes.)  This way of
recursively defining $\Hash(h)$, called a {\em Merkle tree hash}, has
the advantage that, if one explicitly stores $\Hash(h')$ along with
each node $h'$ (resulting in a structure called a {\em Merkle tree},
or, in our case, a {\em Merkle--Patricia tree}), one needs to
recompute only at most $n$ hashes when an element is added to, deleted
from or changed in the hashmap.

In this way, if one represents the global blockchain state by a
suitable Merkle tree hash, it is easy to recompute this state hash
after each transaction.

\nxsubpoint\label{sp:merkle.proof} \embt(Merkle proofs.)  Under the
assumption \eqref{eq:hash.coll} of ``injectivity'' of the chosen hash
function $\Hash$, one can construct a proof that, for a given value
$z$ of $\Hash(h)$, $h:\Hashmap(n,X)$, one has $\hget(h)(i)=x$ for some
$i:\st2^n$ and $x:X$. Such a proof will consist of the path in the
Merkle--Patricia tree from the leaf corresponding to $i$ to the root,
augmented by the hashes of all siblings of all nodes occurring on this
path.

In this way, a light node%
\footnote{A {\em light node\/} is a node that does not keep track of
  the full state of a shardchain; instead, it keeps minimal
  information such as the hashes of the several most recent blocks,
  and relies on information obtained from full nodes when it becomes
  necessary to inspect some parts of the full state.} %
knowing only the value of $\Hash(h)$ for some hashmap $h$ (e.g.,
smart-contract persistent storage or global blockchain state) might
request from a full node%
\footnote{A {\em full node\/} is a node keeping track of the complete
  up-to-date state of the shardchain in question.} %
not only the value $x=h[i]=\hget(h)(i)$, but such a
value along with a Merkle proof starting from the already known value
$\Hash(h)$. Then, under assumption \eqref{eq:hash.coll}, the light
node can check for itself that $x$ is indeed the correct value of
$h[i]$.

In some cases, the client may want to obtain the value
$y=\Hash(x)=\Hash(h[i])$ instead---for example, if $x$ itself is very
large (e.g., a hashmap itself). Then a Merkle proof for $(i,y)$ can be
provided instead. If $x$ is a hashmap as well, then a second Merkle
proof starting from $y=\Hash(x)$ may be obtained from a full node, to
provide a value $x[j]=h[i][j]$ or just its hash.

\nxsubpoint \embt(Importance of Merkle proofs for a multi-chain system
such as TON.)  Notice that a node normally cannot be a full node for
all shardchains existing in the TON environment. It usually is a full
node only for some shardchains---for instance, those containing its
own account, a smart contract it is interested in, or those that this
node has been assigned to be a validator of. For other shardchains, it
must be a light node---otherwise the storage, computing and network
bandwidth requirements would be prohibitive. This means that such a
node cannot directly check assertions about the state of other
shardchains; it must rely on Merkle proofs obtained from full nodes
for those shardchains, which is as safe as checking by itself unless
\eqref{eq:hash.coll} fails (i.e., a hash collision is found).

\nxsubpoint\label{sp:pec.tvm} \embt(Peculiarities of TON VM.)  The TON
VM or TVM (Telegram Virtual Machine), used to run smart contracts in
the masterchain and Workchain Zero, is considerably different from
customary designs inspired by the EVM (Ethereum Virtual Machine): it
works not just with 256-bit integers, but actually with (almost)
arbitrary ``records'', ``structures'', or ``sum-product types'',
making it more suitable to execute code written in high-level
(especially functional) languages. Essentially, TVM uses tagged data
types, not unlike those used in implementations of Prolog or Erlang.

One might imagine first that the state of a TVM smart contract is not
just a hashmap $\st2^{256}\to\st2^{256}$, or
$\Hashmap(256,\st2^{256})$, but (as a first step) $\Hashmap(256,X)$,
where $X$ is a type with several constructors, enabling it to store,
apart from 256-bit integers, other data structures, including other
hashmaps $\Hashmap(256,X)$ in particular. This would mean that a cell
of TVM (persistent or temporary) storage---or a variable or an element
of an array in a TVM smart-contract code---may contain not only an
integer, but a whole new hashmap. Of course, this would mean that a
cell contains not just 256 bits, but also, say, an 8-bit tag,
describing how these 256 bits should be interpreted.

In fact, values do not need to be precisely 256-bit. The value format
used by TVM consists of a sequence of raw bytes and references to
other structures, mixed in arbitrary order, with some descriptor bytes
inserted in suitable locations to be able to distinguish pointers from
raw data (e.g., strings or integers); cf.~\ptref{sp:tvm.cells}.

This raw value format may be used to implement arbitrary sum-product
algebraic types. In this case, the value would contain a raw byte
first, describing the ``constructor'' being used (from the perspective
of a high-level language), and then other ``fields'' or ``constructor
arguments'', consisting of raw bytes and references to other
structures depending on the constructor chosen
(cf.~\ptref{sp:TL}). However, TVM does not know anything about the
correspondence between constructors and their arguments; the mixture
of bytes and references is explicitly described by certain descriptor
bytes.\footnote{These two descriptor bytes, present in any TVM cell,
  describe only the total number of references and the total number of
  raw bytes; references are kept together either before or after all
  raw bytes.}

The Merkle tree hashing is extended to arbitrary such structures: to
compute the hash of such a structure, all references are recursively
replaced by hashes of objects referred to, and then the hash of the
resulting byte string (descriptor bytes included) is computed.

In this way, the Merkle tree hashing for hashmaps, described in
\ptref{sp:merkle.patr.hash}, is just a special case of hashing for
arbitrary (dependent) algebraic data types, applied to type
$\Hashmap(n,X)$ with two constructors.\footnote{Actually, $\leaf$ and
  $\node$ are constructors of an auxiliary type,
  $\tp{HashmapAux}(n,X)$. Type $\Hashmap(n,X)$ has constructors
  $\root$ and $\emptyroot$, with $\root$ containing a value of type
  $\tp{HashmapAux}(n,X)$.}

\nxsubpoint \embt(Persistent storage of TON smart contracts.)
Persistent storage of a TON smart contract essentially consists of its
``global variables'', preserved between calls to the smart
contract. As such, it is just a ``product'', ``tuple'', or ``record''
type, consisting of fields of the correct types, corresponding to one
global variable each. If there are too many global variables, they
cannot fit into one TON cell because of the global restriction on TON
cell size. In such a case, they are split into several records and
organized into a tree, essentially becoming a ``product of products''
or ``product of products of products'' type instead of just a product
type.

\nxsubpoint\label{sp:tvm.cells} \embt(TVM Cells.)  Ultimately, the TON
VM keeps all data in a collection of {\em (TVM) cells}. Each cell
contains two descriptor bytes first, indicating how many bytes of raw
data are present in this cell (up to 128) and how many references to
other cells are present (up to four). Then these raw data bytes and
references follow. Each cell is referenced exactly once, so we might
have included in each cell a reference to its ``parent'' (the only
cell referencing this one). However, this reference need not be
explicit.

In this way, the persistent data storage cells of a TON smart contract
are organized into a tree,\footnote{Logically; the ``bag of cells''
  representation described in~\ptref{sp:bag.of.cells} identifies all
  duplicate cells, transforming this tree into a directed acyclic
  graph (dag) when serialized.} with a reference to the root of this
tree kept in the smart-contract description. If necessary, a Merkle
tree hash of this entire persistent storage is recursively computed,
starting from the leaves and then simply replacing all references in a
cell with the recursively computed hashes of the referenced cells, and
subsequently computing the hash of the byte string thus obtained.

\nxsubpoint\label{sp:gen.merkle.proof} \embt(Generalized Merkle proofs
for values of arbitrary algebraic types.)  Because the TON VM
represents a value of arbitrary algebraic type by means of a tree
consisting of (TVM) cells, and each cell has a well-defined
(recursively computed) Merkle hash, depending in fact on the whole
subtree rooted in this cell, we can provide ``generalized Merkle
proofs'' for (parts of) values of arbitrary algebraic types, intended
to prove that a certain subtree of a tree with a known Merkle hash
takes a specific value or a value with a specific hash. This
generalizes the approach of \ptref{sp:merkle.proof}, where only Merkle
proofs for $x[i]=y$ have been considered.

\nxsubpoint\label{sp:tvm.data.shard} \embt(Support for sharding in TON
VM data structures.)  We have just outlined how the TON VM, without
being overly complicated, supports arbitrary (dependent) algebraic
data types in high-level smart-contract languages. However, sharding
of large (or global) smart contracts requires special support on the
level of TON VM. To this end, a special version of the hashmap type
has been added to the system, amounting to a ``map''
$\Account\dashrightarrow X$. This ``map'' might seem equivalent to
$\Hashmap(m,X)$, where $\Account=\st2^m$. However, when a shard is
split in two, or two shards are merged, such hashmaps are
automatically split in two, or merged back, so as to keep only those
keys that belong to the corresponding shard.

\nxsubpoint \embt(Payment for persistent storage.)  A noteworthy
feature of the TON Blockchain is the payment exacted from smart
contracts for storing their persistent data (i.e., for enlarging the
total state of the blockchain). It works as follows:

Each block declares two rates, nominated in the principal currency of
the blockchain (usually the Gram): the price for keeping one cell in
the persistent storage, and the price for keeping one raw byte in some
cell of the persistent storage. Statistics on the total numbers of
cells and bytes used by each account are stored as part of its state,
so by multiplying these numbers by the two rates declared in the block
header, we can compute the payment to be deducted from the account
balance for keeping its data between the previous block and the
current one.

However, payment for persistent storage usage is not exacted for every
account and smart contract in each block; instead, the sequence number
of the block where this payment was last exacted is stored in the
account data, and when any action is done with the account (e.g., a
value transfer or a message is received and processed by a smart
contract), the storage usage payment for all blocks since the previous
such payment is deducted from the account balance before performing
any further actions. If the account's balance would become negative
after this, the account is destroyed.

A workchain may declare some number of raw data bytes per account to
be ``free'' (i.e., not participating in the persistent storage
payments) in order to make ``simple'' accounts, which keep only their
balance in one or two cryptocurrencies, exempt from these constant
payments.

Notice that, if nobody sends any messages to an account, its
persistent storage payments are not collected, and it can exist
indefinitely. However, anybody can send, for instance, an empty
message to destroy such an account. A small incentive, collected from
part of the original balance of the account to be destroyed, can be
given to the sender of such a message. We expect, however, that the
validators would destroy such insolvent accounts for free, simply to
decrease the global blockchain state size and to avoid keeping large
amounts of data without compensation.

Payments collected for keeping persistent data are distributed among
the validators of the shardchain or the masterchain (proportionally to
their stakes in the latter case).

\nxsubpoint\label{sp:loc.glob.smct} \embt(Local and global smart
contracts; smart-contract instances.)  A smart contract normally
resides just in one shard, selected according to the smart contract's
$\accountid$, similarly to an ``ordinary'' account. This is usually
sufficient for most applications. However, some ``high-load'' smart
contracts may want to have an ``instance'' in each shardchain of some
workchain. To achieve this, they must propagate their creating
transaction into all shardchains, for instance, by committing this
transaction into the ``root'' shardchain $(w,\emptyset)$\footnote{A
  more expensive alternative is to publish such a ``global'' smart
  contract in the masterchain.} of the workchain $w$ and paying a
large commission.\footnote{This is a sort of ``broadcast'' feature for
  all shards, and as such, it must be quite expensive.}

This action effectively creates instances of the smart contract in
each shard, with separate balances. Originally, the balance
transferred in the creating transaction is distributed simply by
giving the instance in shard $(w,s)$ the $2^{-|s|}$ part of the total
balance. When a shard splits into two child shards, balances of all
instances of global smart contracts are split in half; when two shards
merge, balances are added together.

In some cases, splitting/merging instances of global smart contracts
may involve (delayed) execution of special methods of these smart
contracts. By default, the balances are split and merged as described
above, and some special ``account-indexed'' hashmaps are also
automatically split and merged (cf.~\ptref{sp:tvm.data.shard}).

\nxsubpoint \embt(Limiting splitting of smart contracts.)  A global
smart contract may limit its splitting depth $d$ upon its creation, in
order to make persistent storage expenses more predictable. This means
that, if shardchain $(w,s)$ with $|s|\geq d$ splits in two, only one
of two new shardchains inherits an instance of the smart
contract. This shardchain is chosen deterministically: each global
smart contract has some ``$\accountid$'', which is essentially the
hash of its creating transaction, and its instances have the same
$\accountid$ with the first $\leq d$ bits replaced by suitable values
needed to fall into the correct shard. This $\accountid$ selects which
shard will inherit the smart-contract instance after splitting.

\nxsubpoint\label{sp:account.state} \embt(Account/Smart-contract
state.)  We can summarize all of the above to conclude that an account
or smart-contract state consists of the following:
\begin{itemize}
\item A balance in the principal currency of the blockchain
\item A balance in other currencies of the blockchain
\item Smart-contract code (or its hash)
\item Smart-contract persistent data (or its Merkle hash)
\item Statistics on the number of persistent storage cells and raw
  bytes used
\item The last time (actually, the masterchain block number) when
  payment for smart-contract persistent storage was collected
\item The public key needed to transfer currency and send messages
  from this account (optional; by default equal to $\accountid$
  itself). In some cases, more sophisticated signature checking code
  may be located here, similar to what is done for Bitcoin transaction
  outputs; then the $\accountid$ will be equal to the hash of this
  code.
\end{itemize}
We also need to keep somewhere, either in the account state or in some
other account-indexed hashmap, the following data:
\begin{itemize}
\item The output message queue of the account
  (cf.~\ptref{sp:out.queue})
\item The collection of (hashes of) recently delivered messages
  (cf.~\ptref{sp:deliver.q})
\end{itemize}

Not all of these are really required for every account; for example,
smart-contract code is needed only for smart contracts, but not for
``simple'' accounts. Furthermore, while any account must have a
non-zero balance in the principal currency (e.g., Grams for the
masterchain and shardchains of the basic workchain), it may have
balances of zero in other currencies. In order to avoid keeping unused
data, a sum-product type (depending on the workchain) is defined
(during the workchain's creation), which uses different tag bytes
(e.g., TL constructors; cf.~\ptref{sp:TL}) to distinguish between
different ``constructors'' used. Ultimately, the account state is
itself kept as a collection of cells of the TVM persistent storage.

\mysubsection{Messages Between Shardchains}

An important component of the TON Blockchain is the {\em messaging
  system\/} between blockchains. These blockchains may be shardchains
of the same workchain, or of different workchains.

\nxsubpoint \embt(Messages, accounts and transactions: a bird's eye
view of the system.)  {\em Messages\/} are sent from one account to
another. Each {\em transaction\/} consists of an account receiving one
message, changing its state according to certain rules, and generating
several (maybe one or zero) new messages to other accounts. Each
message is generated and received (delivered) exactly once.

This means that messages play a fundamental role in the system,
comparable to that of accounts (smart contracts). From the perspective
of the Infinite Sharding Paradigm (cf.~\ptref{sp:ISP}), each account
resides in its separate ``account-chain'', and the only way it can
affect the state of some other account is by sending a message.

\nxsubpoint\label{sp:actors} \embt(Accounts as processes or actors;
Actor model.)  One might think about accounts (and smart contracts) as
``processes'', or ``actors'', that are able to process incoming
messages, change their internal state and generate some outbound
messages as a result. This is closely related to the so-called {\em
  Actor model}, used in languages such as Erlang (however, actors in
Erlang are usually called ``processes''). Since new actors (i.e.,
smart contracts) are also allowed to be created by existing actors as
a result of processing an inbound message, the correspondence with the
Actor model is essentially complete.

\nxsubpoint \embt(Message recipient.)  Any message has its {\em
  recipient}, characterized by the {\em target workchain identifier
  $w$} (assumed by default to be the same as that of the originating
shardchain), and the {\em recipient account $\accountid$}. The exact
format (i.e., number of bits) of $\accountid$ depends on $w$; however,
the shard is always determined by its first (most significant) 64
bits.

\nxsubpoint\label{sp:msg.sender} \embt(Message sender.)  In most
cases, a message has a {\em sender}, characterized again by a
$(w',\accountid')$ pair. If present, it is located after the message
recipient and message value. Sometimes, the sender is unimportant or
it is somebody outside the blockchain (i.e., not a smart contract), in which case this field is absent.

Notice that the Actor model does not require the messages to have an
implicit sender. Instead, messages may contain a reference to the
Actor to which an answer to the request should be sent; usually it
coincides with the sender. However, it is useful to have an explicit
unforgeable sender field in a message in a cryptocurrency (Byzantine)
environment.

\nxsubpoint \embt(Message value.)  Another important characteristic of
a message is its attached {\em value}, in one or several
cryptocurrencies supported both by the source and by the target
workchain. The value of the message is indicated at its very beginning
immediately after the message recipient; it is essentially a list of
$(\currencyid,\vr{value})$ pairs.

Notice that ``simple'' value transfers between ``simple'' accounts are
just empty (no-op) messages with some value attached to them. On the
other hand, a slightly more complicated message body might contain a
simple text or binary comment (e.g., about the purpose of the
payment).

\nxsubpoint\label{sp:ext.msg} \embt(External messages, or ``messages
from nowhere''.)  Some messages arrive into the system ``from
nowhere''---that is, they are not generated by an account (smart
contract or not) residing in the blockchain. The most typical example
arises when a user wants to transfer some funds from an account
controlled by her to some other account. In this case, the user sends
a ``message from nowhere'' to her own account, requesting it to
generate a message to the receiving account, carrying the specified
value. If this message is correctly signed, her account receives it
and generates the required outbound messages.

In fact, one might consider a ``simple'' account as a special case of
a smart contract with predefined code. This smart contract receives
only one type of message. Such an inbound message must contain a list
of outbound messages to be generated as a result of delivering
(processing) the inbound message, along with a signature. The smart
contract checks the signature, and, if it is correct, generates the
required messages.

Of course, there is a difference between ``messages from nowhere'' and
normal messages, because the ``messages from nowhere'' cannot bear
value, so they cannot pay for their ``gas'' (i.e., their processing)
themselves. Instead, they are tentatively executed with a small gas
limit before even being suggested for inclusion in a new shardchain
block; if the execution fails (the signature is incorrect), the
``message from nowhere'' is deemed incorrect and is discarded. If the
execution does not fail within the small gas limit, the message may be
included in a new shardchain block and processed completely, with the
payment for the gas (processing capacity) consumed exacted from the
receiver's account. ``Messages from nowhere'' can also define some
transaction fee which is deducted from the receiver's account on top
of the gas payment for redistribution to the validators.

In this sense, ``messages from nowhere'' or ``external messages'' take
the role of transaction candidates used in other blockchain systems
(e.g., Bitcoin and Ethereum).

\nxsubpoint \embt(Log messages, or ``messages to nowhere''.)
Similarly, sometimes a special message can be generated and routed to
a specific shardchain not to be delivered to its recipient, but to be
logged in order to be easily observable by anybody receiving updates
about the shard in question. These logged messages may be output in a
user's console, or trigger an execution of some script on an off-chain
server. In this sense, they represent the external ``output'' of the
``blockchain supercomputer'', just as the ``messages from nowhere''
represent the external ``input'' of the ``blockchain supercomputer''.

\nxsubpoint \embt(Interaction with off-chain services and external
blockchains.)  These external input and output messages can be used
for interacting with off-chain services and other (external)
blockchains, such as Bitcoin or Ethe\-reum. One might create tokens or
cryptocurrencies inside the TON Block\-chain pegged to Bitcoins,
Ethers or any ERC-20 tokens defined in the Ethe\-reum blockchain, and
use ``messages from nowhere'' and ``messages to nowhere'', generated
and processed by scripts residing on some third-party off-chain
servers, to implement the necessary interaction between the TON
Blockchain and these external blockchains.

\nxsubpoint \embt(Message body.)  The {\em message body\/} is simply a
sequence of bytes, the meaning of which is determined only by the
receiving workchain and/or smart contract. For blockchains using TON
VM, this could be the serialization of any TVM cell, generated
automatically via the \texttt{Send()} operation. Such a serialization
is obtained simply by recursively replacing all references in a TON VM
cell with the cells referred to. Ultimately, a string of raw bytes
appears, which is usually prepended by a 4-byte ``message type'' or
``message constructor'', used to select the correct method of the
receiving smart contract.

Another option would be to use TL-serialized objects
(cf.~\ptref{sp:TL}) as message bodies. This might be especially useful
for communication between different workchains, one or both of which
are not necessarily using the TON VM.

\nxsubpoint \embt(Gas limit and other workchain/VM-specific
parameters.)  Sometimes a message needs to carry information about the
gas limit, the gas price, transaction fees and similar values that
depend on the receiving workchain and are relevant only for the
receiving workchain, but not necessarily for the originating
workchain. Such parameters are included in or before the message body,
sometimes (depending on the workchain) with special 4-byte prefixes
indicating their presence (which can be defined by a TL-scheme;
cf.~\ptref{sp:TL}).

\nxsubpoint \embt(Creating messages: smart contracts and
transactions.)  There are two sources of new messages. Most messages
are created during smart-contract execution (via the \texttt{Send()}
operation in TON VM), when some smart contract is invoked to process
an incoming message. Alternatively, messages may come from the outside
as ``external messages'' or ``messages from nowhere''
(cf.~\ptref{sp:ext.msg}).%
\footnote{The above needs to be literally true only for the basic
  workchain and its shardchains; other workchains may provide other
  ways of creating messages.}

\nxsubpoint \embt(Delivering messages.)  When a message reaches the
shardchain containing its destination account,\footnote{As a degenerate
  case, this shardchain may coincide with the originating shardchain---for example, if we are working inside a workchain which has not yet
  been split.} it is ``delivered'' to its destination account. What
happens next depends on the workchain; from an outside perspective, it
is important that such a message can never be forwarded further from
this shardchain.

For shardchains of the basic workchain, delivery consists in adding
the message value (minus any gas payments) to the balance of the
receiving account, and possibly in invoking a message-dependent method
of the receiving smart contract afterwards, if the receiving account
is a smart contract. In fact, a smart contract has only one entry
point for processing all incoming messages, and it must distinguish
between different types of messages by looking at their first few
bytes (e.g., the first four bytes containing a TL constructor;
cf.~\ptref{sp:TL}).

\nxsubpoint \embt(Delivery of a message is a transaction.)  Because
the delivery of a message changes the state of an account or smart
contract, it is a special {\em transaction\/} in the receiving
shardchain, and is explicitly registered as such. Essentially, {\em
  all\/} TON Blockchain transactions consist in the delivery of one
inbound message to its receiving account (smart contract), neglecting
some minor technical details.

\nxsubpoint \embt(Messages between instances of the same smart
contract.)  Recall that a smart contract may be {\em local\/} (i.e.,
residing in one shardchain as any ordinary account does) or {\em
  global\/} (i.e., having instances in all shards, or at least in all
shards up to some known depth $d$;
cf.~\ptref{sp:loc.glob.smct}). Instances of a global smart contract
may exchange special messages to transfer information and value
between each other if required. In this case, the (unforgeable) sender
$\accountid$ becomes important (cf.~\ptref{sp:msg.sender}).

\nxsubpoint \embt(Messages to any instance of a smart contract;
wildcard addresses.)  Sometimes a message (e.g., a client request)
needs be delivered to any instance of a global smart contract, usually
the closest one (if there is one residing in the same shardchain as
the sender, it is the obvious candidate). One way of doing this is by
using a ``wildcard recipient address'', with the first $d$ bits of the
destination $\accountid$ allowed to take arbitrary values. In
practice, one will usually set these $d$ bits to the same values as in
the sender's $\accountid$.

\nxsubpoint \embt(Input queue is absent.)  All messages received by a
blockchain (usually a shardchain; sometimes the masterchain)---or,
essentially, by an ``account-chain'' residing inside some
shardchain---are immediately delivered (i.e., processed by the
receiving account). Therefore, there is no ``input queue'' as
such. Instead, if not all messages destined for a specific shardchain
can be processed because of limitations on the total size of blocks
and gas usage, some messages are simply left to accumulate in the
output queues of the originating shardchains.

\nxsubpoint\label{sp:out.queue} \embt(Output queues.)  From the
perspective of the Infinite Sharding Paradigm (cf.~\ptref{sp:ISP}),
each account-chain (i.e., each account) has its own output queue,
consisting of all messages it has generated, but not yet delivered to
their recipients. Of course, account-chains have only a virtual
existence; they are grouped into shardchains, and a shardchain has an
output ``queue'', consisting of the union of the output queues of all
accounts belonging to the shardchain.

This shardchain output ``queue'' imposes only partial order on its
member messages. Namely, a message generated in a preceding block must
be delivered before any message generated in a subsequent block, and
any messages generated by the same account and having the same
destination must be delivered in the order of their generation.

\nxsubpoint\label{sp:intershard.msgs} \embt(Reliable and fast
inter-chain messaging.)  It is of paramount importance for a scalable
multi-blockchain project such as TON to be able to forward and deliver
messages between different shardchains (cf.~\ptref{sp:msg.IHR}), even
if there are millions of them in the system. The messages should be
delivered {\em reliably\/} (i.e., messages should not be lost or
delivered more than once) and {\em quickly}. The TON Blockchain
achieves this goal by using a combination of two ``message routing''
mechanisms.

\nxsubpoint\label{sp:hypercube} \embt(Hypercube routing: ``slow path''
for messages with assured delivery.)  The TON Blockchain uses
``hypercube routing'' as a slow, but safe and reliable way of
delivering messages from one shardchain to another, using several
intermediate shardchains for transit if necessary.  Otherwise,
the validators of any given shardchain would need to keep track of the
state of (the output queues of) all other shardchains, which would
require prohibitive amounts of computing power and network bandwidth
as the total quantity of shardchains grows, thus limiting the
scalability of the system.  Therefore, it is not possible to deliver
messages directly from any shard to every other. Instead, each shard
is ``connected'' only to shards differing in exactly one hexadecimal
digit of their $(w,s)$ shard identifiers
(cf.~\ptref{sp:shard.ident}). In this way, all shardchains constitute
a ``hypercube'' graph, and messages travel along the edges of this
hypercube.

If a message is sent to a shard different from the current one, one of
the hexadecimal digits (chosen deterministically) of the current shard
identifier is replaced by the corresponding digit of the target shard,
and the resulting identifier is used as the proximate target to
forward the message to.\footnote{This is not necessarily the final
  version of the algorithm used to compute the next hop for hypercube
  routing. In particular, hexadecimal digits may be replaced by
  $r$-bit groups, with $r$ a configurable parameter, not necessarily
  equal to four.}

The main advantage of hypercube routing is that the block validity
conditions imply that validators creating blocks of a shardchain must
collect and process messages from the output queues of ``neighboring''
shardchains, on pain of losing their stakes. In this way, any message
can be expected to reach its final destination sooner or later; a
message cannot be lost in transit or delivered twice.

Notice that hypercube routing introduces some additional delays and
expenses, because of the necessity to forward messages through several
intermediate shardchains. However, the number of these intermediate
shardchains grows very slowly, as the logarithm $\log N$ (more
precisely, $\lceil\log_{16}N\rceil-1$) of the total number of
shardchains $N$. For example, if $N\approx250$, there will be at most
one intermediate hop; and for $N\approx4000$ shardchains, at most
two. With four intermediate hops, we can support up to one million
shardchains. We think this is a very small price to pay for the
essentially unlimited scalability of the system. In fact, it is not
necessary to pay even this price:

\nxsubpoint\label{sp:instant.hypercube} \embt(Instant Hypercube
Routing: ``fast path'' for messages.)  A novel feature of the TON
Blockchain is that it introduces a ``fast path'' for forwarding
messages from one shardchain to any other, allowing in most cases to
bypass the ``slow'' hypercube routing of \ptref{sp:hypercube}
altogether and deliver the message into the very next block of the
final destination shardchain.

The idea is as follows. During the ``slow'' hypercube routing, the
message travels (in the network) along the edges of the hypercube, but
it is delayed (for approximately five seconds) at each intermediate
vertex to be committed into the corresponding shardchain before
continuing its voyage.

To avoid unnecessary delays, one might instead relay the message along
with a suitable Merkle proof along the edges of the hypercube, without
waiting to commit it into the intermediate shardchains. In fact, the
network message should be forwarded from the validators of the ``task
group'' (cf.~\ptref{sp:val.task.grp}) of the original shard to the
designated block producer (cf.~\ptref{sp:rot.gen.prio}) of the ``task
group'' of the destination shard; this might be done directly without
going along the edges of the hypercube. When this message with the
Merkle proof reaches the validators (more precisely, the collators;
cf.~\ptref{sp:collators}) of the destination shardchain, they can
commit it into a new block immediately, without waiting for the
message to complete its travel along the ``slow path''. Then a
confirmation of delivery along with a suitable Merkle proof is sent
back along the hypercube edges, and it may be used to stop the travel
of the message along the ``slow path'', by committing a special
transaction.

Note that this ``instant delivery'' mechanism does not replace the
``slow'' but failproof mechanism described
in~\ptref{sp:hypercube}. The ``slow path'' is still needed because the
validators cannot be punished for losing or simply deciding not to
commit the ``fast path'' messages into new blocks of their
blockchains.\footnote{However, the validators have some incentive to do
  so as soon as possible, because they will be able to collect all
  forwarding fees associated with the message that have not yet been
  consumed along the slow path.}

Therefore, both message forwarding methods are run in parallel, and
the ``slow'' mechanism is aborted only if a proof of success of the
``fast'' mechanism is committed into an intermediate shardchain.%
\footnote{In fact, one might temporarily or permanently disable the
  ``instant delivery'' mechanism altogether, and the system would
  continue working, albeit more slowly.}

\nxsubpoint\label{sp:collect.input.msg} \embt(Collecting input
messages from output queues of neighboring shardchains.)  When a new
block for a shardchain is proposed, some of the output messages of the
neighboring (in the sense of the routing hypercube of
\ptref{sp:hypercube}) shardchains are included in the new block as
``input'' messages and immediately delivered (i.e., processed). There
are certain rules as to the order in which these neighbors' output
messages must be processed. Essentially, an ``older'' message (coming
from a shardchain block referring to an older masterchain block) must
be delivered before any ``newer'' message; and for messages coming
from the same neighboring shardchain, the partial order of the output
queue described in \ptref{sp:out.queue} must be observed.

\nxsubpoint\label{sp:out.q.del} \embt(Deleting messages from output
queues.)  Once an output queue message is observed as having been
delivered by a neighboring shardchain, it is explicitly deleted from
the output queue by a special transaction.

\nxsubpoint\label{sp:deliver.q} \embt(Preventing double delivery of
messages.)  To prevent double delivery of messages taken from the
output queues of the neighboring shardchains, each shardchain (more
precisely, each account-chain inside it) keeps the collection of
recently delivered messages (or just their hashes) as part of its
state. When a delivered message is observed to be deleted from the
output queue by its originating neighboring shardchain
(cf.~\ptref{sp:out.q.del}), it is deleted from the collection of
recently delivered messages as well.

\nxsubpoint \embt(Forwarding messages intended for other shardchains.)
Hypercube routing (cf.~\ptref{sp:hypercube}) means that sometimes
outbound messages are delivered not to the shardchain containing the
intended recipient, but to a neighboring shardchain lying on the
hypercube path to the destination. In this case, ``delivery'' consists
in moving the inbound message to the outbound queue. This is reflected
explicitly in the block as a special {\em forwarding transaction},
containing the message itself. Essentially, this looks as if the
message had been received by somebody inside the shardchain, and one
identical message had been generated as result.

\nxsubpoint \embt(Payment for forwarding and keeping a message.)  The
forwarding transaction actually spends some gas (depending on the size
of the message being forwarded), so a gas payment is deducted from the
value of the message being forwarded on behalf of the validators of
this shardchain. This forwarding payment is normally considerably
smaller than the gas payment exacted when the message is finally
delivered to its recipient, even if the message has been forwarded
several times because of hypercube routing. Furthermore, as long as a
message is kept in the output queue of some shardchain, it is part of
the shardchain's global state, so a payment for keeping global data
for a long time may be also collected by special transactions.

\nxsubpoint \embt(Messages to and from the masterchain.)  Messages can
be sent directly from any shardchain to the masterchain, and vice
versa. However, gas prices for sending messages to and for processing
messages in the masterchain are quite high, so this ability will be
used only when truly necessary---for example, by the validators to
deposit their stakes. In some cases, a minimal deposit (attached
value) for messages sent to the masterchain may be defined, which is
returned only if the message is deemed ``valid'' by the receiving
party.

Messages cannot be automatically routed through the masterchain. A
message with $\workchainid\neq-1$ ($-1$ being the special
$\workchainid$ indicating the masterchain) cannot be delivered to the
masterchain.

In principle, one can create a message-forwarding smart contract
inside the masterchain, but the price of using it would be
prohibitive.

\nxsubpoint \embt(Messages between accounts in the same shardchain.)
In some cases, a message is generated by an account belonging to some
shardchain, destined to another account in the same shardchain. For
example, this happens in a new workchain which has not yet split into
several shardchains because the load is manageable.

Such messages might be accumulated in the output queue of the
shardchain and then processed as incoming messages in subsequent
blocks (any shard is considered a neighbor of itself for this
purpose). However, in most cases it is possible to deliver these
messages within the originating block itself.

In order to achieve this, a partial order is imposed on all
transactions included in a shardchain block, and the transactions
(each consisting in the delivery of a message to some account) are
processed respecting this partial order. In particular, a transaction
is allowed to process some output message of a preceding transaction
with respect to this partial order.

In this case, the message body is not copied twice. Instead, the
originating and the processing transactions refer to a shared copy of
the message.

\mysubsection{Global Shardchain State. ``Bag of Cells'' Philosophy.}

Now we are ready to describe the global state of a TON blockchain, or
at least of a shardchain of the basic workchain.

We start with a ``high-level'' or ``logical'' description, which
consists in saying that the global state is a value of algebraic type
$\tp{ShardchainState}$.

\nxsubpoint\label{sp:shard.state} \embt(Shardchain state as a
collection of account-chain states.)  According to the Infinite
Sharding Paradigm (cf.~\ptref{sp:ISP}), any shardchain is just a
(temporary) collection of virtual ``account-chains'', containing
exactly one account each. This means that, essentially, the global
shardchain state must be a hashmap
\begin{equation}\label{eq:simple.shard.st}
  \tp{ShardchainState}:=(\Account\dashrightarrow\tp{AccountState})
\end{equation}
where all $\accountid$ appearing as indices of this hashmap must begin
with prefix $s$, if we are discussing the state of shard $(w,s)$
(cf.~\ptref{sp:shard.ident}).

In practice, we might want to split $\tp{AccountState}$ into several
parts (e.g., keep the account output message queue separate to
simplify its examination by the neighboring shardchains), and have
several hashmaps $(\Account\dashrightarrow\tp{AccountStatePart}_i)$
inside the $\tp{ShardchainState}$. We might also add a small number of
``global'' or ``integral'' parameters to the $\tp{ShardchainState}$,
(e.g., the total balance of all accounts belonging to this
shard, or the total number of messages in all output queues).

However, \eqref{eq:simple.shard.st} is a good first approximation of
what the shardchain global state looks like, at least from a
``logical'' (``high-level'') perspective. The formal description of
algebraic types $\tp{AccountState}$ and $\tp{ShardchainState}$ can be
done with the aid of a TL-scheme (cf.~\ptref{sp:TL}), to be provided
elsewhere.

\nxsubpoint\label{sp:split.merge.state} \embt(Splitting and merging
shardchain states.)  Notice that the Infinite Sharding Paradigm
description of the shardchain state \eqref{eq:simple.shard.st} shows
how this state should be processed when shards are split or merged. In
fact, these state transformations turn out to be very simple
operations with hashmaps.

\nxsubpoint \embt(Account-chain state.)  The (virtual) account-chain
state is just the state of one account, described by type
$\tp{AccountState}$. Usually it has all or some of the fields listed
in~\ptref{sp:account.state}, depending on the specific constructor
used.

\nxsubpoint \embt(Global workchain state.)  Similarly to
\eqref{eq:simple.shard.st}, we may define the global {\em workchain\/}
state by the same formula, but with $\accountid$'s allowed to take any
values, not just those belonging to one shard. Remarks similar to
those made in~\ptref{sp:shard.state} apply in this case as well: we
might want to split this hashmap into several hashmaps, and we might
want to add some ``integral'' parameters such as the total balance.

Essentially, the global workchain state {\em must\/} be given by the
same type $\tp{ShardchainState}$ as the shardchain state, because it
is the shardchain state we would obtain if all existing shardchains of
this workchain suddenly merged into one.

\nxsubpoint\label{sp:bag.of.cells} \embt(Low-level perspective: ``bag
of cells''.)  There is a ``low-level'' description of the
account-chain or shardchain state as well, complementary to the
``high-level'' description given above. This description is quite
important, because it turns out to be pretty universal, providing a
common basis for representing, storing, serializing and transferring
by network almost all data used by the TON Blockchain (blocks,
shardchain states, smart-contract storage, Merkle proofs, etc.). At
the same time, such a universal ``low-level'' description, once
understood and implemented, allows us to concentrate our attention on
the ``high-level'' considerations only.

Recall that the TVM represents values of arbitrary algebraic types
(including, for instance, $\tp{ShardchainState}$
of~\eqref{eq:simple.shard.st}) by means of a tree of {\em TVM cells},
or {\em cells\/} for short (cf.~\ptref{sp:tvm.cells}
and~\ptref{sp:TL}).

Any such cell consists of two {\em descriptor bytes}, defining certain
flags and values $0\leq b\leq 128$, the quantity of raw bytes, and
$0\leq c\leq 4$, the quantity of references to other cells. Then $b$
raw bytes and $c$ cell references follow.\footnote{One can show that, if Merkle proofs for all data stored in a tree of cells are needed equally often, one should use cells with $b+ch\approx 2(h+r)$ to minimize average Merkle proof size, where $h=32$ is the hash size in bytes, and $r\approx4$ is the ``byte size'' of a cell reference. In other words, a cell should contain either two references and a few raw bytes, or one reference and about 36 raw bytes, or no references at all with 72 raw bytes.}

The exact format of cell references depends on the implementation and
on whether the cell is located in RAM, on disk, in a network packet,
in a block, and so on. A useful abstract model consists in imagining
that all cells are kept in content-addressable memory, with the
address of a cell equal to its ($\Sha$) hash. Recall that the (Merkle)
hash of a cell is computed exactly by replacing the references to its
child cells by their (recursively computed) hashes and hashing the
resulting byte string.

In this way, if we use cell hashes to reference cells (e.g., inside
descriptions of other cells), the system simplifies somewhat, and the
hash of a cell starts to coincide with the hash of the byte string
representing it.

Now we see that {\em any object representable by TVM, the global
  shardchain state included, can be represented as a ``bag of
  cells''}---i.e., {\em a collection of cells along with a ``root''
  reference to one of them\/} (e.g., by hash). Notice that duplicate
cells are removed from this description (the ``bag of cells'' is a set
of cells, not a multiset of cells), so the abstract tree
representation might actually become a directed acyclic graph (dag)
representation.

One might even keep this state on disk in a $B$- or $B+$-tree,
containing all cells in question (maybe with some additional data,
such as subtree height or reference counter), indexed by cell
hash. However, a naive implementation of this idea would result in the
state of one smart contract being scattered among distant parts of the
disk file, something we would rather avoid.%
\footnote{A better implementation would be to keep the state of the
  smart contract as a serialized string, if it is small, or in a
  separate $B$-tree, if it is large; then the top-level structure
  representing the state of a blockchain would be a $B$-tree, whose
  leaves are allowed to contain references to other $B$-trees.}

Now we are going to explain in some detail how almost all objects used
by the TON Blockchain can be represented as ``bags of cells'', thus
demonstrating the universality of this approach.

\nxsubpoint \embt(Shardchain block as a ``bag of cells''.)  A
shardchain block itself can be also described by an algebraic type,
and stored as a ``bag of cells''. Then a naive binary representation
of the block may be obtained simply by concatenating the byte strings
representing each of the cells in the ``bag of cells'', in arbitrary
order. This representation might be improved and optimized, for
instance, by providing a list of offsets of all cells at the beginning
of the block, and replacing hash references to other cells with 32-bit
indices in this list whenever possible. However, one should imagine
that a block is essentially a ``bag of cells'', and all other
technical details are just minor optimization and implementation
issues.

\nxsubpoint\label{sp:obj.update} \embt(Update to an object as a ``bag
of cells''.)  Imagine that we have an old version of some object
represented as a ``bag of cells'', and that we want to represent a new
version of the same object, supposedly not too different from the
previous one. One might simply represent the new state as another
``bag of cells'' with its own root, {\em and remove from it all cells
  occurring in the old version}. The remaining ``bag of cells'' is
essentially an {\em update\/} to the object. Everybody who has the old
version of this object and the update can compute the new version,
simply by uniting the two bags of cells, and removing the old root
(decreasing its reference counter and de-allocating the cell if the
reference counter becomes zero).

\nxsubpoint \embt(Updates to the state of an account.)  In particular,
updates to the state of an account, or to the global state of a
shardchain, or to any hashmap can be represented using the idea
described in~\ptref{sp:obj.update}. This means that when we receive a
new shardchain block (which is a ``bag of cells''), we interpret this
``bag of cells'' not just by itself, but by uniting it first with the
``bag of cells'' representing the previous state of the shardchain. In
this sense each block may ``contain'' the whole state of the
blockchain.

\nxsubpoint \embt(Updates to a block.)  Recall that a block itself is
a ``bag of cells'', so, if it becomes necessary to edit a block, one
can similarly define a ``block update'' as a ``bag of cells'',
interpreted in the presence of the ``bag of cells'' which is the
previous version of this block. This is roughly the idea behind the
``vertical blocks'' discussed in~\ptref{sp:inv.sh.blk.corr}.

\nxsubpoint\label{sp:merkle.as.BoC} \embt(Merkle proof as a ``bag of
cells''.)  Notice that a (generalized) Merkle proof---for example, one
asserting that $x[i]=y$ starting from a known value of $\Hash(x)=h$
(cf.~\ptref{sp:merkle.proof} and~\ptref{sp:gen.merkle.proof})---may
also be represented as a ``bag of cells''. Namely, one simply needs to
provide a subset of cells corresponding to a path from the root of
$x:\Hashmap(n,X)$ to its desired leaf with index $i:\st2^n$ and value
$y:X$. References to children of these cells not lying on this path
will be left ``unresolved'' in this proof, represented by cell
hashes. One can also provide a simultaneous Merkle proof of, say,
$x[i]=y$ and $x[i']=y'$, by including in the ``bag of cells'' the
cells lying on the union of the two paths from the root of $x$ to
leaves corresponding to indices $i$ and~$i'$.

\nxsubpoint\label{sp:merkle.query.resp} \embt(Merkle proofs as query
responses from full nodes.)  In essence, a full node with a complete
copy of a shardchain (or account-chain) state can provide a Merkle
proof when requested by a light node (e.g., a network node running a
light version of the TON Blockchain client), enabling the receiver to
perform some simple queries without external help, using only the
cells provided in this Merkle proof. The light node can send its
queries in a serialized format to the full node, and receive the
correct answers with Merkle proofs---or just the Merkle proofs,
because the requester should be able to compute the answers using only
the cells included in the Merkle proof. This Merkle proof would
consist simply of a ``bag of cells'', containing only those cells
belonging to the shardchain's state that have been accessed by the
full node while executing the light node's query. This approach can be
used in particular for executing ``get queries'' of smart contracts
(cf.~\ptref{sp:tent.exec.get}).

\nxsubpoint\label{sp:aug.upd} \embt(Augmented update, or state update
with Merkle proof of validity.)  Recall (cf.~\ptref{sp:obj.update})
that we can describe the changes in an object state from an old value
$x:X$ to a new value $x':X$ by means of an ``update'', which is simply
a ``bag of cells'', containing those cells that lie in the subtree
representing new value $x'$, but not in the subtree representing old
value $x$, because the receiver is assumed to have a copy of the old
value $x$ and all its cells.

However, if the receiver does not have a full copy of $x$, but knows
only its (Merkle) hash $h=\Hash(x)$, it will not be able to check the
validity of the update (i.e., that all ``dangling'' cell references in
the update do refer to cells present in the tree of $x$). One would
like to have ``verifiable'' updates, augmented by Merkle proofs of
existence of all referred cells in the old state. Then anybody knowing
only $h=\Hash(x)$ would be able to check the validity of the update
and compute the new $h'=\Hash(x')$ by itself.

Because our Merkle proofs are ``bags of cells'' themselves
(cf.~\ptref{sp:merkle.as.BoC}), one can construct such an {\em
  augmented update\/} as a ``bag of cells'', containing the old root
of $x$, some of its descendants along with paths from the root of $x$
to them, and the new root of $x'$ and all its descendants that are not
part of $x$.

\nxsubpoint \embt(Account state updates in a shardchain block.)  In
particular, account state updates in a shardchain block should be
augmented as discussed in~\ptref{sp:aug.upd}. Otherwise, somebody
might commit a block containing an invalid state update, referring to
a cell absent in the old state; proving the invalidity of such a block
would be problematic (how is the challenger to prove that a cell is
{\em not\/} part of the previous state?).

Now, if all state updates included in a block are augmented, their
validity is easily checked, and their invalidity is also easily shown
as a violation of the recursive defining property of (generalized)
Merkle hashes.

\nxsubpoint\label{sp:everything.is.BoC} \embt(``Everything is a bag of
cells'' philosophy.)  Previous considerations show that everything we
need to store or transfer, either in the TON Block\-chain or in the
network, is representable as a ``bag of cells''. This is an important
part of the TON Blockchain design philosophy. Once the ``bag of
cells'' approach is explained and some ``low-level'' serializations of
``bags of cells'' are defined, one can simply define everything (block
format, shardchain and account state, etc.) on the high level of
abstract (dependent) algebraic data types.

The unifying effect of the ``everything is a bag of cells'' philosophy
considerably simplifies the implementation of seemingly unrelated
services; cf.~\ptref{sp:ton.smart.pc.supp} for an example involving
payment channels.

\nxsubpoint \embt(Block ``headers'' for TON blockchains.)  Usually, a
block in a block\-chain begins with a small header, containing the
hash of the previous block, its creation time, the Merkle hash of the
tree of all transactions contained in the block, and so on. Then the
block hash is defined to be the hash of this small block
header. Because the block header ultimately depends on all data
included in the block, one cannot alter the block without changing its
hash.

In the ``bag of cells'' approach used by the blocks of TON
blockchains, there is no designated block header. Instead, the block
hash is defined as the (Merkle) hash of the root cell of the
block. Therefore, the top (root) cell of the block might be considered
a small ``header'' of this block.

However, the root cell might not contain all the data usually expected
from such a header. Essentially, one wants the header to contain some
of the fields defined in the $\Block$ datatype. Normally, these fields
will be contained in several cells, including the root. These are the
cells that together constitute a ``Merkle proof'' for the values of
the fields in question. One might insist that a block contain these
``header cells'' in the very beginning, before any other cells. Then
one would need to download only the first several bytes of a block
serialization in order to obtain all of the ``header cells'', and to
learn all of the expected fields.

\mysubsection{Creating and Validating New Blocks}\label{sect:validators}

The TON Blockchain ultimately consists of shardchain and masterchain
blocks. These blocks must be created, validated and propagated through
the network to all parties concerned, in order for the system to
function smoothly and correctly.

\nxsubpoint\label{sp:validators} \embt(Validators.)  New blocks are
created and validated by special designated nodes, called {\em
  validators}. Essentially, any node wishing to become a validator may
become one, provided it can deposit a sufficiently large stake (in TON
coins, i.e., Grams; cf.\ Appendix~\ptref{app:coins}) into the
masterchain. Validators obtain some ``rewards'' for good work, namely,
the transaction, storage and gas fees from all transactions (messages)
committed into newly generated blocks, and some newly minted coins,
reflecting the ``gratitude'' of the whole community to the validators
for keeping the TON Blockchain working. This income is distributed
among all participating validators proportionally to their stakes.

However, being a validator is a high responsibility. If a validator
signs an invalid block, it can be punished by losing part or all of
its stake, and by being temporarily or permanently excluded from the
set of validators. If a validator does not participate in creating a
block, it does not receive its share of the reward associated with
that block. If a validator abstains from creating new blocks for a
long time, it may lose part of its stake and be suspended or
permanently excluded from the set of validators.

All this means that the validator does not get its money ``for
nothing''. Indeed, it must keep track of the states of all or some
shardchains (each validator is responsible for validating and creating
new blocks in a certain subset of shardchains), perform all
computations requested by smart contracts in these shardchains,
receive updates about other shardchains and so on. This activity
requires considerable disk space, computing power and network
bandwidth.

\nxsubpoint \embt(Validators instead of miners.)  Recall that the TON
Blockchain uses the Proof-of-Stake approach, instead of the
Proof-of-Work approach adopted by Bitcoin, the current version of
Ethereum, and most other cryptocurrencies. This means that one cannot
``mine'' a new block by presenting some proof-of-work (computing a lot
of otherwise useless hashes) and obtain some new coins as a
result. Instead, one must become a validator and spend one's computing
resources to store and process TON Blockchain requests and data. In
short, {\em one must be a validator to mine new coins.} In this
respect, {\em validators are the new miners.}

However, there are some other ways to earn coins apart from being a
validator.

\nxsubpoint\label{sp:nominators} \embt(Nominators and ``mining
pools''.)  To become a validator, one would normally need to buy and
install several high-performance servers and acquire a good Internet
connection for them. This is not so expensive as the ASIC equipment
currently required to mine Bitcoins. However, one definitely cannot
mine new TON coins on a home computer, let alone a smartphone.

In the Bitcoin, Ethereum and other Proof-of-Work cryptocurrency mining
communities there is a notion of {\em mining pools}, where a lot of
nodes, having insufficient computing power to mine new blocks by
themselves, combine their efforts and share the reward afterwards.

A corresponding notion in the Proof-of-Stake world is that of a {\em
  nominator}. Essentially, this is a node lending its money to help a
validator increase its stake; the validator then distributes the
corresponding share of its reward (or some previously agreed fraction
of it---say, 50\%) to the nominator.

In this way, a nominator can also take part in the ``mining'' and
obtain some reward proportional to the amount of money it is willing
to deposit for this purpose. It receives only a fraction of the
corresponding share of the validator's reward, because it provides
only the ``capital'', but does not need to buy computing power,
storage and network bandwidth.

However, if the validator loses its stake because of invalid behavior,
the nominator loses its share of the stake as well. In this sense the
nominator {\em shares the risk}. It must choose its nominated
validator wisely, otherwise it can lose money. In this sense,
nominators make a weighted decision and ``vote'' for certain
validators with their funds.

On the other hand, this nominating or lending system enables one to
become a validator without investing a large amount of money into
Grams (TON coins) first. In other words, it prevents those keeping
large amounts of Grams from monopolizing the supply of validators.

\nxsubpoint\label{sp:fish} \embt(Fishermen: obtaining money by
pointing out others' mistakes.)  Another way to obtain some rewards
without being a validator is by becoming a {\em
  fisherman}. Essentially, any node can become a fisherman by making a
small deposit in the masterchain. Then it can use special masterchain
transactions to publish (Merkle) invalidity proofs of some (usually
shardchain) blocks previously signed and published by validators. If
other validators agree with this invalidity proof, the offending
validators are punished (by losing part of their stake), and the
fisherman obtains some reward (a fraction of coins confiscated from
the offending validators). Afterwards, the invalid (shardchain) block
must be corrected as outlined
in~\ptref{sp:inv.sh.blk.corr}. Correcting invalid masterchain blocks
may involve creating ``vertical'' blocks on top of previously
committed masterchain blocks (cf.~\ptref{sp:inv.sh.blk.corr}); there
is no need to create a fork of the masterchain.

Normally, a fisherman would need to become a full node for at least
some shardchains, and spend some computing resources by running the
code of at least some smart contracts. While a fisherman does not need
to have as much computing power as a validator, we think that a
natural candidate to become a fisherman is a would-be validator that
is ready to process new blocks, but has not yet been elected as a
validator (e.g., because of a failure to deposit a sufficiently large
stake).

\nxsubpoint\label{sp:collators} \embt(Collators: obtaining money by
suggesting new blocks to validators.)  Yet another way to obtain some
rewards without being a validator is by becoming a {\em
  collator}. This is a node that prepares and suggests to a validator
new shardchain block candidates, complemented (collated) with data
taken from the state of this shardchain and from other (usually
neighboring) shardchains, along with suitable Merkle proofs. (This is
necessary, for example, when some messages need to be forwarded from
neighboring shardchains.) Then a validator can easily check the
proposed block candidate for validity, without having to download the
complete state of this or other shardchains.

Because a validator needs to submit new (collated) block candidates to
obtain some (``mining'') rewards, it makes sense to pay some part of
the reward to a collator willing to provide suitable block
candidates. In this way, a validator may free itself from the necessity
of watching the state of the neighboring shardchains, by outsourcing
it to a collator.

However, we expect that during the system's initial deployment phase
there will be no separate designated collators, because all validators
will be able to act as collators for themselves.

\nxsubpoint \embt(Collators or validators: obtaining money for
including user transactions.)  Users can open micropayment channels to
some collators or validators and pay small amounts of coins in
exchange for the inclusion of their transactions in the shardchain.

\nxsubpoint\label{sp:global.valid} \embt(Global validator set
election.)  The ``global'' set of validators is elected once each
month (actually, every $2^{19}$ masterchain blocks). This set is
determined and universally known one month in advance.

In order to become a validator, a node must transfer some TON coins
(Grams) into the masterchain, and then send them to a special smart
contract as its suggested stake $s$. Another parameter, sent along
with the stake, is $l\geq 1$, the maximum validating load this node is
willing to accept relative to the minimal possible. There is also a
global upper bound (another configurable parameter) $L$ on $l$, equal
to, say, 10.

Then the global set of validators is elected by this smart contract,
simply by selecting up to $T$ candidates with maximal suggested stakes
and publishing their identities. Originally, the total number of
validators is $T=100$; we expect it to grow to 1000 as the load
increases. It is a configurable parameter
(cf.~\ptref{sp:config.param}).

The actual stake of each validator is computed as follows: If the top
$T$ proposed stakes are $s_1\geq s_2\geq\cdots\geq s_T$, the actual
stake of $i$-th validator is set to $s'_i:=\min(s_i,l_i\cdot s_T)$. In
this way, $s'_i/s'_T\leq l_i$, so the $i$-th validator does not obtain
more than $l_i\leq L$ times the load of the weakest validator (because
the load is ultimately proportional to the stake).

Then elected validators may withdraw the unused part of their stake,
$s_i-s'_i$. Unsuccessful validator candidates may withdraw all of
their proposed stake.

Each validator publishes its {\em public signing key}, not necessarily
equal to the public key of the account the stake came
from.\footnote{It makes sense to generate and use a new key pair for
  every validator election.}

The stakes of the validators are frozen until the end of the period for
which they have been elected, and one month more, in case new disputes
arise (i.e., an invalid block signed by one of these validators is
found). After that, the stake is returned, along with the validator's
share of coins minted and fees from transactions processed during this
time.

\nxsubpoint\label{sp:val.task.grp} \embt(Election of validator ``task
groups''.)  The whole global set of validators (where each validator
is considered present with multiplicity equal to its stake---otherwise
a validator might be tempted to assume several identities and split
its stake among them) is used only to validate new masterchain
blocks. The shardchain blocks are validated only by specially selected
subsets of validators, taken from the global set of validators chosen
as described in~\ptref{sp:global.valid}.

These validator ``subsets'' or ``task groups'', defined for every
shard, are rotated each hour (actually, every $2^{10}$ masterchain
blocks), and they are known one hour in advance, so that every
validator knows which shards it will need to validate, and can prepare
for that (e.g., by downloading missing shardchain data).

The algorithm used to select validator task groups for each shard
$(w,s)$ is deterministic pseudorandom. It uses pseudorandom numbers
embedded by validators into each masterchain block (generated by a
consensus using threshold signatures) to create a random seed, and
then computes for example
$\Hash(\code(w).\code(s).\vr{validator\_id}.\vr{rand\_seed})$ for each
validator. Then validators are sorted by the value of this hash, and
the first several are selected, so as to have at least $20/T$ of the
total validator stakes and consist of at least 5 validators.

This selection could be done by a special smart contract. In that
case, the selection algorithm would easily be upgradable without hard
forks by the voting mechanism mentioned
in~\ptref{sp:config.param}. All other ``constants'' mentioned so far
(such as $2^{19}$, $2^{10}$, $T$, 20, and 5) are also configurable
parameters.

\nxsubpoint\label{sp:rot.gen.prio} \embt(Rotating priority order on
each task group.)  There is a certain ``priority'' order imposed
on the members of a shard task group, depending on the hash of the
previous masterchain block and (shardchain) block sequence
number. This order is determined by generating and sorting some hashes
as described above.

When a new shardchain block needs to be generated, the shard task
group validator selected to create this block is normally the first
one with respect to this rotating ``priority'' order. If it fails to
create the block, the second or third validator may do
it. Essentially, all of them may suggest their block candidates, but
the candidate suggested by the validator having the highest priority
should win as the result of Byzantine Fault Tolerant (BFT) consensus
protocol.

\nxsubpoint\label{sp:sh.blk.cand.prop} \embt(Propagation of shardchain
block candidates.)  Because shardchain task group membership is
known one hour in advance, their members can use that time to build a
dedicated ``shard validators multicast overlay network'', using the
general mechanisms of the TON Network (cf.~\ptref{sect:overlay}). When
a new shardchain block needs to be generated---normally one or two
seconds after the most recent masterchain block has been
propagated---everybody knows who has the highest priority to generate
the next block (cf.~\ptref{sp:rot.gen.prio}). This validator will
create a new collated block candidate, either by itself or with the
aid of a collator (cf.~\ptref{sp:collators}). The validator must check
(validate) this block candidate (especially if it has been prepared by
some collator) and sign it with its (validator) private key. Then the
block candidate is propagated to the remainder of the task group
using the prearranged multicast overlay network (the task group
creates its own private overlay network as explained
in~\ptref{sect:overlay}, and then uses a version of the streaming
multicast protocol described in~\ptref{sp:streaming.multicast} to
propagate block candidates).

A truly BFT way of doing this would be to use a Byzantine multicast
protocol, such as the one used in Honey Badger BFT~\cite{HoneyBadger}:
encode the block candidate by an $(N,2N/3)$-erasure code, send $1/N$
of the resulting data directly to each member of the group, and expect
them to multicast directly their part of the data to all other members
of the group.

However, a faster and more straightforward way of doing this
(cf.\ also \ptref{sp:streaming.multicast}) is to split the block
candidate into a sequence of signed one-kilobyte blocks (``chunks''),
augment their sequence by a Reed--Solomon or a fountain code (such as
the RaptorQ code~\cite{RaptorQ} \cite{Raptor}), and start transmitting
chunks to the neighbors in the ``multicast mesh'' (i.e., the overlay
network), expecting them to propagate these chunks further. Once a
validator obtains enough chunks to reconstruct the block candidate
from them, it signs a confirmation receipt and propagates it through
its neighbors to the whole of the group. Then its neighbors stop
sending new chunks to it, but may continue to send the (original)
signatures of these chunks, believing that this node can generate the
subsequent chunks by applying the Reed--Solomon or fountain code by
itself (having all data necessary), combine them with signatures, and
propagate to its neighbors that are not yet ready.

If the ``multicast mesh'' (overlay network) remains connected after
removing all ``bad'' nodes (recall that up to one-third of nodes are
allowed to be bad in a Byzantine way, i.e., behave in arbitrary
malicious fashion), this algorithm will propagate the block candidate
as quickly as possible.

Not only the designated high-priority block creator may multicast its
block candidate to the whole of the group. The second and third
validator by priority may start multicasting their block candidates,
either immediately or after failing to receive a block candidate from
the top priority validator. However, normally only the block candidate
with maximal priority will be signed by all (actually, by at least
two-thirds of the task group) validators and committed as a new
shardchain block.

\nxsubpoint \embt(Validation of block candidates.)  Once a block
candidate is received by a validator and the signature of its
originating validator is checked, the receiving validator checks the
validity of this block candidate, by performing all transactions in it
and checking that their result coincides with the one claimed. All
messages imported from other blockchains must be supported by suitable
Merkle proofs in the collated data, otherwise the block candidate is
deemed invalid (and, if a proof of this is committed to the
masterchain, the validators having already signed this block candidate
may be punished). On the other hand, if the block candidate is found
to be valid, the receiving validator signs it and propagates its
signature to other validators in the group, either through the ``mesh
multicast network'', or by direct network messages.

We would like to emphasize that {\em a validator does not need access
  to the states of this or neighboring shardchains in order to check
  the validity of a (collated) block candidate}.%
\footnote{A possible exception is the state of output queues of the
  neighboring shardchains, needed to guarantee the message ordering
  requirements described in~\ptref{sp:collect.input.msg}, because the
  size of Merkle proofs might become prohibitive in this case.}  This
allows the validation to proceed very quickly (without disk accesses),
and lightens the computational and storage burden on the validators
(especially if they are willing to accept the services of outside
collators for creating block candidates).

\nxsubpoint\label{sp:new.shardc.blk} \embt(Election of the next block
candidate.)  Once a block candidate collects at least two-thirds (by
stake) of the validity signatures of validators in the task group, it
is eligible to be committed as the next shardchain block. A BFT
protocol is run to achieve consensus on the block candidate chosen
(there may be more than one proposed), with all ``good'' validators
preferring the block candidate with the highest priority for this
round. As a result of running this protocol, the block is augmented by
signatures of at least two-thirds of the validators (by stake). These
signatures testify not only to the validity of the block in question,
but also to its being elected by the BFT protocol. After that, the
block (without collated data) is combined with these signatures,
serialized in a deterministic way, and propagated through the network
to all parties concerned.

\nxsubpoint \embt(Validators must keep the blocks they have signed.)
During their membership in the task group and for at least one hour
(or rather $2^{10}$ blocks) afterward, the validators are expected to
keep the blocks they have signed and committed.  The failure to
provide a signed block to other validators may be punished.

\nxsubpoint \embt(Propagating the headers and signatures of new
shardchain blocks to all validators.)  Validators propagate the
headers and signatures of newly-generated shardchain blocks to the
{\em global\/} set of validators, using a multicast mesh network
similar to the one created for each task group.

\nxsubpoint\label{sp:new.master.blk} \embt(Generation of new
masterchain blocks.)  After all (or almost all) new shardchain blocks
have been generated, a new masterchain block may be generated. The
procedure is essentially the same as for shardchain blocks
(cf.~\ptref{sp:new.shardc.blk}), with the difference that {\em all\/}
validators (or at least two-thirds of them) must participate in this
process. Because the headers and signatures of new shardchain blocks
are propagated to all validators, hashes of the newest blocks in each
shardchain can and must be included in the new masterchain block. Once
these hashes are committed into the masterchain block, outside
observers and other shardchains may consider the new shardchain blocks
committed and immutable (cf.~\ptref{sp:sc.hash.mc}).

\nxsubpoint \embt(Validators must keep the state of masterchain.)  A
noteworthy difference between the masterchain and the shardchains is
that all validators are expected to keep track of the masterchain
state, without relying on collated data. This is important because the
knowledge of validator task groups is derived from the masterchain
state.

\nxsubpoint \embt(Shardchain blocks are generated and propagated in
parallel.)  Normally, each validator is a member of several shardchain
task groups; their quantity (hence the load on the validator) is
approximately proportional to the validator's stake. This means that
the validator runs several instances of new shardchain block
generation protocol in parallel.

\nxsubpoint \embt(Mitigation of block retention attacks.)  Because the
total set of validators inserts a new shardchain block's hash into the
masterchain after having seen only its header and signatures, there is
a small probability that the validators that have generated this block
will conspire and try to avoid publishing the new block in its
entirety. This would result in the inability of validators of
neighboring shardchains to create new blocks, because they must know
at least the output message queue of the new block, once its hash has
been committed into the masterchain.

In order to mitigate this, the new block must collect signatures from
some other validators (e.g., two-thirds of the union of task groups of
neighboring shardchains) testifying that these validators do have
copies of this block and are willing to send them to any other
validators if required. Only after these signatures are presented may
the new block's hash be included in the masterchain.

\nxsubpoint \embt(Masterchain blocks are generated later than
shardchain blocks.)  Masterchain blocks are generated approximately
once every five seconds, as are shardchain blocks. However, while the
generation of new blocks in all shardchains runs essentially at the
same time (normally triggered by the release of a new masterchain
block), the generation of new masterchain blocks is deliberately
delayed, to allow the inclusion of hashes of newly-generated
shardchain blocks in the masterchain.

\nxsubpoint\label{sp:slow.valid} \embt(Slow validators may receive
lower rewards.)  If a validator is ``slow'', it may fail to validate
new block candidates, and two-thirds of the signatures required to
commit the new block may be gathered without its participation. In
this case, it will receive a lower share of the reward associated with
this block.

This provides an incentive for the validators to optimize their
hardware, software, and network connection in order to process user
transactions as fast as possible.

However, if a validator fails to sign a block before it is committed,
its signature may be included in one of the next blocks, and then a
part of the reward (exponentially decreasing depending on how many
blocks have been generated since---e.g., $0.9^k$ if the validator is
$k$ blocks late) will be still given to this validator.

\nxsubpoint\label{sp:val.sign.depth} \embt(``Depth'' of validator
signatures.)  Normally, when a validator signs a block, the signature
testifies only to the {\em relative validity\/} of a block: this block
is valid provided all previous blocks in this and other shardchains
are valid. The validator cannot be punished for taking for granted
invalid data committed into previous blocks.

However, the validator signature of a block has an integer parameter
called ``depth''. If it is non-zero, it means that the validator
asserts the (relative) validity of the specified number of previous
blocks as well. This is a way for ``slow'' or ``temporarily offline''
validators to catch up and sign some of the blocks that have been
committed without their signatures. Then some part of the block reward
will still be given to them (cf.~\ptref{sp:slow.valid}).

\nxsubpoint\label{sp:abs.val.from.rel} \embt(Validators are
responsible for {\em relative\/} validity of signed shardchain blocks;
absolute validity follows.)  We would like to emphasize once again
that a validator's signature on a shardchain block $B$ testifies to
only the {\em relative\/} validity of that block (or maybe of $d$
previous blocks as well, if the signature has ``depth'' $d$,
cf.~\ptref{sp:val.sign.depth}; but this does not affect the following
discussion much). In other words, the validator asserts that the next
state $s'$ of the shardchain is obtained from the previous state $s$
by applying the block evaluation function $\evblock$ described
in~\ptref{sp:blk.transf}:
\begin{equation}\label{eq:ev.block.2}
  s'=\evblock(B)(s)
\end{equation}
In this way, the validator that signed block $B$ cannot be punished if
the original state $s$ turns out to be ``incorrect'' (e.g., because of
the invalidity of one of the previous blocks). A fisherman
(cf.~\ptref{sp:fish}) should complain only if it finds a block that is
{\em relatively\/} invalid. The PoS system as a whole endeavors to
make every block {\em relatively\/} valid, not {\em recursively (or
  absolutely)} valid. Notice, however, that {\em if all blocks in a
  blockchain are relatively valid, then all of them and the blockchain
  as a whole are absolutely valid}; this statement is easily shown
using mathematical induction on the length of the blockchain. In this
way, easily verifiable assertions of {\em relative\/} validity of
blocks together demonstrate the much stronger {\em absolute validity\/}
of the whole blockchain.

Note that by signing a block~$B$ the validator asserts that the block
is valid given the original state $s$ (i.e., that the result
of~\eqref{eq:ev.block.2} is not the value $\bot$ indicating that the
next state cannot be computed). In this way, the validator must
perform minimal formal checks of the cells of the original state that
are accessed during the evaluation of~\eqref{eq:ev.block.2}.

For example, imagine that the cell expected to contain the original
balance of an account accessed from a transaction committed into a
block turns out to have zero raw bytes instead of the expected 8 or
16. Then the original balance simply cannot be retrieved from the
cell, and an ``unhandled exception'' happens while trying to process
the block. In this case, the validator should not sign such a block on
pain of being punished.

\nxsubpoint \embt(Signing masterchain blocks.)  The situation with the
masterchain blocks is somewhat different: by signing a masterchain
block, the validator asserts not only its relative validity, but also
the relative validity of all preceding blocks up to the very first
block when this validator assumed its responsibility (but not further
back).

\nxsubpoint \embt(The total number of validators.)  The upper
limit $T$ for the total number of validators to be elected
(cf.~\ptref{sp:global.valid}) cannot become, in the system described
so far, more than, say, several hundred or a thousand, because all
validators are expected to participate in a BFT consensus protocol to
create each new masterchain block, and it is not clear whether such
protocols can scale to thousands of participants. Even more
importantly, masterchain blocks must collect the signatures of at
least two-thirds of all the validators (by stake), and these
signatures must be included in the new block (otherwise all other
nodes in the system would have no reason to trust the new block
without validating it by themselves). If more than, say, one thousand
validator signatures would have to be included in each masterchain
block, this would imply more data in each masterchain block, to be
stored by all full nodes and propagated through the network, and more
processing power spent to check these signatures (in a PoS system,
full nodes do not need to validate blocks by themselves, but they need
to check the validators' signatures instead).

While limiting $T$ to a thousand validators seems more than sufficient
for the first phase of the deployment of the TON Blockchain, a
provision must be made for future growth, when the total number of
shardchains becomes so large that several hundred validators will not
suffice to process all of them. To this end, we introduce an
additional configurable parameter $T'\leq T$ (originally equal
to~$T$), and only the top $T'$ elected validators (by stake) are
expected to create and sign new masterchain blocks.

\nxsubpoint \embt(Decentralization of the system.)  One might suspect
that a Proof-of-Stake system such as the TON Blockchain, relying on
$T\approx1000$ validators to create all shardchain and masterchain
blocks, is bound to become ``too centralized'', as opposed to
conventional Proof-of-Work blockchains like Bitcoin or Ethereum, where
everybody (in principle) might mine a new block, without an explicit
upper limit on the total number of miners.

However, popular Proof-of-Work blockchains, such as Bitcoin and
Ether\-eum, currently require vast amounts of computing power (high
``hash rates'') to mine new blocks with non-negligible probability of
success. Thus, the mining of new blocks tends to be concentrated in the
hands of several large players, who invest huge amounts money into
datacenters filled with custom-designed hardware optimized for mining;
and in the hands of several large mining pools, which concentrate and
coordinate the efforts of larger groups of people who are not able to
provide a sufficient ``hash rate'' by themselves.

Therefore, as of 2017, more than 75\% of new Ethereum or Bitcoin
blocks are produced by less than ten miners. In fact, the two largest
Ethereum mining pools produce together more than half of all new
blocks! Clearly, such a system is much more centralized than one
relying on $T\approx1000$ nodes to produce new blocks.

One might also note that the investment required to become a TON
Blockchain validator---i.e., to buy the hardware (say, several
high-performance servers) and the stake (which can be easily collected
through a pool of nominators if necessary;
cf.~\ptref{sp:nominators})---is much lower than that required to
become a successful stand-alone Bitcoin or Ethereum miner. In fact,
the parameter $L$ of~\ptref{sp:global.valid} will force nominators not
to join the largest ``mining pool'' (i.e., the validator that has
amassed the largest stake), but rather to look for smaller validators
currently accepting funds from nominators, or even to create new
validators, because this would allow a higher proportion $s'_i/s_i$ of
the validator's---and by extension also the nominator's---stake to be
used, hence yielding larger rewards from mining. In this way, the TON
Proof-of-Stake system actually {\em encourages\/} decentralization
(creating and using more validators) and {\em punishes\/}
centralization.

\nxsubpoint\label{sp:rel.rel} \embt(Relative reliability of a block.)
The {\em (relative) reliability\/} of a block is simply the total
stake of all validators that have signed this block. In other words,
this is the amount of money certain actors would lose if this block
turns out to be invalid. If one is concerned with transactions
transferring value lower than the reliability of the block, one can
consider them to be safe enough. In this sense, the relative
reliability is a measure of trust an outside observer can have in a
particular block.

Note that we speak of the {\em relative\/} reliability of a block,
because it is a guarantee that the block is valid {\em provided the
  previous block and all other shardchains' blocks referred to are
  valid\/} (cf.~\ptref{sp:abs.val.from.rel}).

The relative reliability of a block can grow after it is
committed---for example, when belated validators' signatures are added
(cf.~\ptref{sp:val.sign.depth}). On the other hand, if one of these
validators loses part or all of its stake because of its misbehavior
related to some other blocks, the relative reliability of a block may
{\em decrease}.

\nxsubpoint \embt(``Strengthening'' the blockchain.)  It is important
to provide incentives for validators to increase the relative
reliability of blocks as much as possible. One way of doing this is by
allocating a small reward to validators for adding signatures to
blocks of other shardchains. Even ``would-be'' validators, who have
deposited a stake insufficient to get into the top $T$ validators by
stake and to be included in the global set of validators
(cf.~\ptref{sp:global.valid}), might participate in this activity (if
they agree to keep their stake frozen instead of withdrawing it after
having lost the election). Such would-be validators might double as
fishermen (cf.~\ptref{sp:fish}): if they have to check the validity of
certain blocks anyway, they might as well opt to report invalid blocks and collect the associated rewards.

\nxsubpoint\label{sp:rec.rel} \embt(Recursive reliability of a block.)
One can also define the {\em recursive reliability\/} of a block to be
the minimum of its relative reliability and the recursive
reliabilities of all blocks it refers to (i.e., the masterchain block,
the previous shardchain block, and some blocks of neighboring
shardchains). In other words, if the block turns out to be invalid,
either because it is invalid by itself or because one of the blocks it
depends on is invalid, then at least this amount of money would be
lost by someone. If one is truly unsure whether to trust a specific
transaction in a block, one should compute the {\em recursive\/}
reliability of this block, not just the {\em relative\/} one.

It does not make sense to go too far back when computing recursive
reliability, because, if we look too far back, we will see blocks
signed by validators whose stakes have already been unfrozen and
withdrawn. In any case, we do not allow the validators to
automatically reconsider blocks that are that old (i.e., created more
than two months ago, if current values of configurable parameters are
used), and create forks starting from them or correct them with the
aid of ``vertical blockchains'' (cf.~\ptref{sp:inv.sh.blk.corr}), even
if they turn out to be invalid. We assume that a period of two months
provides ample opportunities for detecting and reporting any invalid
blocks, so that if a block is not challenged during this period, it is
unlikely to be challenged at all.

\nxsubpoint \embt(Consequence of Proof-of-Stake for light nodes.)  An
important consequence of the Proof-of-Stake approach used by the TON
Blockchain is that a light node (running light client software) for
the TON Blockchain does not need to download the ``headers'' of all
shardchain or even masterchain blocks in order to be able to check by
itself the validity of Merkle proofs provided to it by full nodes as
answers to its queries.

Indeed, because the most recent shardchain block hashes are included
in the masterchain blocks, a full node can easily provide a Merkle
proof that a given shardchain block is valid starting from a known
hash of a masterchain block. Next, the light node needs to know only
the very first block of the masterchain (where the very first set of
validators is announced), which (or at least the hash of which) might
be built-in into the client software, and only one masterchain block
approximately every month afterwards, where newly-elected validator
sets are announced, because this block will have been signed by the
previous set of validators. Starting from that, it can obtain the
several most recent masterchain blocks, or at least their headers and
validator signatures, and use them as a base for checking Merkle
proofs provided by full nodes.

\mysubsection{Splitting and Merging
  Shardchains}\label{sect:split.merge}

One of the most characteristic and unique features of the TON
Blockchain is its ability to automatically split a shardchain in two
when the load becomes too high, and merge them back if the load
subsides (cf.~\ptref{sp:dyn.split.merge}). We must discuss it in some
detail because of its uniqueness and its importance to the scalability
of the whole project.

\nxsubpoint \embt(Shard configuration.)  Recall that, at any given
moment of time, each workchain $w$ is split into one or several
shardchains $(w,s)$ (cf.~\ptref{sp:shard.ident}). These shardchains
may be represented by leaves of a binary tree, with root
$(w,\emptyset)$, and each non-leaf node $(w,s)$ having children
$(w,s.0)$ and $(w,s.1)$. In this way, every account belonging to
workchain $w$ is assigned to exactly one shard, and everybody who
knows the current shardchain configuration can determine the shard
$(w,s)$ containing account $\accountid$: it is the only shard with
binary string $s$ being a prefix of $\accountid$.

The shard configuration---i.e., this {\em shard binary tree}, or the
collection of all active $(w,s)$ for a given $w$ (corresponding to the
leaves of the shard binary tree)---is part of the masterchain state
and is available to everybody who keeps track of the
masterchain.\footnote{Actually, the shard configuration is completely
  determined by the last masterchain block; this simplifies getting
  access to the shard configuration.}

\nxsubpoint \embt(Most recent shard configuration and state.)  Recall
that hashes of the most recent shardchain blocks are included in each
masterchain block. These hashes are organized in a shard binary tree
(actually, a collection of trees, one for each workchain). In this
way, each masterchain block contains the most recent shard
configuration.

\nxsubpoint \embt(Announcing and performing changes in the shard
configuration.)  The shard configuration may be changed in two ways:
either a shard $(w,s)$ can be {\em split\/} into two shards $(w,s.0)$
and $(w,s.1)$, or two ``sibling'' shards $(w,s.0)$ and $(w,s.1)$ can
be {\em merged\/} into one shard $(w,s)$.

These split/merge operations are announced several (e.g., $2^6$; this
is a configurable parameter) blocks in advance, first in the
``headers'' of the corresponding shardchain blocks, and then in the
masterchain block that refers to these shardchain blocks. This advance
announcement is needed for all parties concerned to prepare for the
planned change (e.g., build an overlay multicast network to distribute
new blocks of the newly-created shardchains, as discussed
in~\ptref{sect:overlay}). Then the change is committed, first into the
(header of the) shardchain block (in case of a split; for a merge,
blocks of both shardchains should commit the change), and then
propagated to the masterchain block. In this way, the masterchain
block defines not only the most recent shard configuration {\em
  before\/} its creation, but also the next immediate shard
configuration.

\nxsubpoint \embt(Validator task groups for new shardchains.)  Recall
that each shard, i.e., each shardchain, normally is assigned a subset
of validators (a validator task group) dedicated to creating and
validating new blocks in the corresponding shardchain
(cf.~\ptref{sp:val.task.grp}). These task groups are elected for some
period of time (approximately one hour) and are known some time in
advance (also approximately one hour), and are immutable during this
period.\footnote{Unless some validators are temporarily or permanently
  banned because of signing invalid blocks---then they are
  automatically excluded from all task groups.}

However, the actual shard configuration may change during this period
because of split/merge operations. One must assign task groups to
newly created shards. This is done as follows:

Notice that any active shard $(w,s)$ will either be a descendant of
some uniquely determined original shard $(w,s')$, meaning that $s'$ is
a prefix of $s$, or it will be the root of a subtree of original
shards $(w,s')$, where $s$ will be a prefix of every $s'$. In the
first case, we simply take the task group of the original shard
$(w,s')$ to double as the task group of the new shard $(w,s)$. In the
latter case, the task group of the new shard $(w,s)$ will be the union
of task groups of all original shards $(w,s')$ that are descendants of
$(w,s)$ in the shard tree.

In this way, every active shard $(w,s)$ gets assigned a well-defined
subset of validators (task group). When a shard is split, both
children inherit the whole of the task group from the original
shard. When two shards are merged, their task groups are also merged.

Anyone who keeps track of the masterchain state can compute validator
task groups for each of the active shards.

\nxsubpoint \embt(Limit on split/merge operations during the period of
responsibility of original task groups.)  Ultimately, the new shard
configuration will be taken into account, and new dedicated validator
subsets (task groups) will automatically be assigned to each
shard. Before that happens, one must impose a certain limit on
split/merge operations; otherwise, an original task group may end up
validating $2^k$ shardchains for a large $k$ at the same time, if the
original shard quickly splits into $2^k$ new shards.

This is achieved by imposing limits on how far the active shard
configuration may be removed from the original shard configuration
(the one used to select validator task groups currently in
charge). For example, one might require that the distance in the shard
tree from an active shard $(w,s)$ to an original shard $(w,s')$ must
not exceed 3, if $s'$ is a predecessor of $s$ (i.e., $s'$ is a prefix
of binary string $s$), and must not exceed 2, if $s'$ is a successor
of $s$ (i.e., $s$ is a prefix of $s'$). Otherwise, the split or merge
operation is not permitted.

Roughly speaking, one is imposing a limit on the number of times a
shard can be split (e.g., three) or merged (e.g., two) during the
period of responsibility of a given collection of validator task
groups. Apart from that, after a shard has been created by merging or
splitting, it cannot be reconfigured for some period of time (some
number of blocks).

\nxsubpoint\label{sp:split.necess} \embt(Determining the necessity of
split operations.)  The split operation for a shardchain is triggered
by certain formal conditions (e.g., if for 64 consecutive blocks the
shardchain blocks are at least $90\%$ full). These conditions are
monitored by the shardchain task group. If they are met, first a
``split preparation'' flag is included in the header of a new
shardchain block (and propagated to the masterchain block referring to
this shardchain block). Then, several blocks afterwards, the ``split
commit'' flag is included in the header of the shardchain block (and
propagated to the next masterchain block).

\nxsubpoint \embt(Performing split operations.)  After the ``split
commit'' flag is included in a block $B$ of shardchain $(w,s)$, there
cannot be a subsequent block $B'$ in that shardchain. Instead, two
blocks $B'_0$ and $B'_1$ of shardchains $(w,s.0)$ and $(w,s.1)$,
respectively, will be created, both referring to block $B$ as their
previous block (and both of them will indicate by a flag in the header
that the shard has been just split). The next masterchain block will
contain hashes of blocks $B'_0$ and $B'_1$ of the new shardchains; it
is not allowed to contain the hash of a new block $B'$ of shardchain
$(w,s)$, because a ``split commit'' event has already been committed
into the previous masterchain block.

Notice that both new shardchains will be validated by the same
validator task group as the old one, so they will automatically have a
copy of their state. The state splitting operation itself is quite
simple from the perspective of the Infinite Sharding Paradigm
(cf.~\ptref{sp:split.merge.state}).

\nxsubpoint\label{sp:merge.necess} \embt(Determining the necessity of
merge operations.)  The necessity of shard merge operations is also
detected by certain formal conditions (e.g., if for 64 consecutive
blocks the sum of the sizes of the two blocks of sibling shardchains
does not exceed $60\%$ of maximal block size). These formal conditions
should also take into account the total gas spent by these blocks and
compare it to the current block gas limit, otherwise the blocks may
happen to be small because there are some computation-intensive
transactions that prevent the inclusion of more transactions.

These conditions are monitored by validator task groups of both
sibling shards $(w,s.0)$ and $(w,s.1)$. Notice that siblings are
necessarily neighbors with respect to hypercube routing
(cf.~\ptref{sp:hypercube}), so validators from the task group of any
shard will be monitoring the sibling shard to some extent anyways.

When these conditions are met, either one of the validator subgroups
can suggest to the other that they merge by sending a special
message. Then they combine into a provisional ``merged task group'',
with combined membership, capable of running BFT consensus algorithms
and of propagating block updates and block candidates if necessary.

If they reach consensus on the necessity and readiness of merging,
``merge prepare'' flags are committed into the headers of some blocks
of each shardchain, along with the signatures of at least two-thirds
of the validators of the sibling's task group (and are propagated to
the next masterchain blocks, so that everybody can get ready for the
imminent reconfiguration). However, they continue to create separate
shardchain blocks for some predefined number of blocks.

\nxsubpoint \embt(Performing merge operations.)  After that, when the
validators from the union of the two original task groups are ready to
become validators for the merged shardchain (this might involve a
state transfer from the sibling shardchain and a state merge
operation), they commit a ``merge commit'' flag in the headers of
blocks of their shardchain (this event is propagated to the next
masterchain blocks), and stop creating new blocks in separate
shardchains (once the merge commit flag appears, creating blocks in
separate shardchains is forbidden). Instead, a merged shardchain block
is created (by the union of the two original task groups), referring
to both of its ``preceding blocks'' in its ``header''. This is
reflected in the next masterchain block, which will contain the hash
of the newly created block of the merged shardchain. After that, the
merged task group continues creating blocks in the merged shardchain.

\mysubsection{Classification of Blockchain
  Projects}\label{sect:class.blkch}

We will conclude our brief discussion of the TON Blockchain by
comparing it with existing and proposed blockchain projects. Before
doing this, however, we must introduce a sufficiently general
classification of blockchain projects. The comparison of particular
blockchain projects, based on this classification, is postponed
until~\ptref{sect:compare.blkch}.

\nxsubpoint \embt(Classification of blockchain projects.)  As a first
step, we suggest some classification criteria for blockchains (i.e.,
for blockchain projects). Any such classification is somewhat
incomplete and superficial, because it must ignore some of the most
specific and unique features of the projects under
consideration. However, we feel that this is a necessary first step in
providing at least a very rough and approximate map of the blockchain
projects territory.

The list of criteria we consider is the following:
\begin{itemize}
\item Single-blockchain vs.\ multi-blockchain architecture
  (cf.~\ptref{sp:single.multi})
\item Consensus algorithm: Proof-of-Stake vs.\ Proof-of-Work
  (cf.~\ptref{sp:pow.pos})
\item For Proof-of-Stake systems, the exact block generation,
  validation and consensus algorithm used (the two principal options
  are DPOS vs.\ BFT; cf.~\ptref{sp:dpos.bft})
\item Support for ``arbitrary'' (Turing-complete) smart contracts
  (cf.~\ptref{sp:smartc.supp})
\end{itemize}
Multi-blockchain systems have additional classification criteria
(cf.~\ptref{sp:class.multichain}):
\begin{itemize}
\item Type and rules of member blockchains: homogeneous, heterogeneous
  (cf.~\ptref{sp:blkch.hom.het}), mixed
  (cf.~\ptref{sp:mixed.het.hom}). Confederations
  (cf.~\ptref{sp:het.confed}).
\item Absence or presence of a {\em masterchain}, internal or external
  (cf.~\ptref{sp:pres.masterch})
\item Native support for sharding (cf.~\ptref{sp:shard.supp}). Static
  or dynamic sharding (cf.~\ptref{sp:dyn.stat.shard}).
\item Interaction between member blockchains: loosely-coupled and
  tightly-coupled systems (cf.~\ptref{sp:blkch.interact})
\end{itemize}

\nxsubpoint\label{sp:single.multi} \embt(Single-blockchain
vs.\ multi-blockchain projects.)  The first classification criterion
is the quantity of blockchains in the system. The oldest and simplest
projects consist of a {\em single blockchain\/} (``singlechain
projects'' for short); more sophisticated projects use (or, rather,
plan to use) {\em multiple blockchains\/} (``multichain projects'').

Singlechain projects are generally simpler and better tested; they
have withstood the test of time. Their main drawback is low
performance, or at least transaction throughput, which is on the level
of ten (Bitcoin) to less than one hundred\footnote{More like 15, for
  the time being. However, some upgrades are being planned to make
  Ethereum transaction throughput several times larger.}  (Ethereum)
transactions per second for general-purpose systems. Some specialized
systems (such as Bitshares) are capable of processing tens of
thousands of specialized transactions per second, at the expense of
requiring the blockchain state to fit into memory, and limiting the
processing to a predefined special set of transactions, which are then
executed by highly-optimized code written in languages like C++ (no
VMs here).

Multichain projects promise the scalability everybody craves. They may
support larger total states and more transactions per second, at the
expense of making the project much more complex, and its
implementation more challenging. As a result, there are few multichain
projects already running, but most proposed projects are
multichain. We believe that the future belongs to multichain projects.

\nxsubpoint\label{sp:pow.pos} \embt(Creating and validating blocks:
Proof-of-Work vs.\ Proof-of-Stake.)  Another important distinction is
the algorithm and protocol used to create and propagate new blocks,
check their validity, and select one of several forks if they appear.

The two most common paradigms are {\em Proof-of-Work (PoW)} and {\em
  Proof-of-Stake (PoS)}. The Proof-of-Work approach usually allows any
node to create (``mine'') a new block (and obtain some reward
associated with mining a block) if it is lucky enough to solve an
otherwise useless computational problem (usually involving the
computation of a large amount of hashes) before other competitors
manage to do this. In the case of forks (for example, if two nodes
publish two otherwise valid but different blocks to follow the
previous one) the longest fork wins. In this way, the guarantee of
immutability of the blockchain is based on the amount of {\em work\/}
(computational resources) spent to generate the blockchain: anybody
who would like to create a fork of this blockchain would need to re-do
this work to create alternative versions of the already committed
blocks. For this, one would need to control more than $50\%$ of the
total computing power spent creating new blocks, otherwise the
alternative fork will have exponentially low chances of becoming the
longest.

The Proof-of-Stake approach is based on large {\em stakes\/}
(nominated in cryptocurrency) made by some special nodes ({\em
  validators}) to assert that they have checked ({\em validated\/})
some blocks and have found them correct. Validators sign blocks, and
receive some small rewards for this; however, if a validator is ever
caught signing an incorrect block, and a proof of this is presented,
part or all of its stake is forfeit. In this way, the guarantee of
validity and immutability of the blockchain is given by the total
volume of stakes put by validators on the validity of the blockchain.

The Proof-of-Stake approach is more natural in the respect that it
incentivizes the validators (which replace PoW miners) to perform
useful computation (needed to check or create new blocks, in
particular, by performing all transactions listed in a block) instead
of computing otherwise useless hashes. In this way, validators would
purchase hardware that is better adapted to processing user
transactions, in order to receive rewards associated with these
transactions, which seems quite a useful investment from the
perspective of the system as a whole.

However, Proof-of-Stake systems are somewhat more challenging to
implement, because one must provide for many rare but possible
conditions. For example, some malicious validators might conspire to
disrupt the system to extract some profit (e.g., by altering
their own cryptocurrency balances). This leads to some non-trivial
game-theoretic problems.

In short, Proof-of-Stake is more natural and more promising,
especially for multichain projects (because Proof-of-Work would
require prohibitive amounts of computational resources if there are
many blockchains), but must be more carefully thought out and
implemented. Most currently running blockchain projects, especially
the oldest ones (such as Bitcoin and at least the original Ethereum),
use Proof-of-Work.

\nxsubpoint\label{sp:dpos.bft} \embt(Variants of Proof-of-Stake. DPOS
vs.\ BFT.)  While Proof-of-Work algorithms are very similar to each
other and differ mostly in the hash functions that must be computed
for mining new blocks, there are more possibilities for Proof-of-Stake
algorithms. They merit a sub-classification of their own.

Essentially, one must answer the following questions about a
Proof-of-Stake algorithm:
\begin{itemize}
\item Who can produce (``mine'') a new block---any full node, or only
  a member of a (relatively) small subset of validators?  (Most PoS
  systems require new blocks to be generated and signed by one of
  several designated validators.)
\item Do validators guarantee the validity of the blocks by their
  signatures, or are all full nodes expected to validate all blocks by
  themselves? (Scalable PoS systems must rely on validator signatures
  instead of requiring all nodes to validate all blocks of all
  blockchains.)
\item Is there a designated producer for the next blockchain block,
  known in advance, such that nobody else can produce that block
  instead?
\item Is a newly-created block originally signed by only one validator
  (its producer), or must it collect a majority of validator
  signatures from the very beginning?
\end{itemize}

While there seem to be $2^4$ possible classes of PoS algorithms
depending on the answers to these questions, the distinction in
practice boils down to two major approaches to PoS. In fact, most
modern PoS algorithms, designed to be used in scalable multi-chain
systems, answer the first two questions in the same fashion: only
validators can produce new blocks, and they guarantee block validity
without requiring all full nodes to check the validity of all blocks
by themselves.

As to the two last questions, their answers turn out to be highly
correlated, leaving essentially only two basic options:
\begin{itemize}
\item {\em Delegated Proof-of-Stake (DPOS)}: There is a universally
  known designated producer for every block; no one else can produce
  that block; the new block is originally signed only by its producing
  validator.
\item {\em Byzantine Fault Tolerant (BFT)} PoS algorithms: There is a
  known subset of validators, any of which can suggest a new block;
  the choice of the actual next block among several suggested
  candidates, which must be validated and signed by a majority of
  validators before being released to the other nodes, is achieved by
  a version of Byzantine Fault Tolerant consensus protocol.
\end{itemize}

\nxsubpoint\label{sp:dpos.bft.compare} \embt(Comparison of DPOS and
BFT PoS.)  The BFT approach has the advantage that a newly-produced
block has {\em from the very beginning\/} the signatures of a majority
of validators testifying to its validity. Another advantage is that,
if a majority of validators executes the BFT consensus protocol
correctly, no forks can appear at all. On the other hand, BFT
algorithms tend to be quite convoluted and require more time for the
subset of validators to reach consensus. Therefore, blocks cannot be
generated too often. This is why we expect the TON Blockchain (which
is a BFT project from the perspective of this classification) to
produce a block only once every five seconds. In practice, this
interval might be decreased to 2--3 seconds (though we do not promise
this), but not further, if validators are spread across the globe.

The DPOS algorithm has the advantage of being quite simple and
straightforward. It can generate new blocks quite often---say, once
every two seconds, or maybe even once every second,\footnote{Some
  people even claim DPOS block generation times of half a second,
  which does not seem realistic if validators are scattered across
  several continents.} because of its reliance on designated block
producers known in advance.

However, DPOS requires all nodes---or at least all validators---to
validate all blocks received, because a validator producing and
signing a new block confirms not only the {\em relative\/} validity of
this block, but also the validity of the previous block it refers to,
and all the blocks further back in the chain (maybe up to the
beginning of the period of responsibility of the current subset of
validators). There is a predetermined order on the current subset of
validators, so that for each block there is a designated producer
(i.e., validator expected to generate that block); these designated
producers are rotated in a round-robin fashion. In this way, a block
is at first signed only by its producing validator; then, when the
next block is mined, and its producer chooses to refer to this block
and not to one of its predecessors (otherwise its block would lie in a
shorter chain, which might lose the ``longest fork'' competition in
the future), the signature of the next block is essentially an
additional signature on the previous block as well. In this way, a new
block gradually collects the signatures of more validators---say,
twenty signatures in the time needed to generate the next twenty
blocks. A full node will either need to wait for these twenty
signatures, or validate the block by itself, starting from a
sufficiently confirmed block (say, twenty blocks back), which might be
not so easy.

The obvious disadvantage of the DPOS algorithm is that a new block
(and transactions committed into it) achieves the same level of trust
(``recursive reliability'' as discussed in~\ptref{sp:rec.rel}) only
after twenty more blocks are mined, compared to the BFT algorithms,
which deliver this level of trust (say, twenty signatures)
immediately. Another disadvantage is that DPOS uses the ``longest fork
wins'' approach for switching to other forks; this makes forks quite
probable if at least some producers fail to produce subsequent blocks
after the one we are interested in (or we fail to observe these blocks
because of a network partition or a sophisticated attack).

We believe that the BFT approach, while more sophisticated to
implement and requiring longer time intervals between blocks than
DPOS, is better adapted to ``tightly-coupled''
(cf.~\ptref{sp:blkch.interact}) multichain systems, because other
blockchains can start acting almost immediately after seeing a
committed transaction (e.g., generating a message intended for them)
in a new block, without waiting for twenty confirmations of validity
(i.e., the next twenty blocks), or waiting for the next six blocks to
be sure that no forks appear and verifying the new block by themselves
(verifying blocks of other blockchains may become prohibitive in a
scalable multi-chain system). Thus they can achieve scalability while
preserving high reliability and availability
(cf.~\ptref{sp:shard.supp}).

On the other hand, DPOS might be a good choice for a
``loosely-coupled'' multi-chain system, where fast interaction between
blockchains is not required -- e.g., if each blockchain
(``workchain'') represents a separate distributed exchange, and
inter-blockchain interaction is limited to rare transfers of tokens
from one workchain into another (or, rather, trading one altcoin
residing in one workchain for another at a rate approaching
$1:1$). This is what is actually done in the BitShares project, which
uses DPOS quite successfully.

To summarize, while DPOS can {\em generate\/} new blocks and {\em
  include transactions\/} into them {\em faster\/} (with smaller
intervals between blocks), these transactions reach the level of trust
required to use them in other blockchains and off-chain applications
as ``committed'' and ``immutable'' {\em much more slowly\/} than in
the BFT systems---say, in thirty seconds%
\footnote{For instance, EOS, one of the best DPOS projects proposed up
  to this date, promises a 45-second confirmation and inter-blockchain
  interaction delay (cf.~\cite{EOSWP}, ``Transaction Confirmation''
  and ``Latency of Interchain Communication'' sections).}
instead of five. Faster transaction {\em inclusion\/} does not mean
faster transaction {\em commitment}. This could become a huge problem
if fast inter-blockchain interaction is required. In that case, one
must abandon DPOS and opt for BFT PoS instead.

\nxsubpoint\label{sp:smartc.supp} \embt(Support for Turing-complete
code in transactions, i.e., essentially arbitrary smart contracts.)
Blockchain projects normally collect some {\em transactions\/} in
their blocks, which alter the blockchain state in a way deemed useful
(e.g., transfer some amount of cryptocurrency from one account to
another). Some blockchain projects might allow only some specific
predefined types of transactions (such as value transfers from one
account to another, provided correct signatures are presented). Others
might support some limited form of scripting in the
transactions. Finally, some blockchains support the execution of
arbitrarily complex code in transactions, enabling the system (at
least in principle) to support arbitrary applications, provided the
performance of the system permits. This is usually associated with
``Turing-complete virtual machines and scripting languages'' (meaning
that any program that can be written in any other computing language
may be re-written to be performed inside the blockchain), and ``smart
contracts'' (which are programs residing in the blockchain).

Of course, support for arbitrary smart contracts makes the system
truly flexible. On the other hand, this flexibility comes at a cost:
the code of these smart contracts must be executed on some virtual
machine, and this must be done every time for each transaction in the
block when somebody wants to create or validate a block. This slows
down the performance of the system compared to the case of a
predefined and immutable set of types of simple transactions, which
can be optimized by implementing their processing in a language such
as C++ (instead of some virtual machine).

Ultimately, support for Turing-complete smart contracts seems to be
desirable in any general-purpose blockchain project; otherwise, the
designers of the blockchain project must decide in advance which
applications their blockchain will be used for. In fact, the lack of
support for smart contracts in the Bitcoin blockchain was the
principal reason why a new blockchain project, Ethereum, had to be
created.

In a (heterogeneous; cf.~\ptref{sp:blkch.hom.het}) multi-chain system,
one might have ``the best of both worlds'' by supporting
Turing-complete smart contracts in some blockchains (i.e.,
workchains), and a small predefined set of highly-optimized
transactions in others.

\nxsubpoint\label{sp:class.multichain} \embt(Classification of
multichain systems.)  So far, the classification was valid both for
single-chain and multi-chain systems. However, multi-chain systems
admit several more classification criteria, reflecting the
relationship between the different blockchains in the system. We now
discuss these criteria.

\nxsubpoint\label{sp:blkch.hom.het} \embt(Blockchain types:
homogeneous and heterogeneous systems.)  In a multi-chain system, all
blockchains may be essentially of the same type and have the same
rules (i.e., use the same format of transactions, the same virtual
machine for executing smart-contract code, share the same
cryptocurrency, and so on), and this similarity is explicitly
exploited, but with different data in each blockchain. In this case,
we say that the system is {\em homogeneous}. Otherwise, different
blockchains (which will usually be called {\em workchains\/} in this
case) can have different ``rules''. Then we say that the system is
{\em heterogeneous}.

\nxsubpoint\label{sp:mixed.het.hom} \embt(Mixed
heterogeneous-homogeneous systems.)  Sometimes we have a mixed system,
where there are several sets of types or rules for blockchains, but
many blockchains with the same rules are present, and this fact is
explicitly exploited. Then it is a mixed {\em
  heterogeneous-homogeneous system}. To our knowledge, the TON
Blockchain is the only example of such a system.

\nxsubpoint\label{sp:het.confed} \embt(Heterogeneous systems with
several workchains having the same rules, or {\em confederations}.)
In some cases, several blockchains (work\-chains) with the same rules
can be present in a heterogeneous system, but the interaction between
them is the same as between blockchains with different rules (i.e.,
their similarity is not exploited explicitly). Even if they appear to
use ``the same'' cryptocurrency, they in fact use different
``altcoins'' (independent incarnations of the
cryptocurrency). Sometimes one can even have certain mechanisms to
convert these altcoins at a rate near to $1:1$. However, this does not
make the system homogeneous in our view; it remains heterogeneous. We
say that such a heterogeneous collection of workchains with the same
rules is a {\em confederation}.

While making a heterogeneous system that allows one to create several
work\-chains with the same rules (i.e., a confederation) may seem a
cheap way of building a scalable system, this approach has a lot of
drawbacks, too. Essentially, if someone hosts a large project in many
workchains with the same rules, she does not obtain a large project,
but rather a lot of small instances of this project. This is like
having a chat application (or a game) that allows having at most 50
members in any chat (or game) room, but ``scales'' by creating new
rooms to accommodate more users when necessary. As a result, a lot of
users can participate in the chats or in the game, but can we say that
such a system is truly scalable?

\nxsubpoint\label{sp:pres.masterch} \embt(Presence of a masterchain,
external or internal.)  Sometimes, a multi-chain project has a
distinguished ``masterchain'' (sometimes called ``control
blockchain''), which is used, for example, to store the overall
configuration of the system (the set of all active blockchains, or
rather workchains), the current set of validators (for a
Proof-of-Stake system), and so on. Sometimes other blockchains are
``bound'' to the masterchain, for example by committing the hashes of
their latest blocks into it (this is something the TON Blockchain
does, too).

In some cases, the masterchain is {\em external}, meaning that it is
not a part of the project, but some other pre-existing blockchain,
originally completely unrelated to its use by the new project and
agnostic of it. For example, one can try to use the Ethereum
blockchain as a masterchain for an external project, and publish
special smart contracts into the Ethereum blockchain for this purpose
(e.g., for electing and punishing validators).

\nxsubpoint\label{sp:shard.supp} \embt(Sharding support.)  Some
blockchain projects (or systems) have native support for {\em
  sharding}, meaning that several (necessarily homogeneous;
cf.~\ptref{sp:blkch.hom.het}) blockchains are thought of as {\em
  shards\/} of a single (from a high-level perspective) virtual
blockchain. For example, one can create 256 shard blockchains
(``shardchains'') with the same rules, and keep the state of an
account in exactly one shard selected depending on the first byte of
its $\accountid$.

Sharding is a natural approach to scaling blockchain systems, because,
if it is properly implemented, users and smart contracts in the system
need not be aware of the existence of sharding at all. In fact, one
often wants to add sharding to an existing single-chain project (such
as Ethereum) when the load becomes too high.

An alternative approach to scaling would be to use a ``confederation''
of heterogeneous workchains as described in~\ptref{sp:het.confed},
allowing each user to keep her account in one or several workchains of
her choice, and transfer funds from her account in one workchain to
another workchain when necessary, essentially performing a $1:1$
altcoin exchange operation. The drawbacks of this approach have
already been discussed in~\ptref{sp:het.confed}.

However, sharding is not so easy to implement in a fast and reliable
fashion, because it implies a lot of messages between different
shardchains. For example, if accounts are evenly distributed between
$N$ shards, and the only transactions are simple fund transfers from
one account to another, then only a small fraction ($1/N$) of all
transactions will be performed within a single blockchain; almost all
($1-1/N$) transactions will involve two blockchains, requiring
inter-blockchain communication. If we want these transactions to be
fast, we need a fast system for transferring messages between
shardchains. In other words, the blockchain project needs to be
``tightly-coupled'' in the sense described
in~\ptref{sp:blkch.interact}.

\nxsubpoint\label{sp:dyn.stat.shard} \embt(Dynamic and static
sharding.)  Sharding might be {\em dynamic\/} (if additional shards
are automatically created when necessary) or {\em static\/} (when
there is a predefined number of shards, which is changeable only
through a hard fork at best). Most sharding proposals are static; the
TON Blockchain uses dynamic sharding (cf.~\ptref{sect:split.merge}).

\nxsubpoint\label{sp:blkch.interact} \embt(Interaction between
blockchains: loosely-coupled and tightly-coupled systems.)
Multi-blockchain projects can be classified according to the supported
level of interaction between the constituent blockchains.

The least level of support is the absence of any interaction between
different blockchains whatsoever. We do not consider this case here,
because we would rather say that these blockchains are not parts of
one blockchain system, but just separate instances of the same
blockchain protocol.

The next level of support is the absence of any specific support for
messaging between blockchains, making interaction possible in
principle, but awkward. We call such systems ``loosely-coupled''; in
them one must send messages and transfer value between blockchains as
if they had been blockchains belonging to completely separate
blockchain projects (e.g., Bitcoin and Ethereum; imagine two parties
want to exchange some Bitcoins, kept in the Bitcoin blockchain, into
Ethers, kept in the Ethereum blockchain). In other words, one must
include the outbound message (or its generating transaction) in a
block of the source blockchain. Then she (or some other party) must
wait for enough confirmations (e.g., a given number of subsequent
blocks) to consider the originating transaction to be ``committed''
and ``immutable'', so as to be able to perform external actions based
on its existence. Only then may a transaction relaying the message
into the target blockchain (perhaps along with a reference and a
Merkle proof of existence for the originating transaction) be
committed.

If one does not wait long enough before transferring the message, or
if a fork happens anyway for some other reason, the joined state of
the two blockchains turns out to be inconsistent: a message is
delivered into the second blockchain that has never been generated in
(the ultimately chosen fork of) the first blockchain.

Sometimes partial support for messaging is added, by standardizing the
format of messages and the location of input and output message queues
in the blocks of all workchains (this is especially useful in
heterogeneous systems). While this facilitates messaging to a certain
extent, it is conceptually not too different from the previous case,
so such systems are still ``loosely-coupled''.

By contrast, ``tightly-coupled'' systems include special mechanisms to
provide fast messaging between all blockchains. The desired behavior
is to be able to deliver a message into another workchain immediately
after it has been generated in a block of the originating
blockchain. On the other hand, ``tightly-coupled'' systems are also
expected to maintain overall consistency in the case of forks. While
these two requirements appear to be contradictory at first glance, we
believe that the mechanisms used by the TON Blockchain (the inclusion
of shardchain block hashes into masterchain blocks; the use of
``vertical'' blockchains for fixing invalid blocks,
cf.~\ptref{sp:inv.sh.blk.corr}; hypercube routing,
cf.~\ptref{sp:hypercube}; Instant Hypercube Routing,
cf.~\ptref{sp:instant.hypercube}) enable it to be a
``tightly-coupled'' system, perhaps the only one so far.

Of course, building a ``loosely-coupled'' system is much simpler;
however, fast and efficient sharding (cf.~\ptref{sp:shard.supp})
requires the system to be ``tightly-coupled''.

\nxsubpoint\label{sp:blkch.gen} \embt(Simplified
classification. Generations of blockchain projects.)  The
classification we have suggested so far splits all blockchain projects
into a large number of classes. However, the classification criteria
we use happen to be quite correlated in practice. This enables us to
suggest a simplified ``generational'' approach to the classification
of blockchain projects, as a very rough approximation of reality,
with some examples. Projects that have not been implemented and
deployed yet are shown in {\em italics}; the most important
characteristics of a generation are shown in {\bf bold}.
\begin{itemize}
\item First generation: Single-chain, {\bf PoW}, no support for smart
  contracts. Examples: Bitcoin (2009) and a lot of otherwise
  uninteresting imitators (Litecoin, Monero, \dots).
\item Second generation: Single-chain, PoW, {\bf smart-contract
  support}. Example: Ethereum (2013; deployed in 2015), at least in
  its original form.
\item Third generation: Single-chain, {\bf PoS}, smart-contract
  support. Example: {\em future Ethereum} (2018 or later).
\item Alternative third ($3'$) generation: {\bf Multi-chain}, PoS, no
  support for smart contracts, loosely-coupled. Example: Bitshares
  (2013--2014; uses DPOS).
\item Fourth generation: {\bf Multi-chain, PoS, smart-contract
  support}, loosely-coupled. Examples: {\em EOS\/} (2017; uses DPOS),
  {\em PolkaDot\/} (2016; uses BFT).
\item Fifth generation: Multi-chain, PoS with BFT, smart-contract
  support, {\bf tightly-coupled, with sharding}. Examples: {\em TON\/}
  (2017).
\end{itemize}
While not all blockchain projects fall precisely into one of these
categories, most of them do.

\nxsubpoint\label{sp:genome.change.never} \embt(Complications of
changing the ``genome'' of a blockchain project.)  The above
classification defines the ``genome'' of a blockchain project. This
genome is quite ``rigid'': it is almost impossible to change it once
the project is deployed and is used by a lot of people. One would need
a series of hard forks (which would require the approval of the
majority of the community), and even then the changes would need to be
very conservative in order to preserve backward compatibility (e.g.,
changing the semantics of the virtual machine might break existing
smart contracts). An alternative would be to create new ``sidechains''
with their different rules, and bind them somehow to the blockchain
(or the blockchains) of the original project. One might use the
blockchain of the existing single-blockchain project as an external
masterchain for an essentially new and separate project.\footnote{For
  example, the Plasma project plans to use the Ethereum blockchain as
  its (external) masterchain; it does not interact much with Ethereum
  otherwise, and it could have been suggested and implemented by a
  team unrelated to the Ethereum project.}

Our conclusion is that the genome of a project is very hard to change
once it has been deployed. Even starting with PoW and planning to
replace it with PoS in the future is quite complicated.\footnote{As of
  2017, Ethereum is still struggling to transition from PoW to a
  combined PoW+PoS system; we hope it will become a truly PoS system
  someday.} Adding shards to a project originally designed without
support for them seems almost impossible.\footnote{There are sharding
  proposals for Ethereum dating back to 2015; it is unclear how they
  might be implemented and deployed without disrupting Ethereum or
  creating an essentially independent parallel project.} In fact,
adding support for smart contracts into a project (namely, Bitcoin)
originally designed without support for such features has been deemed
impossible (or at least undesirable by the majority of the Bitcoin
community) and eventually led to the creation of a new blockchain
project, Ethereum.

\nxsubpoint \embt(Genome of the TON Blockchain.)  Therefore, if one
wants to build a scalable blockchain system, one must choose its
genome carefully from the very beginning. If the system is meant to
support some additional specific functionality in the future not known
at the time of its deployment, it should support ``heterogeneous''
workchains (having potentially different rules) from the start. For
the system to be truly scalable, it must support sharding from the
very beginning; sharding makes sense only if the system is
``tightly-coupled'' (cf.~\ptref{sp:blkch.interact}), so this in turn
implies the existence of a masterchain, a fast system of
inter-blockchain messaging, usage of BFT PoS, and so on.

When one takes into account all these implications, most of the design
choices made for the TON Blockchain project appear natural, and almost
the only ones possible.

\mysubsection{Comparison to Other Blockchain
  Projects}\label{sect:compare.blkch}

We conclude our brief discussion of the TON Blockchain and its most
important and unique features by trying to find a place for it on a
map containing existing and proposed blockchain projects. We use the
classification criteria described in~\ptref{sect:class.blkch} to
discuss different blockchain projects in a uniform way and construct
such a ``map of blockchain projects''. We represent this map as
Table~\ref{tab:blkch.proj}, and then briefly discuss a few projects
separately to point out their peculiarities that may not fit into the
general scheme.

\begin{table}
  \captionsetup{font=scriptsize}
  \begin{tabular}{|c|cc|ccc|ccc|}
    \hline Project & Year & G. & Cons. & Sm. & Ch. & R. & Sh. &
    Int. \\ \hline Bitcoin & 2009 & 1 & PoW & no & 1 \\ Ethereum &
    2013, 2015 & 2 & PoW & yes & 1 \\ NXT & 2014 & 2+ & PoS & no & 1
    \\ Tezos & 2017, ? & 2+ & PoS & yes & 1 \\ Casper & 2015, (2017) &
    3 & PoW/PoS & yes & 1 \\ \hline BitShares & 2013, 2014 & $3'$ &
    DPoS & no & m & ht. & no & L \\ EOS & 2016, (2018) & 4 & DPoS &
    yes & m & ht. & no & L \\ PolkaDot & 2016, (2019) & 4 & PoS BFT & yes &
    m & ht. & no & L \\ Cosmos & 2017, ?  & 4 & PoS BFT & yes & m &
    ht. & no & L \\ TON & 2017, (2018) & 5 & PoS BFT & yes & m & mix &
    dyn. & T \\ \hline
  \end{tabular}
  \caption{A summary of some notable blockchain projects. The columns
    are: {\em Project} -- project name; {\em Year} -- year announced
    and year deployed; {\em G.} -- generation
    (cf.~\ptref{sp:blkch.gen}); {\em Cons.} -- consensus algorithm
    (cf.~\ptref{sp:pow.pos} and~\ptref{sp:dpos.bft}); {\em Sm.} --
    support for arbitrary code (smart contracts;
    cf.~\ptref{sp:smartc.supp}); {\em Ch.} -- single/multiple
    blockchain system (cf.~\ptref{sp:single.multi}); {\em R.} --
    heterogeneous/homogeneous multichain systems
    (cf.~\ptref{sp:blkch.hom.het}); {\em Sh.} -- sharding support
    (cf.~\ptref{sp:shard.supp}); {\em Int.} -- interaction between
    blockchains, (L)oose or (T)ight (cf.~\ptref{sp:blkch.interact}).
  }\label{tab:blkch.proj}
\end{table}

\nxsubpoint \embt(Bitcoin \cite{BitcWP}; \url{https://bitcoin.org/}.)
            {\em Bitcoin\/} (2009) is the first and the most famous
            block\-chain project. It is a typical {\em
              first-generation} blockchain project: it is
            single-chain, it uses Proof-of-Work with a
            ``longest-fork-wins'' fork selection algorithm, and it
            does not have a Turing-complete scripting language
            (however, simple scripts without loops are supported). The
            Bitcoin blockchain has no notion of an account; it uses
            the UTXO (Unspent Transaction Output) model instead.

\nxsubpoint \embt(Ethereum \cite{EthWP}; \url{https://ethereum.org/}.)
            {\em Ethereum\/} (2015) is the first blockchain with
            support for Turing-complete smart contracts. As such, it
            is a typical {\em second-generation\/} project, and the
            most popular among them. It uses Proof-of-Work on a single
            blockchain, but has smart contracts and accounts.

\nxsubpoint \embt(NXT; \url{https://nxtplatform.org/}.)  {\em NXT\/}
(2014) is the first PoS-based blockchain and currency. It is still
single-chain, and has no smart contract support.

\nxsubpoint \embt(Tezos; \url{https://www.tezos.com/}.)  {\em Tezos\/}
(2018 or later) is a proposed PoS-based single-blockchain project. We
mention it here because of its unique feature: its block
interpretation function $\evblock$ (cf.~\ptref{sp:blk.transf}) is not
fixed, but is determined by an OCaml module, which can be upgraded by
committing a new version into the blockchain (and collecting some
votes for the proposed change). In this way, one will be able to
create custom single-chain projects by first deploying a ``vanilla''
Tezos blockchain, and then gradually changing the block interpretation
function in the desired direction, without any need for hard forks.

This idea, while intriguing, has the obvious drawback that it forbids
any optimized implementations in other languages like C++, so a
Tezos-based blockchain is destined to have lower performance. We think
that a similar result might have been obtained by publishing a formal
{\em specification\/} of the proposed block interpretation function
$\evtrans$, without fixing a particular {\em implementation}.

\nxsubpoint
\embt(Casper.)%
\footnote{\url{https://blog.ethereum.org/2015/08/01/introducing-casper-friendly-ghost/}}
{\em Casper\/} is an upcoming PoS algorithm for Ethereum; its gradual
deployment in 2017 (or 2018), if successful, will change Ethereum into
a single-chain PoS or mixed PoW+PoS system with smart contract
support, transforming Ethereum into a {\em third-generation\/}
project.

\nxsubpoint \embt(BitShares \cite{BitShWP};
\url{https://bitshares.org}.)  {\em BitShares\/} (2014) is a platform
for distributed blockchain-based exchanges. It is a heterogeneous
multi-blockchain DPoS system without smart contracts; it achieves its
high performance by allowing only a small set of predefined
specialized transaction types, which can be efficiently implemented in
C++, assuming the blockchain state fits into memory. It is also the
first blockchain project to use Delegated Proof-of-Stake (DPoS),
demonstrating its viability at least for some specialized purposes.

\nxsubpoint\label{sp:discuss.EOS} \embt(EOS \cite{EOSWP};
\url{https://eos.io}.)  {\em EOS\/} (2018 or later) is a proposed
heterogeneous multi-blockchain DPoS system {\em with\/} smart contract
support and with some minimal support for messaging (still
loosely-coupled in the sense described
in~\ptref{sp:blkch.interact}). It is an attempt by the same team that
has previously successfully created the BitShares and SteemIt
projects, demonstrating the strong points of the DPoS consensus
algorithm. Scalability will be achieved by creating specialized
workchains for projects that need it (e.g., a distributed exchange
might use a workchain supporting a special set of optimized
transactions, similarly to what BitShares did) and by creating
multiple workchains with the same rules ({\em confederations\/} in the
sense described in~\ptref{sp:het.confed}). The drawbacks and
limitations of this approach to scalability have been discussed in
{\em loc.~cit.} Cf.\ also \ptref{sp:dpos.bft.compare},
\ptref{sp:shard.supp}, and \ptref{sp:blkch.interact} for a more
detailed discussion of DPoS, sharding, interaction between workchains
and their implications for the scalability of a blockchain system.

    At the same time, even if one will not be able to ``create a
    Facebook inside a blockchain''
    (cf.~\ptref{sp:blockchain.facebook}), EOS or otherwise, we think
    that EOS might become a convenient platform for some
    highly-specialized weakly interacting distributed applications,
    similar to BitShares (decentralized exchange) and SteemIt
    (decentralized blog platform).

\nxsubpoint\label{sp:discuss.PolkaDot} \embt(PolkaDot \cite{PolkaWP};
\url{https://polkadot.io/}.)  {\em PolkaDot\/} (2019 or later) is one
of the best thought-out and most detailed proposed multichain
Proof-of-Stake projects; its development is led by one of the
Ethereum co-founders. This project is one of the closest projects to
the TON Blockchain on our map. (In fact, we are indebted for our
terminology for ``fishermen'' and ``nominators'' to the PolkaDot
project.)

PolkaDot is a heterogeneous loosely-coupled multichain Proof-of-Stake
project, with Byzantine Fault Tolerant (BFT) consensus for generation
of new blocks and a masterchain (which might be external---e.g., the
Ethereum blockchain). It also uses hypercube routing, somewhat like
(the slow version of) TON's as described in~\ptref{sp:hypercube}.

Its unique feature is its ability to create not only {\em public}, but
also {\em private\/} blockchains. These private blockchains would also
be able to interact with other public blockchains, PolkaDot or
otherwise.

As such, PolkaDot might become a platform for large-scale {\em
  private\/} block\-chains, which might be used, for example, by bank
consortiums to quickly transfer funds to each other, or for any other
uses a large corporation might have for private blockchain technology.

However, PolkaDot has no sharding support and is not
tightly-coupled. This somewhat hampers its scalability, which is
similar to that of EOS. (Perhaps a bit better, because PolkaDot uses
BFT PoS instead of DPoS.)

\nxsubpoint \embt(Universa; \url{https://universa.io}.)  The only
reason we mention this unusual blockchain project here is because it
is the only project so far to make in passing an explicit reference to
something similar to our Infinite Sharding Paradigm
(cf.~\ptref{sp:ISP}). Its other peculiarity is that it bypasses all
complications related to Byzantine Fault Tolerance by promising that
only trusted and licensed partners of the project will be admitted as
validators, hence they will never commit invalid blocks. This is an
interesting decision; however, it essentially makes a blockchain
project deliberately {\em centralized}, something blockchain projects
usually want to avoid (why does one need a blockchain at all to work
in a trusted centralized environment?).

\nxsubpoint \embt(Plasma; \url{https://plasma.io}).)  {\em Plasma\/}
(2019?) is an unconventional blockchain project from another
co-founder of Ethereum. It is supposed to mitigate some limitations of
Ethereum without introducing sharding. In essence, it is a separate
project from Ethereum, introducing a hierarchy of (heterogeneous)
workchains, bound to the Ethereum blockchain (to be used as an
external masterchain) at the top level. Funds can be transferred from
any blockchain up in the hierarchy (starting from the Ethereum
blockchain as the root), along with a description of a job to be
done. Then the necessary computations are done in the child workchain
(possibly requiring forwarding of parts of the original job further
down the tree), their results are passed up, and a reward is
collected. The problem of achieving consistency and validating these
workchains is circumvented by a (payment channel-inspired) mechanism
allowing users to unilaterally withdraw their funds from a misbehaving
workchain to its parent workchain (albeit slowly), and re-allocate
their funds and their jobs to another workchain.

In this way, Plasma might become a platform for distributed
computations bound to the Ethereum blockchain, something like a
``mathematical co-processor''. However, this does not seem like a way
to achieve true general-purpose scalability.

\nxsubpoint \embt(Specialized blockchain projects.)  There are also
some specialized blockchain projects, such as FileCoin (a system that
incentivizes users to offer their disk space for storing the files of
other users who are willing to pay for it), Golem (a blockchain-based
platform for renting and lending computing power for specialized
applications such as 3D-rendering) or SONM (another similar computing
power-lending project). Such projects do not introduce anything
conceptually new on the level of blockchain organization; rather, they
are particular blockchain applications, which could be implemented by
smart contracts running in a general-purpose blockchain, provided it
can deliver the required performance. As such, projects of this kind
are likely to use one of the existing or planned blockchain projects
as their base, such as EOS, PolkaDot or TON. If a project needs
``true'' scalability (based on sharding), it would better use TON; if
it is content to work in a ``confederated'' context by defining a
family of workchains of its own, explicitly optimized for its purpose,
it might opt for EOS or PolkaDot.

\nxsubpoint \embt(The TON Blockchain.)  The TON (Telegram Open
Network) Block\-chain (planned 2018) is the project we are describing
in this document. It is designed to be the first fifth-generation
blockchain project---that is, a BFT PoS-multichain project, mixed
homogeneous/heterogeneous, with support for (shardable) custom
workchains, with native sharding support, and tightly-coupled (in
particular, capable of forwarding messages between shards almost
instantly while preserving a consistent state of all shardchains). As
such, it will be a truly scalable general-purpose blockchain project,
capable of accommodating essentially any applications that can be
implemented in a blockchain at all. When augmented by the other
components of the TON Project (cf.~\ptref{sect:ton.components}), its
possibilities expand even further.

\nxsubpoint\label{sp:blockchain.facebook} \embtx(Is it possible to
``upload Facebook into a blockchain''?)  Sometimes people claim that
it will be possible to implement a social network on the scale of
Facebook as a distributed application residing in a
blockchain. Usually a favorite blockchain project is cited as a
possible ``host'' for such an application.

We cannot say that this is a technical impossibility. Of course, one
needs a tightly-coupled blockchain project with true sharding (i.e.,
TON) in order for such a large application not to work too slowly
(e.g., deliver messages and updates from users residing in one
shardchain to their friends residing in another shardchain with
reasonable delays). However, we think that this is not needed and will
never be done, because the price would be prohibitive.

Let us consider ``uploading Facebook into a blockchain'' as a thought
experiment; any other project of similar scale might serve as an
example as well. Once Facebook is uploaded into a blockchain, all
operations currently done by Facebook's servers will be serialized as
transactions in certain blockchains (e.g., TON's shardchains), and
will be performed by all validators of these blockchains. Each
operation will have to be performed, say, at least twenty times, if we
expect every block to collect at least twenty validator signatures
(immediately or eventually, as in DPOS systems). Similarly, all data
kept by Facebook's servers on their disks will be kept on the disks of
all validators for the corresponding shardchain (i.e., in at least
twenty copies).

Because the validators are essentially the same servers (or perhaps
clusters of servers, but this does not affect the validity of this
argument) as those currently used by Facebook, we see that the total
hardware expenses associated with running Facebook in a blockchain are
at least twenty times higher than if it were implemented in the
conventional way.

In fact, the expenses would be much higher still, because the
blockchain's virtual machine is slower than the ``bare CPU'' running
optimized compiled code, and its storage is not optimized for
Facebook-specific problems. One might partially mitigate this problem
by crafting a specific workchain with some special transactions
adapted for Facebook; this is the approach of BitShares and EOS to
achieving high performance, available in the TON Blockchain as
well. However, the general blockchain design would still impose some
additional restrictions by itself, such as the necessity to register
all operations as transactions in a block, to organize these
transactions in a Merkle tree, to compute and check their Merkle
hashes, to propagate this block further, and so on.

Therefore, a conservative estimate is that one would need 100 times
more servers of the same performance as those used by Facebook now in
order to validate a blockchain project hosting a social network of
that scale. Somebody will have to pay for these servers, either the
company owning the distributed application (imagine seeing 700 ads on
each Facebook page instead of 7) or its users. Either way, this does
not seem economically viable.

We believe that {\em it is not true that everything should be uploaded
  into the blockchain}. For example, it is not necessary to keep user
photographs in the blockchain; registering the hashes of these
photographs in the blockchain and keeping the photographs in a
distributed off-chain storage (such as FileCoin or TON Storage) would
be a better idea. This is the reason why TON is not just a blockchain
project, but a collection of several components (TON P2P Network, TON
Storage, TON Services) centered around the TON Blockchain as outlined
in Chapters~\ptref{sect:ton.components} and~\ptref{sect:services}.

%%%%%%%%%%%%%%%%%%%%%%%%%%%%%%%%%%%%%%%%%%%%%%%%%
%
%
%                  NETWORK
%
%
%%%%%%%%%%%%%%%%%%%%%%%%%%%%%%%%%%%%%%%%%%%%%%%%%

\clearpage
\mysection{TON Networking}\label{sect:network}

Any blockchain project requires not only a specification of block
format and blockchain validation rules, but also a network protocol
used to propagate new blocks, send and collect transaction candidates
and so on. In other words, a specialized peer-to-peer network must be
set up by every blockchain project. This network must be peer-to-peer,
because blockchain projects are normally expected to be decentralized,
so one cannot rely on a centralized group of servers and use
conventional client-server architecture, as, for instance, classical
online banking applications do. Even light clients (e.g., light
cryptocurrency wallet smartphone applications), which must connect to
full nodes in a client-server--like fashion, are actually free to
connect to another full node if their previous peer goes down,
provided the protocol used to connect to full nodes is standardized
enough.

While the networking demands of single-blockchain projects, such as
Bitcoin or Ethereum, can be met quite easily (one essentially needs to
construct a ``random'' peer-to-peer overlay network, and propagate all
new blocks and transaction candidates by a gossip protocol),
multi-blockchain projects, such as the TON Blockchain, are much more
demanding (e.g., one must be able to subscribe to updates of only some
shardchains, not necessarily all of them). Therefore, the networking
part of the TON Blockchain and the TON Project as a whole merits at
least a brief discussion.

On the other hand, once the more sophisticated network protocols
needed to support the TON Blockchain are in place, it turns out that
they can easily be used for purposes not necessarily related to the
immediate demands of the TON Blockchain, thus providing more
possibilities and flexibility for creating new services in the TON
ecosystem.

\mysubsection{Abstract Datagram Network Layer}\label{sect:ANL}

The cornerstone in building the TON networking protocols is the {\em
  (TON) Abstract (Datagram) Network Layer}. It enables all nodes to
assume certain ``network identities'', represented by 256-bit
``abstract network addresses'', and communicate (send datagrams to
each other, as a first step) using only these 256-bit network
addresses to identify the sender and the receiver. In particular, one
does not need to worry about IPv4 or IPv6 addresses, UDP port numbers,
and the like; they are hidden by the Abstract Network Layer.

\nxsubpoint\label{sp:abs.addr} \embt(Abstract network addresses.)
An {\em abstract network address}, or an {\em abstract address}, or
just {\em address\/} for short, is a 256-bit integer, essentially
equal to a 256-bit ECC public key. This public key can be generated
arbitrarily, thus creating as many different network identities as the
node likes. However, one must know the corresponding {\em private\/}
key in order to receive (and decrypt) messages intended for such an
address.

In fact, the address is {\em not\/} the public key itself; instead, it
is a 256-bit hash ($\Hash=\Sha$) of a serialized TL-object
(cf.~\ptref{sp:TL}) that can describe several types of public keys and
addresses depending on its constructor (first four bytes). In the
simplest case, this serialized TL-object consists just of a 4-byte
magic number and a 256-bit elliptic curve cryptography (ECC) public
key; in this case, the address will equal the hash of this 36-byte
structure. One might use, however, 2048-bit RSA keys, or any other
scheme of public-key cryptography instead.

When a node learns another node's abstract address, it must also
receive its ``preimage'' (i.e., the serialized TL-object, the hash of
which equals that abstract address) or else it will not be able to
encrypt and send datagrams to that address.

\nxsubpoint \embt(Lower-level networks. UDP implementation.)  From the
perspective of almost all TON Networking components, the only thing
that exists is a network (the Abstract Datagram Networking Layer) able
to (unreliably) send datagrams from one abstract address to
another. In principle, the Abstract Datagram Networking Layer (ADNL)
can be implemented over different existing network
technologies. However, we are going to implement it over UDP in
IPv4/IPv6 networks (such as the Internet or intranets), with an
optional TCP fallback if UDP is not available.

\nxsubpoint\label{sp:net.simple.dg} \embt(Simplest case of ADNL over
UDP.)  The simplest case of sending a datagram from a sender's
abstract address to any other abstract address (with known preimage)
can be implemented as follows.

Suppose that the sender somehow knows the IP-address and the UDP port
of the receiver who owns the destination abstract address, and that
both the receiver and the sender use abstract addresses derived from
256-bit ECC public keys.

In this case, the sender simply augments the datagram to be sent by
its ECC signature (done with its private key) and its source address
(or the preimage of the source address, if the receiver is not known
to know that preimage yet). The result is encrypted with the
recipient's public key, embedded into a UDP datagram and sent to the
known IP and port of the recipient. Because the first 256 bits of the
UDP datagram contain the recipient's abstract address, the recipient
can identify which private key should be used to decrypt the remainder
of the datagram. Only after that is the sender's identity revealed.

\nxsubpoint\label{sp:net.simplest.dg} \embt(Less secure way, with the
sender's address in plaintext.)  Sometimes a less secure scheme is
sufficient, when the recipient's and the sender's addresses are kept
in plaintext in the UDP datagram; the sender's private key and the
recipient's public key are combined together using ECDH (Elliptic
Curve Diffie--Hellman) to generate a 256-bit shared secret, which is
used afterwards, along with a random 256-bit nonce also included in
the unencrypted part, to derive AES keys used for encryption. The
integrity may be provided, for instance, by concatenating the hash of
the original plaintext data to the plaintext before encryption.

This approach has the advantage that, if more than one datagram is
expected to be exchanged between the two addresses, the shared secret
can be computed only once and then cached; then slower elliptic curve
operations will no longer be required for encrypting or decrypting the
next datagrams.

\nxsubpoint\label{sp:net.channels} \embt(Channels and channel
identifiers.)  In the simplest case, the first 256 bits of a UDP
datagram carrying an embedded TON ADNL datagram will be equal to the
recipient's address. However, in general they constitute a {\em
  channel identifier}. There are different types of channels. Some of
them are point-to-point; they are created by two parties who wish to
exchange a lot of data in the future and generate a shared secret by
exchanging several packets encrypted as described
in~\ptref{sp:net.simple.dg} or~\ptref{sp:net.simplest.dg}, by running
classical or elliptic curve Diffie--Hellman (if extra security is
required), or simply by one party generating a random shared secret
and sending it to the other party.

After that, a channel identifier is derived from the shared secret
combined with some additional data (such as the sender's and
recipient's addresses), for instance by hashing, and that identifier
is used as the first 256 bits of UDP datagrams carrying data encrypted
with the aid of that shared secret.

\nxsubpoint\label{sp:tunnels} \embt(Channel as a tunnel identifier.)
In general, a ``channel'', or ``channel identifier'' simply selects a
way of processing an inbound UDP datagram, known to the receiver. If
the channel is the receiver's abstract address, the processing is done
as outlined in~\ptref{sp:net.simple.dg} or \ptref{sp:net.simplest.dg};
if the channel is an established point-to-point channel discussed
in~\ptref{sp:net.channels}, the processing consists in decrypting the
datagram with the aid of the shared secret as explained in {\em
  loc.~cit.}, and so on.

In particular, a channel identifier can actually select a ``tunnel'',
when the immediate recipient simply forwards the received message to
somebody else---the actual recipient or another proxy. Some encryption
or decryption steps (reminiscent of ``onion routing'' \cite{Onion} or
even ``garlic
routing''\footnote{\url{https://geti2p.net/en/docs/how/garlic-routing}})
might be done along the way, and another channel identifier might be
used for re-encrypted forwarded packets (for example, a peer-to-peer
channel could be employed to forward the packet to the next recipient
on the path).

In this way, some support for ``tunneling'' and
``proxying''---somewhat similar to that provided by the TOR or $I^2P$
projects---can be added on the level of the TON Abstract Datagram
Network Layer, without affecting the functionality of all higher-level
TON network protocols, which would be agnostic of such an
addition. This opportunity is exploited by the {\em TON Proxy\/}
service (cf.~\ptref{sp:ex.ton.proxy}).

\nxsubpoint\label{sp:net.startup} \embt(Zero channel and the bootstrap
problem.)  Normally, a TON ADNL node will have some ``neighbor
table'', containing information about other known nodes, such as their
abstract addresses and their preimages (i.e., public keys) and their
IP addresses and UDP ports. Then it will gradually extend this table
by using information learned from these known nodes as answers to
special queries, and sometimes prune obsolete records.

However, when a TON ADNL node just starts up, it may happen that it
does not know any other node, and can learn only the IP address and
UDP port of a node, but not its abstract address. This happens, for
example, if a light client is not able to access any of the previously
cached nodes and any nodes hardcoded into the software, and must ask
the user to enter an IP address or a DNS domain of a node, to be
resolved through DNS.

In this case, the node will send packets to a special ``zero channel''
of the node in question. This does not require knowledge of the
recipient's public key (but the message should still contain the
sender's identity and signature), so the message is transferred
without encryption.  It should be normally used only to obtain an
identity (maybe a one-time identity created especially for this
purpose) of the receiver, and then to start communicating in a safer
way.

Once at least one node is known, it is easy to populate the ``neighbor
table'' and ``routing table'' by more entries, learning them from
answers to special queries sent to the already known nodes.

Not all nodes are required to process datagrams sent to the zero
channel, but those used to bootstrap light clients should support this
feature.

\nxsubpoint \embt(TCP-like stream protocol over ADNL.)  The ADNL,
being an unreliable (small-size) datagram protocol based on 256-bit
abstract addresses, can be used as a base for more sophisticated
network protocols. One can build, for example, a TCP-like stream
protocol, using ADNL as an abstract replacement for IP. However, most
components of the TON Project do not need such a stream protocol.

\nxsubpoint\label{sp:RLDP} \embt(RLDP, or Reliable Large Datagram
Protocol over ADNL.)  A reliable arbitrary-size datagram protocol
built upon the ADNL, called RLDP, is used instead of a TCP-like
protocol. This reliable datagram protocol can be employed, for
instance, to send RPC queries to remote hosts and receive answers from
them (cf.~\ptref{sp:pure.net.serv}).

\mysubsection{TON DHT: Kademlia-like Distributed Hash
  Table}\label{sect:kademlia}

The {\em TON Distributed Hash Table (DHT)\/} plays a crucial role in
the networking part of the TON Project, being used to locate other
nodes in the network. For example, a client wanting to commit a
transaction into a shardchain might want to find a validator or a
collator of that shardchain, or at least some node that might relay
the client's transaction to a collator. This can be done by looking up
a special key in the TON DHT. Another important application of the TON
DHT is that it can be used to quickly populate a new node's neighbor
table (cf.~\ptref{sp:net.startup}), simply by looking up a random key,
or the new node's address. If a node uses proxying and tunneling for
its inbound datagrams, it publishes the tunnel identifier and its
entry point (e.g., IP address and UDP port) in the TON DHT; then all
nodes wishing to send datagrams to that node will obtain this contact
information from the DHT first.

The TON DHT is a member of the family of {\em Kademlia-like distributed
  hash tables\/}~\cite{Kademlia}.

\nxsubpoint \embt(Keys of the TON DHT.)  The {\em keys\/} of the TON
DHT are simply 256-bit integers. In most cases, they are computed as
$\Sha$ of a TL-serialized object (cf.~\ptref{sp:TL}), called {\em
  preimage\/} of the key, or {\em key description}. In some cases, the
abstract addresses of the TON Network nodes (cf.~\ptref{sp:abs.addr})
can also be used as keys of the TON DHT, because they are also
256-bit, and they are also hashes of TL-serialized objects. For
example, if a node is not afraid of publishing its IP address, it can
be found by anybody who knows its abstract address by simply looking
up that address as a key in the DHT.

\nxsubpoint \embt(Values of the DHT.)  The {\em values\/} assigned to
these 256-bit keys are essentially arbitrary byte strings of limited
length. The interpretation of such byte strings is determined by the
preimage of the corresponding key; it is usually known both by the
node that looks up the key, and by the node that stores the key.

\nxsubpoint \embt(Nodes of the DHT. Semi-permanent network
identities.)  The key-value mapping of the TON DHT is kept on the {\em
  nodes\/} of the DHT---essentially, all members of the TON
Network. To this end, any node of the TON Network (perhaps with the
exception of some very light nodes), apart from any number of
ephemeral and permanent abstract addresses described
in~\ptref{sp:abs.addr}, has at least one ``semi-permanent address'',
which identifies it as a member of the TON DHT. This {\em
  semi-permanent\/} or {\em DHT address\/} should not to be changed
too often, otherwise other nodes would be unable to locate the keys
they are looking for. If a node does not want to reveal its ``true''
identity, it generates a separate abstract address to be used only for
the purpose of participating in the DHT. However, this abstract
address must be public, because it will be associated with the node's
IP address and port.

\nxsubpoint \embt(Kademlia distance.)  Now we have both 256-bit keys
and 256-bit (semi-permanent) node addresses. We introduce the
so-called {\em XOR distance\/} or {\em Kademlia distance~$d_K$} on the
set of 256-bit sequences, given by
\begin{equation}
  d_K(x,y):=(x\oplus y)\quad\text{interpreted as an unsigned 256-bit
    integer}
\end{equation}
Here $x\oplus y$ denotes the bitwise eXclusive OR (XOR) of two bit
sequences of the same length.

The Kademlia distance introduces a metric on the set $\st2^{256}$ of
all 256-bit sequences. In particular, we have $d_K(x,y)=0$ if and only
if $x=y$, $d_K(x,y)=d_K(y,x)$, and $d_K(x,z)\leq
d_K(x,y)+d_K(y,z)$. Another important property is that {\em there is
  only one point at any given distance from~$x$}: $d_K(x,y)=d_K(x,y')$
implies $y=y'$.

\nxsubpoint \embt(Kademlia-like DHTs and the TON DHT.)  We say that a
distributed hash table (DHT) with 256-bit keys and 256-bit node
addresses is a {\em Kademlia-like DHT\/} if it is expected to keep the
value of key $K$ on $s$ Kademlia-nearest nodes to $K$ (i.e., the $s$
nodes with smallest Kademlia distance from their addresses to $K$.)

Here $s$ is a small parameter, say, $s=7$, needed to improve
reliability of the DHT (if we would keep the key only on one node, the
nearest one to~$K$, the value of that key would be lost if that only
node goes offline).

The TON DHT is a Kademlia-like DHT, according to this definition. It
is implemented over the ADNL protocol described in~\ptref{sect:ANL}.

\nxsubpoint \embt(Kademlia routing table.)  Any node participating in
a Kademlia-like DHT usually maintains a {\em Kademlia routing
  table}. In the case of TON DHT, it consists of $n=256$ buckets,
numbered from $0$ to $n-1$. The $i$-th bucket will contain information
about some known nodes (a fixed number $t$ of ``best'' nodes, and
maybe some extra candidates) that lie at a Kademlia distance from
$2^i$ to $2^{i+1}-1$ from the node's address $a$.\footnote{If there
  are sufficiently many nodes in a bucket, it can be subdivided
  further into, say, eight sub-buckets depending on the top four bits
  of the Kademlia distance. This would speed up DHT lookups.} This
information includes their (semi-permanent) addresses, IP addresses
and UDP ports, and some availability information such as the time and
the delay of the last ping.

When a Kademlia node learns about any other Kademlia node as a result
of some query, it includes it into a suitable bucket of its routing
table, first as a candidate. Then, if some of the ``best'' nodes in
that bucket fail (e.g., do not respond to ping queries for a long
time), they can be replaced by some of the candidates. In this way the
Kademlia routing table stays populated.

New nodes from the Kademlia routing table are also included in the
ADNL neighbor table described in~\ptref{sp:net.startup}. If a ``best''
node from a bucket of the Kademlia routing table is used often, a
channel in the sense described in~\ptref{sp:net.channels} can be
established to facilitate the encryption of datagrams.

A special feature of the TON DHT is that it tries to select nodes with
the smallest round-trip delays as the ``best'' nodes for the buckets
of the Kademlia routing table.

\nxsubpoint (Kademlia network queries.)  A Kademlia node usually
supports the following network queries:
\begin{itemize}
\item $\Ping$ -- Checks node availability.
\item $\Store(key,value)$ -- Asks the node to keep $value$ as a value
  for key $key$. For TON DHT, the $\Store$ queries are slightly more
  complicated (cf.~\ptref{sp:DHT.store}).
\item $\FindNode(key,l)$ -- Asks the node to return $l$
  Kademlia-nearest known nodes (from its Kademlia routing table) to
  $key$.
\item $\FindValue(key,l)$ -- The same as above, but if the node knows
  the value corresponding to key $key$, it just returns that value.
\end{itemize}

When any node wants to look up the value of a key $K$, it first
creates a set $S$ of $s'$ nodes (for some small value of $s'$, say,
$s'=5$), nearest to $K$ with respect to the Kademlia distance among
all known nodes (i.e., they are taken from the Kademlia routing
table). Then a $\FindValue$ query is sent to each of them, and nodes
mentioned in their answers are included in $S$. Then the $s'$ nodes
from $S$, nearest to $K$, are also sent a $\FindValue$ query if this
hasn't been done before, and the process continues until the value is
found or the set $S$ stops growing. This is a sort of ``beam search''
of the node nearest to $K$ with respect to Kademlia distance.

If the value of some key $K$ is to be set, the same procedure is run
for $s'\geq s$, with $\FindNode$ queries instead of $\FindValue$, to
find $s$ nearest nodes to $K$. Afterwards, $\Store$ queries are sent
to all of them.

There are some less important details in the implementation of a
Kademlia-like DHT (for example, any node should look up $s$ nearest
nodes to itself, say, once every hour, and re-publish all stored keys
to them by means of $\Store$ queries). We will ignore them for the
time being.

\nxsubpoint \embt(Booting a Kademlia node.)  When a Kademlia node goes
online, it first populates its Kademlia routing table by looking up
its own address. During this process, it identifies the $s$ nearest
nodes to itself. It can download from them all $(key,value)$ pairs
known to them to populate its part of the DHT.

\nxsubpoint\label{sp:DHT.store} \embt(Storing values in TON DHT.)
Storing values in TON DHT is slightly different from a general
Kademlia-like DHT. When someone wishes to store a value, she must
provide not only the key $K$ itself to the $\Store$ query, but also
its {\em preimage\/}---i.e., a TL-serialized string (with one of
several predefined TL-constructors at the beginning) containing a
``description'' of the key. This key description is later kept by the
node, along with the key and the value.

The key description describes the ``type'' of the object being stored,
its ``owner'', and its ``update rules'' in case of future updates. The
owner is usually identified by a public key included in the key
description. If it is included, normally only updates signed by the
corresponding private key will be accepted. The ``type'' of the stored
object is normally just a byte string. However, in some cases it can
be more sophisticated---for example, an input tunnel description
(cf.~\ptref{sp:tunnels}), or a collection of node addresses.

The ``update rules'' can also be different. In some cases, they simply
permit replacing the old value with the new value, provided the new
value is signed by the owner (the signature must be kept as part of
the value, to be checked later by any other nodes after they obtain
the value of this key). In other cases, the old value somehow affects
the new value. For example, it can contain a sequence number, and the
old value is overwritten only if the new sequence number is larger (to
prevent replay attacks).

\nxsubpoint\label{sp:distr.torr.tr} \embt(Distributed ``torrent
trackers'' and ``network interest groups'' in TON DHT.)  Yet another
interesting case is when the value contains a list of nodes---perhaps
with their IP addresses and ports, or just with their abstract
addresses---and the ``update rule'' consists in including the
requester in this list, provided she can confirm her identity.

This mechanism might be used to create a distributed ``torrent
tracker'', where all nodes interested in a certain ``torrent'' (i.e.,
a certain file) can find other nodes that are interested in the same
torrent, or already have a copy.

{\em TON Storage\/} (cf.~\ptref{sp:ex.ton.storage}) uses this
technology to find the nodes that have a copy of a required file
(e.g., a snapshot of the state of a shardchain, or an old
block). However, its more important use is to create ``overlay
multicast subnetworks'' and ``network interest groups''
(cf.~\ptref{sect:overlay}). The idea is that only some nodes are
interested in the updates of a specific shardchain. If the number of
shardchains becomes very large, finding even one node interested in
the same shard may become complicated. This ``distributed torrent
tracker'' provides a convenient way to find some of these
nodes. Another option would be to request them from a validator, but
this would not be a scalable approach, and validators might choose not
to respond to such queries coming from arbitrary unknown nodes.

\nxsubpoint \embt(Fall-back keys.)  Most of the ``key types''
described so far have an extra 32-bit integer field in their TL
description, normally equal to zero. However, if the key obtained by
hashing that description cannot be retrieved from or updated in the
TON DHT, the value in this field is increased, and a new attempt is
made. In this way, one cannot ``capture'' and ``censor'' a key (i.e.,
perform a key retention attack) by creating a lot of abstract
addresses lying near the key under attack and controlling the
corresponding DHT nodes.

\nxsubpoint\label{sp:loc.serv} \embt(Locating services.)  Some
services, located in the TON Network and available through the
(higher-level protocols built upon the) TON ADNL described
in~\ptref{sect:ANL}, may want to publish their abstract addresses
somewhere, so that their clients would know where to find them.

However, publishing the service's abstract address in the TON
Blockchain may not be the best approach, because the abstract address
might need to be changed quite often, and because it could make sense
to provide several addresses, for reliability or load balancing
purposes.

An alternative is to publish a public key into the TON Blockchain, and
use a special DHT key indicating that public key as its ``owner'' in
the TL description string (cf.~\ptref{sp:TL}) to publish an up-to-date
list of the service's abstract addresses. This is one of the
approaches exploited by TON Services.

\nxsubpoint \embt(Locating owners of TON blockchain accounts.)  In
most cases, owners of TON blockchain accounts would not like to be
associated with abstract network addresses, and especially IP
addresses, because this can violate their privacy. In some cases,
however, the owner of a TON blockchain account may want to publish
one or several abstract addresses where she could be contacted.

A typical case is that of a node in the TON Payments ``lightning
network'' (cf.~\ptref{sect:lightning}), the platform for instant
cryptocurrency transfers. A public TON Payments node may want not only
to establish payment channels with other peers, but also to publish an
abstract network address that could be used to contact it at a later
time for transferring payments along the already-established channels.

One option would be to include an abstract network address in the
smart contract creating the payment channel. A more flexible option is
to include a public key in the smart contract, and then use DHT as
explained in~\ptref{sp:loc.serv}.

The most natural way would be to use the same private key that
controls the account in the TON Blockchain to sign and publish updates
in the TON DHT about the abstract addresses associated with that
account. This is done almost in the same way as described
in~\ptref{sp:loc.serv}; however, the DHT key employed would require a
special key description, containing only the $\accountid$ itself,
equal to $\Sha$ of the ``account description'', which contains the
public key of the account. The signature, included in the value of
this DHT key, would contain the account description as well.

In this way, a mechanism for locating abstract network addresses of
some owners of the TON Blockchain accounts becomes available.

\nxsubpoint\label{sp:loc.abs.addr} \embt(Locating abstract addresses.)
Notice that the TON DHT, while being implemented over TON ADNL, is
itself used by the TON ADNL for several purposes.

The most important of them is to locate a node or its contact data
starting from its 256-bit abstract address. This is necessary because
the TON ADNL should be able to send datagrams to arbitrary 256-bit
abstract addresses, even if no additional information is provided.

To this end, the 256-bit abstract address is simply looked up as a key
in the DHT. Either a node with this address (i.e., using this address
as a public semi-persistent DHT address) is found, in which case its
IP address and port can be learned; or, an input tunnel description
may be retrieved as the value of the key in question, signed by the
correct private key, in which case this tunnel description would be
used to send ADNL datagrams to the intended recipient.

Notice that in order to make an abstract address ``public'' (reachable
from any nodes in the network), its owner must either use it as a
semi-permanent DHT address, or publish (in the DHT key equal to the
abstract address under consideration) an input tunnel description with
another of its public abstract addresses (e.g., the semi-permanent
address) as the tunnel's entry point. Another option would be to
simply publish its IP address and UDP port.

\mysubsection{Overlay Networks and Multicasting
  Messages}\label{sect:overlay}

In a multi-blockchain system like the TON Blockchain, even full nodes
would normally be interested in obtaining updates (i.e., new blocks)
only about some shardchains. To this end, a special overlay
(sub)network must be built inside the TON Network, on top of the ADNL
protocol discussed in~\ptref{sect:ANL}, one for each shardchain.

Therefore, the need to build arbitrary overlay subnetworks, open to
any nodes willing to participate, arises. Special gossip protocols,
built upon ADNL, will be run in these overlay networks. In particular,
these gossip protocols may be used to propagate (broadcast) arbitrary
data inside such a subnetwork.

\nxsubpoint \embt(Overlay networks.)  An {\em overlay (sub)network\/}
is simply a (virtual) network implemented inside some larger
network. Usually only some nodes of the larger network participate in
the overlay subnetwork, and only some ``links'' between these nodes,
physical or virtual, are part of the overlay subnetwork.

In this way, if the encompassing network is represented as a graph
(perhaps a full graph in the case of a datagram network such as ADNL,
where any node can easily communicate to any other), the overlay
subnetwork is a {\em subgraph\/} of this graph.

In most cases, the overlay network is implemented using some protocol
built upon the network protocol of the larger network. It may use the
same addresses as the larger network, or use custom addresses.

\nxsubpoint\label{sp:ton.overlays} \embt(Overlay networks in TON.)
Overlay networks in TON are built upon the ADNL protocol discussed
in~\ptref{sect:ANL}; they use 256-bit ADNL abstract addresses as
addresses in the overlay networks as well. Each node usually selects
one of its abstract addresses to double as its address in the overlay
network.

In contrast to ADNL, the TON overlay networks usually do not support
sending datagrams to arbitrary other nodes. Instead, some
``semipermanent links'' are established between some nodes (called
``neighbors'' with respect to the overlay network under
consideration), and messages are usually forwarded along these links
(i.e., from a node to one of its neighbors). In this way, a TON
overlay network is a (usually not full) subgraph inside the (full)
graph of the ADNL network.

Links to neighbors in TON overlay networks can be implemented using
dedicated peer-to-peer ADNL channels (cf.~\ptref{sp:net.channels}).

Each node of an overlay network maintains a list of neighbors (with
respect to the overlay network), containing their abstract addresses
(which they use to identify them in the overlay network) and some link
data (e.g., the ADNL channel used to communicate with them).

\nxsubpoint \embt(Private and public overlay networks.)  Some overlay
networks are {\em public}, meaning that any node can join them at
will. Other are {\em private}, meaning that only certain nodes can be
admitted (e.g., those that can prove their identities as validators.)
Some private overlay networks can even be unknown to the ``general
public''. The information about such overlay networks is made
available only to certain trusted nodes; for example, it can be
encrypted with a public key, and only nodes having a copy of the
corresponding private key will be able to decrypt this information.

\nxsubpoint \embt(Centrally controlled overlay networks.)  Some
overlay networks are {\em centrally controlled}, by one or several
nodes, or by the owner of some widely-known public key. Others are
{\em decentralized}, meaning that there are no specific nodes
responsible for them.

\nxsubpoint \embt(Joining an overlay network.)  When a node wants to
join an overlay network, it first must learn its 256-bit {\em network
  identifier}, usually equal to $\Sha$ of the {\em description\/} of
the overlay network---a TL-serialized object (cf.~\ptref{sp:TL}) which
may contain, for instance, the central authority of the overlay
network (i.e., its public key and perhaps its abstract
address,\footnote{Alternatively, the abstract address might be stored
  in the DHT as explained in~\ptref{sp:loc.serv}.}) a string with the
name of the overlay network, a TON Blockchain shard identifier if this
is an overlay network related to that shard, and so on.

Sometimes it is possible to recover the overlay network description
starting from the network identifier, simply by looking it up in the
TON DHT. In other cases (e.g., for private overlay networks), one must
obtain the network description along with the network identifier.

\nxsubpoint\label{sp:loc.1.mem} \embt(Locating one member of the
overlay network.)  After a node learns the network identifier and the
network description of the overlay network it wants to join, it must
locate at least one node belonging to that network.

This is also needed for nodes that do not want to join the overlay
network, but want just to communicate with it; for example, there
might be an overlay network dedicated to collecting and propagating
transaction candidates for a specific shardchain, and a client might
want to connect to any node of this network to suggest a transaction.

The method used for locating members of an overlay network is defined
in the description of that network. Sometimes (especially for private
networks) one must already know a member node to be able to join. In
other cases, the abstract addresses of some nodes are contained in the
network description. A more flexible approach is to indicate in the
network description only the central authority responsible for the
network, and then the abstract addresses will be available through
values of certain DHT keys, signed by that central authority.

Finally, truly decentralized public overlay networks can use the
``distributed torrent-tracker'' mechanism described
in~\ptref{sp:distr.torr.tr}, also implemented with the aid of the TON
DHT.

\nxsubpoint\label{sp:loc.many.mem} \embt(Locating more members of the
overlay network. Creating links.)  Once one node of the overlay
network is found, a special query may be sent to that node requesting
a list of other members, for instance, neighbors of the node being
queried, or a random selection thereof.

This enables the joining member to populate her ``adjacency'' or
``neighbor list'' with respect to the overlay network, by selecting
some newly-learned network nodes and establishing links to them (i.e.,
dedicated ADNL point-to-point channels, as outlined
in~\ptref{sp:ton.overlays}). After that, special messages are sent to
all neighbors indicating that the new member is ready to work in the
overlay network. The neighbors include their links to the new member
in their neighbor lists.

\nxsubpoint\label{sp:rand.mem} \embt(Maintaining the neighbor list.)
An overlay network node must update its neighbor list from time to
time. Some neighbors, or at least links (channels) to them, may stop
responding; in this case, these links must be marked as ``suspended'',
some attempts to reconnect to such neighbors must be made, and, if
these attempts fail, the links must be destroyed.

On the other hand, every node sometimes requests from a randomly
chosen neighbor its list of neighbors (or some random selection
thereof), and uses it to partially update its own neighbor list, by
adding some newly-discovered nodes to it, and removing some of the old
ones, either randomly or depending on their response times and
datagram loss statistics.

\nxsubpoint \embt(The overlay network is a random subgraph.)  In this
way, the overlay network becomes a random subgraph inside the ADNL
network. If the degree of each vertex is at least three (i.e., if each
node is connected to at least three neighbors), this random graph is
known to be {\em connected\/} with a probability almost equal to
one. More precisely, the probability of a random graph with $n$
vertices being {\em dis\/}connected is exponentially small, and this
probability can be completely neglected if, say, $n\geq20$. (Of
course, this does not apply in the case of a global network partition,
when nodes on different sides of the partition have no chance to learn
about each other.) On the other hand, if $n$ is smaller than 20, it
would suffice to require each vertex to have, say, at least ten
neighbors.

\nxsubpoint\label{sp:ov.opt.low.lat} \embt(TON overlay networks are
optimized for lower latency.)  TON overlay networks optimize the
``random'' network graph generated by the previous method as
follows. Every node tries to retain at least three neighbors with the
minimal round-trip time, changing this list of ``fast neighbors'' very
rarely. At the same time, it also has at least three other ``slow
neighbors'' that are chosen completely randomly, so that the overlay
network graph would always contain a random subgraph. This is required
to maintain connectivity and prevent splitting of the overlay network
into several unconnected regional subnetworks. At least three
``intermediate neighbors'', which have intermediate round-trip times,
bounded by a certain constant (actually, a function of the round-trip
times of the fast and the slow neighbors), are also chosen and
retained.

In this way, the graph of an overlay network still maintains enough
randomness to be connected, but is optimized for lower latency and
higher throughput.

\nxsubpoint \embt(Gossip protocols in an overlay network.)  An overlay
network is often used to run one of the so-called {\em gossip
  protocols}, which achieve some global goal while letting every node
interact only with its neighbors. For example, there are gossip
protocols to construct an approximate list of all members of a (not
too large) overlay network, or to compute an estimate of the number of
members of an (arbitrarily large) overlay network, using only a
bounded amount of memory at each node (cf.~\cite[4.4.3]{DistrSys} or
\cite{Birman} for details).

\nxsubpoint \embt(Overlay network as a broadcast domain.)  The most
important gossip protocol running in an overlay network is the {\em
  broadcast protocol}, intended to propagate broadcast messages
generated by any node of the network, or perhaps by one of the
designated sender nodes, to all other nodes.

There are in fact several broadcast protocols, optimized for different
use cases. The simplest of them receives new broadcast messages and
relays them to all neighbors that have not yet sent a copy of that
message themselves.

\nxsubpoint \embt(More sophisticated broadcast protocols.)  Some
applications may warrant more sophisticated broadcast protocols. For
instance, for broadcasting messages of substantial size, it makes
sense to send to the neighbors not the newly-received message itself,
but its hash (or a collection of hashes of new messages). The neighbor
may request the message itself after learning a previously unseen
message hash, to be transferred, say, using the reliable large
datagram protocol (RLDP) discussed in~\ptref{sp:RLDP}. In this way,
the new message will be downloaded from one neighbor only.

\nxsubpoint \embt(Checking the connectivity of an overlay network.)
The connectivity of an overlay network can be checked if there is a
known node (e.g., the ``owner'' or the ``creator'' of the overlay
network) that must be in this overlay network. Then the node in
question simply broadcasts from time to time short messages containing
the current time, a sequence number and its signature. Any other node
can be sure that it is still connected to the overlay network if it
has received such a broadcast not too long ago. This protocol can be
extended to the case of several well-known nodes; for example, they
all will send such broadcasts, and all other nodes will expect to
receive broadcasts from more than half of the well-known nodes.

In the case of an overlay network used for propagating new blocks (or
just new block headers) of a specific shardchain, a good way for a
node to check connectivity is to keep track of the most recent block
received so far. Because a block is normally generated every five
seconds, if no new block is received for more than, say, thirty
seconds, the node probably has been disconnected from the overlay
network.

\nxsubpoint\label{sp:streaming.multicast} \embt(Streaming broadcast
protocol.)  Finally, there is a {\em streaming broadcast protocol\/}
for TON overlay networks, used, for example, to propagate block
candidates among validators of some shardchain (``shardchain task
group''), who, of course, create a private overlay network for that
purpose. The same protocol can be used to propagate new shardchain
blocks to all full nodes for that shardchain.

This protocol has already been outlined
in~\ptref{sp:sh.blk.cand.prop}: the new (large) broadcast message is
split into, say, $N$ one-kilobyte chunks; the sequence of these chunks
is augmented to $M\geq N$ chunks by means of an erasure code such as
the Reed--Solomon or a fountain code (e.g., the RaptorQ code
\cite{RaptorQ} \cite{Raptor}), and these $M$ chunks are streamed to
all neighbors in ascending chunk number order. The participating nodes
collect these chunks until they can recover the original large message
(one would have to successfully receive at least $N$ of the chunks for
this), and then instruct their neighbors to stop sending new chunks of
the stream, because now these nodes can generate the subsequent chunks
on their own, having a copy of the original message. Such nodes
continue to generate the subsequent chunks of the stream and send them
to their neighbors, unless the neighbors in turn indicate that this is
no longer necessary.

In this way, a node does not need to download a large message in its
entirety before propagating it further. This minimizes broadcast
latency, especially when combined with the optimizations described
in~\ptref{sp:ov.opt.low.lat}.

\nxsubpoint \embt(Constructing new overlay networks based on existing
ones.)  Sometimes one does not want to construct an overlay network
from scratch. Instead, one or several previously existing overlay
networks are known, and the membership of the new overlay network is
expected to overlap significantly with the combined membership of
these overlay networks.

An important example arises when a TON shardchain is split in two, or
two sibling shardchains are merged into one
(cf.~\ptref{sect:split.merge}). In the first case, the overlay
networks used for propagating new blocks to full nodes must be
constructed for each of the new shardchains; however, each of these
new overlay networks can be expected to be contained in the block
propagation network of the original shardchain (and comprise
approximately half its members). In the second case, the overlay
network for propagating new blocks of the merged shardchain will
consist approximately of the union of members of the two overlay
networks related to the two sibling shardchains being merged.

In such cases, the description of the new overlay network may contain
an explicit or implicit reference to a list of related existing
overlay networks. A node wishing to join the new overlay network may
check whether it is already a member of one of these existing
networks, and query its neighbors in these networks whether they are
interested in the new network as well. In case of a positive answer,
new point-to-point channels can be established to such neighbors, and
they can be included in the neighbor list for the new overlay network.

This mechanism does not totally supplant the general mechanism
described in~\ptref{sp:loc.1.mem} and \ptref{sp:loc.many.mem}; rather,
both are run in parallel and are used to populate the neighbor
list. This is needed to prevent inadvertent splitting of the new
overlay network into several unconnected subnetworks.

\nxsubpoint\label{sp:net.within.net} \embt(Overlay networks within
overlay networks.)  Another interesting case arises in the
implementation of {\em TON Payments} (a ``lightning network'' for
instant off-chain value transfers; cf.~\ptref{sect:lightning}). In
this case, first an overlay network containing all transit nodes of
the ``lightning network'' is constructed. However, some of these nodes
have established payment channels in the blockchain; they must always
be neighbors in this overlay network, in addition to any ``random''
neighbors selected by the general overlay network algorithms described
in~\ptref{sp:loc.1.mem}, \ptref{sp:loc.many.mem}
and~\ptref{sp:rand.mem}. These ``permanent links'' to the neighbors
with established payment channels are used to run specific lightning
network protocols, thus effectively creating an overlay subnetwork
(not necessarily connected, if things go awry) inside the encompassing
(almost always connected) overlay network.

%%%%%%%%%%%%%%%%%%%%%%%%%%%%%%%%%%%%%%%%%%%%%%%%%
%
%
%                  SERVICES
%
%
%%%%%%%%%%%%%%%%%%%%%%%%%%%%%%%%%%%%%%%%%%%%%%%%%
\clearpage
\mysection{TON Services and Applications}\label{sect:services}

We have discussed the TON Blockchain and TON Networking technologies
at some length. Now we explain some ways in which they can be combined
to create a wide range of services and applications, and discuss some
of the services that will be provided by the TON Project itself,
either from the very beginning or at a later time.

\mysubsection{TON Service Implementation Strategies}%
\label{sect:ton.service.impl}

We start with a discussion of how different blockchain and
network-related applications and services may be implemented inside
the TON ecosystem. First of all, a simple classification is in order:

\nxsubpoint \embt(Applications and services.)  We will use the words
``application'' and ``service'' interchangeably. However, there is a
subtle and somewhat vague distinction: an {\em application\/} usually
provides some services directly to human users, while a {\em
  service\/} is usually exploited by other applications and
services. For example, TON Storage is a service, because it is
designed to keep files on behalf of other applications and services,
even though a human user might use it directly as well. A hypothetical
``Facebook in a blockchain'' (cf.~\ptref{sp:blockchain.facebook}) or
Telegram messenger, if made available through the TON Network (i.e.,
implemented as a ``ton-service''; cf.~\ptref{sp:telegram.ton.serv}),
would rather be an {\em application}, even though some ``bots'' might
access it automatically without human intervention.

\nxsubpoint\label{sp:on.off.chain} \embt(Location of the application:
on-chain, off-chain or mixed.)  A service or an application designed
for the TON ecosystem needs to keep its data and process that data
somewhere. This leads to the following classification of applications
(and services):
\begin{itemize}
\item {\em On-chain\/} applications (cf.~\ptref{sp:pure.blockchain}):
  All data and processing are in the TON Blockchain.
\item {\em Off-chain\/} applications (cf.~\ptref{sp:pure.net.serv}):
  All data and processing are outside the TON Blockchain, on servers
  available through the TON Network.
\item {\em Mixed\/} applications (cf.~\ptref{sp:mixed.serv}): Some,
  but not all, data and processing are in the TON Blockchain; the rest
  are on off-chain servers available through the TON Network.
\end{itemize}

\nxsubpoint \embt(Centralization: centralized and decentralized, or
distributed, applications.)  Another classification criterion is
whether the application (or service) relies on a centralized server
cluster, or is really ``distributed'' (cf.~\ptref{sp:fog}). All
on-chain applications are automatically decentralized and
distributed. Off-chain and mixed applications may exhibit different
degrees of centralization.

\medbreak
Now let us consider the above possibilities in more detail.

\nxsubpoint\label{sp:pure.blockchain} \embt(Pure ``on-chain''
applications: distributed applications, or ``dapps'', residing in the
blockchain.)  One of the possible approaches, mentioned
in~\ptref{sp:on.off.chain}, is to deploy a ``distributed application''
(commonly abbreviated as ``dapp'') completely in the TON Blockchain,
as one smart contract or a collection of smart contracts. All data
will be kept as part of the permanent state of these smart contracts,
and all interaction with the project will be done by means of (TON
Blockchain) messages sent to or received from these smart contracts.

We have already discussed in~\ptref{sp:blockchain.facebook} that this
approach has its drawbacks and limitations. It has its advantages,
too: such a distributed application needs no servers to run on or to
store its data (it runs ``in the blockchain''---i.e., on the
validators' hardware), and enjoys the blockchain's extremely high
(Byzantine) reliability and accessibility. The developer of such a
distributed application does not need to buy or rent any hardware; all
she needs to do is develop some software (i.e., the code for the smart
contracts). After that, she will effectively rent the computing power
from the validators, and will pay for it in Grams, either herself or
by putting this burden on the shoulders of her users.

\nxsubpoint\label{sp:pure.net.serv} \embt(Pure network services:
``ton-sites'' and ``ton-services''.)  Another extreme option is to
deploy the service on some servers and make it available to the users
through the ADNL protocol described in~\ptref{sect:ANL}, and maybe
some higher level protocol such as the RLDP discussed
in~\ptref{sp:RLDP}, which can be used to send RPC queries to the
service in any custom format and obtain answers to these queries. In
this way, the service will be totally off-chain, and will reside in
the TON Network, almost without using the TON Blockchain.

The TON Blockchain might be used only to locate the abstract address
or addresses of the service, as outlined in~\ptref{sp:loc.serv},
perhaps with the aid of a service such as the TON DNS
(cf.~\ptref{sp:ton.dns}) to facilitate translation of domain-like
human-readable strings into abstract addresses.

To the extent the ADNL network (i.e., the TON Network) is similar to
the Invisible Internet Project ($I^2P$), such (almost) purely network
services are analogous to the so-called ``eep-services'' (i.e.,
services that have an $I^2P$-address as their entry point, and are
available to clients through the $I^2P$ network). We will say that
such purely network services residing in the TON Network are
``ton-services''.

An ``eep-service'' may implement HTTP as its client-server protocol;
in the TON Network context, a ``ton-service'' might simply use RLDP
(cf.~\ptref{sp:RLDP}) datagrams to transfer HTTP queries and responses
to them. If it uses the TON DNS to allow its abstract address to be
looked up by a human-readable domain name, the analogy to a web site
becomes almost perfect. One might even write a specialized browser, or
a special proxy (``ton-proxy'') that is run locally on a user's
machine, accepts arbitrary HTTP queries from an ordinary web browser
the user employs (once the local IP address and the TCP port of the
proxy are entered into the browser's configuration), and forwards
these queries through the TON Network to the abstract address of the
service. Then the user would have a browsing experience similar to
that of the World Wide Web (WWW).

In the $I^2P$ ecosystem, such ``eep-services'' are called
``eep-sites''. One can easily create ``ton-sites'' in the TON
ecosystem as well. This is facilitated somewhat by the existence of
services such as the TON DNS, which exploit the TON Blockchain and the
TON DHT to translate (TON) domain names into abstract addresses.

\nxsubpoint\label{sp:telegram.ton.serv} \embt(Telegram Messenger as a
ton-service; MTProto over RLDP.)  We would like to mention in passing
that the MTProto
protocol,\footnote{\url{https://core.telegram.org/mtproto}} used by
Telegram Messenger\footnote{\url{https://telegram.org/}} for
client-server interaction, can be easily embedded into the RLDP
protocol discussed in~\ptref{sp:RLDP}, thus effectively transforming
Telegram into a ton-service. Because the TON Proxy technology can be
switched on transparently for the end user of a ton-site or a
ton-service, being implemented on a lower level than the RLDP and ADNL
protocols (cf.~\ptref{sp:tunnels}), this would make Telegram
effectively unblockable. Of course, other messaging and social
networking services might benefit from this technology as well.

\nxsubpoint\label{sp:mixed.serv} \embt(Mixed services: partly
off-chain, partly on-chain.)  Some services might use a mixed
approach: do most of the processing off-chain, but also have some
on-chain part (for example, to register their obligations towards
their users, and vice versa). In this way, part of the state would
still be kept in the TON Blockchain (i.e., an immutable public
ledger), and any misbehavior of the service or of its users could be
punished by smart contracts.

\nxsubpoint\label{sp:ex.ton.storage} \embt(Example: keeping files
off-chain; TON Storage.)  An example of such a service is given by
{\em TON Storage}. In its simplest form, it allows users to store
files off-chain, by keeping on-chain only a hash of the file to be
stored, and possibly a smart contract where some other parties agree
to keep the file in question for a given period of time for a
pre-negotiated fee. In fact, the file may be subdivided into chunks of
some small size (e.g., 1 kilobyte), augmented by an erasure code such
as a Reed--Solomon or a fountain code, a Merkle tree hash may be
constructed for the augmented sequence of chunks, and this Merkle tree
hash might be published in the smart contract instead of or along with
the usual hash of the file. This is somewhat reminiscent of the way
files are stored in a torrent.

An even simpler form of storing files is completely off-chain: one
might essentially create a ``torrent'' for a new file, and use TON DHT
as a ``distributed torrent tracker'' for this torrent
(cf.~\ptref{sp:distr.torr.tr}). This might actually work pretty well
for popular files. However, one does not get any availability
guarantees. For example, a hypothetical ``blockchain Facebook''
(cf.~\ptref{sp:blockchain.facebook}), which would opt to keep the
profile photographs of its users completely off-chain in such
``torrents'', might risk losing photographs of ordinary (not
especially popular) users, or at least risk being unable to present
these photographs for prolonged periods. The TON Storage technology,
which is mostly off-chain, but uses an on-chain smart contract to
enforce availability of the stored files, might be a better match for
this task.

\nxsubpoint\label{sp:fog} \embt(Decentralized mixed services, or ``fog
services''.)  So far, we have discussed {\em centralized\/} mixed
services and applications. While their on-chain component is processed
in a decentralized and distributed fashion, being located in the
blockchain, their off-chain component relies on some servers
controlled by the service provider in the usual centralized
fashion. Instead of using some dedicated servers, computing power
might be rented from a cloud computing service offered by one of the
large companies. However, this would not lead to decentralization of
the off-chain component of the service.

A decentralized approach to implementing the off-chain component of a
service consists in creating a {\em market}, where anybody possessing
the required hardware and willing to rent their computing power or
disk space would offer their services to those needing them.

For example, there might exist a registry (which might also be called
a ``market'' or an ``exchange'') where all nodes interested in keeping
files of other users publish their contact information, along with
their available storage capacity, availability policy, and
prices. Those needing these services might look them up there, and, if
the other party agrees, create smart contracts in the blockchain and
upload files for off-chain storage. In this way a service like {\em
  TON Storage\/} becomes truly decentralized, because it does not need
to rely on any centralized cluster of servers for storing files.

\nxsubpoint \embt(Example: ``fog computing'' platforms as
decentralized mixed services.)  Another example of such a
decentralized mixed application arises when one wants to perform some
specific computations (e.g., 3D rendering or training neural
networks), often requiring specific and expensive hardware. Then those
having such equipment might offer their services through a similar
``exchange'', and those needing such services would rent them, with
the obligations of the sides registered by means of smart
contracts. This is similar to what ``fog computing'' platforms, such
as Golem (\url{https://golem.network/}) or SONM
(\url{https://sonm.io/}), promise to deliver.

\nxsubpoint\label{sp:ex.ton.proxy} \embt(Example: TON Proxy is a fog
service.)  {\em TON Proxy\/} provides yet another example of a fog
service, where nodes wishing to offer their services (with or without
compensation) as tunnels for ADNL network traffic might register,
and those needing them might choose one of these nodes depending on
the price, latency and bandwidth offered. Afterwards, one might use
payment channels provided by {\em TON Payments\/} for processing
micropayments for the services of those proxies, with payments
collected, for instance, for every 128~KiB transferred.

\nxsubpoint \embt(Example: TON Payments is a fog service.)  The TON
Payments platform (cf.~\ptref{sect:payments}) is also an example of
such a decentralized mixed application.

\mysubsection{Connecting Users and Service
  Providers}\label{sect:reg.markt}

We have seen in~\ptref{sp:fog} that ``fog services'' (i.e., mixed
decentralized services) will usually need some {\em markets}, {\em
  exchanges\/} or {\em registries}, where those needing specific
services might meet those providing them.

Such markets are likely to be implemented as on-chain, off-chain or
mixed services themselves, centralized or distributed.

\nxsubpoint \embt(Example: connecting to TON Payments.)  For example,
if one wants to use TON Payments (cf.~\ptref{sect:payments}), the
first step would be to find at least some existing transit nodes of
the ``lightning network'' (cf.~\ptref{sect:lightning}), and establish
payment channels with them, if they are willing. Some nodes can be
found with the aid of the ``encompassing'' overlay network, which is
supposed to contain all transit lightning network nodes
(cf.~\ptref{sp:net.within.net}). However, it is not clear whether
these nodes will be willing to create new payment channels. Therefore,
a registry is needed where nodes ready to create new links can publish
their contact information (e.g., their abstract addresses).

\nxsubpoint \embt(Example: uploading a file into TON Storage.)
Similarly, if one wants to upload a file into the TON Storage, she
must locate some nodes willing to sign a smart contract binding them
to keep a copy of that file (or of any file below a certain size
limit, for that matter). Therefore, a registry of nodes offering their
services for storing files is needed.

\nxsubpoint \embt(On-chain, mixed and off-chain registries.)  Such a
registry of service providers might be implemented completely
on-chain, with the aid of a smart contract which would keep the
registry in its permanent storage. However, this would be quite slow
and expensive. A mixed approach is more efficient, where the
relatively small and rarely changed on-chain registry is used only to
point out some nodes (by their abstract addresses, or by their public
keys, which can be used to locate actual abstract addresses as
described in~\ptref{sp:loc.serv}), which provide off-chain
(centralized) registry services.

Finally, a decentralized, purely off-chain approach might consist of a
public overlay network (cf.~\ptref{sect:overlay}), where those willing
to offer their services, or those looking to buy somebody's services,
simply broadcast their offers, signed by their private keys. If the
service to be provided is very simple, even broadcasting the offers
might be not necessary: the approximate membership of the overlay
network itself might be used as a ``registry'' of those willing to
provide a particular service. Then a client requiring this service
might locate (cf.~\ptref{sp:loc.many.mem}) and query some nodes of
this overlay network, and then query their neighbors, if the nodes
already known are not ready to satisfy its needs.

\nxsubpoint\label{sp:side.chain.reg} \embt(Registry or exchange in a
side-chain.)  Another approach to implementing decentralized mixed
registries consists in creating an independent specialized blockchain
(``side-chain''), maintained by its own set of self-proclaimed
validators, who publish their identities in an on-chain smart contract
and provide network access to all interested parties to this
specialized blockchain, collecting transaction candidates and
broadcasting block updates through dedicated overlay networks
(cf.~\ptref{sect:overlay}). Then any full node for this sidechain can
maintain its own copy of the shared registry (essentially equal to the
global state of this side-chain), and process arbitrary queries
related to this registry.

\nxsubpoint \embt(Registry or exchange in a workchain.)  Another
option is to create a dedicated workchain inside the TON Blockchain,
specialized for creating registries, markets and exchanges. This might
be more efficient and less expensive than using smart contracts
residing in the basic workchain
(cf.~\ptref{sp:basic.workchain}). However, this would still be more
expensive than maintaining registries in side-chains
(cf.~\ptref{sp:side.chain.reg}).

\mysubsection{Accessing TON Services}

We have discussed in~\ptref{sect:ton.service.impl} the different
approaches one might employ for creating new services and applications
residing in the TON ecosystem. Now we discuss how these services might
be accessed, and some of the ``helper services'' that will be provided
by TON, including {\em TON DNS\/} and {\em TON Storage}.

\nxsubpoint\label{sp:ton.dns} \embt(TON DNS: a mostly on-chain
hierarchical domain name service.)  The {\em TON DNS\/} is a
predefined service, which uses a collection of smart contracts to keep
a map from human-readable domain names to (256-bit) addresses of ADNL
network nodes and TON Blockchain accounts and smart contracts.

While anybody might in principle implement such a service using the
TON Blockchain, it is useful to have such a predefined service with a
well-known interface, to be used by default whenever an application or
a service wants to translate human-readable identifiers into
addresses.

\nxsubpoint \embt(TON DNS use cases.)  For example, a user looking to
transfer some cryptocurrency to another user or to a merchant may
prefer to remember a TON DNS domain name of the account of that user
or merchant, instead of keeping their 256-bit account identifiers at
hand and copy-pasting them into the recipient field in their light
wallet client.

Similarly, TON DNS may be used to locate account identifiers of smart
contracts or entry points of ton-services and ton-sites
(cf.~\ptref{sp:pure.net.serv}), enabling a specialized client
(``ton-browser''), or a usual internet browser combined with a
specialized ton-proxy extension or stand-alone application, to deliver
a WWW-like browsing experience to the user.

\nxsubpoint \embt(TON DNS smart contracts.)  The TON DNS is
implemented by means of a tree of special (DNS) smart contracts. Each
DNS smart contract is responsible for registering subdomains of some
fixed domain. The ``root'' DNS smart contract, where level one domains
of the TON DNS system will be kept, is located in the masterchain. Its
account identifier must be hardcoded into all software that wishes to
access the TON DNS database directly.

Any DNS smart contract contains a hashmap, mapping variable-length
null-terminated UTF-8 strings into their ``values''. This hashmap is
implemented as a binary Patricia tree, similar to that described
in~\ptref{sp:patricia} but supporting variable-length bitstrings as
keys.

\nxsubpoint \embt(Values of the DNS hashmap, or TON DNS records.)  As
to the values, they are ``TON DNS records'' described by a TL-scheme
(cf.~\ptref{sp:TL}). They consist of a ``magic number'', selecting one
of the options supported, and then either an account identifier, or a
smart-contract identifier, or an abstract network address
(cf.~\ptref{sect:ANL}), or a public key used to locate abstract
addresses of a service (cf.~\ptref{sp:loc.serv}), or a description of
an overlay network, and so on. An important case is that of another
DNS smart contract: in such a case, that smart contract is used to
resolve subdomains of its domain. In this way, one can create separate
registries for different domains, controlled by the owners of those
domains.

These records may also contain an expiration time, a caching time
(usually very large, because updating values in a blockchain too often
is expensive), and in most cases a reference to the owner of the
subdomain in question. The owner has the right to change this record
(in particular, the owner field, thus transferring the domain to
somebody else's control), and to prolong it.

\nxsubpoint \embt(Registering new subdomains of existing domains.)  In
order to register a new subdomain of an existing domain, one simply
sends a message to the smart contract, which is the registrar of that
domain, containing the subdomain (i.e., the key) to be registered, the
value in one of several predefined formats, an identity of the owner,
an expiration date, and some amount of cryptocurrency as determined by
the domain's owner.

Subdomains are registered on a ``first-come, first-served'' basis.

\nxsubpoint\label{sp:dns.get} \embt(Retrieving data from a DNS smart
contract.)  In principle, any full node for the masterchain or
shardchain containing a DNS smart contract might be able to look up
any subdomain in the database of that smart contract, if the structure
and the location of the hashmap inside the persistent storage of the
smart contract are known.

However, this approach would work only for certain DNS smart
contracts. It would fail miserably if a non-standard DNS smart
contract were used.

Instead, an approach based on {\em general smart contract
  interfaces\/} and {\em get methods\/} (cf.~\ptref{sp:get.methods})
is used. Any DNS smart contract must define a ``get method'' with a
``known signature'', which is invoked to look up a key. Since this
approach makes sense for other smart contracts as well, especially
those providing on-chain and mixed services, we explain it in some
detail in~\ptref{sp:get.methods}.

\nxsubpoint \embt(Translating a TON DNS domain.)  Once any full node,
acting by itself or on behalf of some light client, can look up
entries in the database of any DNS smart contract, arbitrary TON DNS
domain names can be recursively translated, starting from the
well-known and fixed root DNS smart contract (account) identifier.

For example, if one wants to translate \texttt{A.B.C}, one looks up
keys \texttt{.C}, \texttt{.B.C}, and \texttt{A.B.C} in the root domain
database. If the first of them is not found, but the second is, and
its value is a reference to another DNS smart contract, then
\texttt{A} is looked up in the database of that smart contract and the
final value is retrieved.

\nxsubpoint \embt(Translating TON DNS domains for light nodes.)  In
this way, a full node for the masterchain---and also for all
shardchains involved in the domain look-up process---might translate
any domain name into its current value without external help. A light
node might request a full node to do this on its behalf and return the
value, along with a Merkle proof
(cf.~\ptref{sp:merkle.query.resp}). This Merkle proof would enable the
light node to verify that the answer is correct, so such TON DNS
responses cannot be ``spoofed'' by a malicious interceptor, in
contrast to the usual DNS protocol.

Because no node can be expected to be a full node with respect to all
shardchains, actual TON DNS domain translation would involve a
combination of these two strategies.

\nxsubpoint \embt(Dedicated ``TON DNS servers''.)  One might provide a
simple ``TON DNS server'', which would receive RPC ``DNS'' queries
(e.g., via the ADNL or RLDP protocols described in~\ptref{sect:ANL}),
requesting that the server translate a given domain, process these
queries by forwarding some subqueries to other (full) nodes if
necessary, and return answers to the original queries, augmented by
Merkle proofs if required.

Such ``DNS servers'' might offer their services (for free or not) to
any other nodes and especially light clients, using one of the methods
described in~\ptref{sect:reg.markt}. Notice that these servers, if
considered part of the TON DNS service, would effectively transform it
from a distributed on-chain service into a distributed mixed service
(i.e., a ``fog service'').

This concludes our brief overview of the TON DNS service, a scalable
on-chain registry for human-readable domain names of TON Blockchain
and TON Network entities.

\nxsubpoint \embt(Accessing data kept in smart contracts.)  We have
already seen that it is sometimes necessary to access data stored in a
smart contract without changing its state.

If one knows the details of the smart-contract implementation, one can
extract all the needed information from the smart contract's
persistent storage, available to all full nodes of the shardchain the
smart contract resides in. However, this is quite an inelegant way of
doing things, depending very much on the smart-contract
implementation.

\nxsubpoint\label{sp:get.methods} \embt(``Get methods'' of smart
contracts.)  A better way would be to define some {\em get methods\/}
in the smart contract, that is, some types of inbound messages that do
not affect the state of the smart contract when delivered, but
generate one or more output messages containing the ``result'' of the
get method. In this way, one can obtain data from a smart contract,
knowing only that it implements a get method with a known signature
(i.e., a known format of the inbound message to be sent and outbound
messages to be received as a result).

This way is much more elegant and in line with object oriented
programming (OOP). However, it has an obvious defect so far: one must
actually commit a transaction into the blockchain (sending the get
message to the smart contract), wait until it is committed and
processed by the validators, extract the answer from a new block, and
pay for gas (i.e., for executing the get method on the validators'
hardware). This is a waste of resources: get methods do not change the
state of the smart contract anyways, so they need not be executed in
the blockchain.

\nxsubpoint\label{sp:tent.exec.get} \embt(Tentative execution of get
methods of smart contracts.)  We have already remarked
(cf.~\ptref{sp:ext.msg}) that any full node can tentatively execute
any method of any smart contract (i.e., deliver any message to a smart
contract), starting from a given state of the smart contract, without
actually committing the corresponding transaction. The full node can
simply load the code of the smart contract under consideration into
the TON VM, initialize its persistent storage from the global state of
the shardchain (known to all full nodes of the shardchain), and
execute the smart-contract code with the inbound message as its input
parameter. The output messages created will yield the result of this
computation.

In this way, any full node can evaluate arbitrary get methods of
arbitrary smart contracts, provided their signature (i.e., the format
of inbound and outbound messages) is known. The node may keep track of
the cells of the shardchain state accessed during this evaluation, and
create a Merkle proof of the validity of the computation performed,
for the benefit of a light node that might have asked the full node to
do so (cf.~\ptref{sp:merkle.query.resp}).

\nxsubpoint \embt(Smart-contract interfaces in TL-schemes.)  Recall
that the methods implemented by a smart contract (i.e., the input
messages accepted by it) are essentially some TL-serialized objects,
which can be described by a TL-scheme (cf.~\ptref{sp:TL}). The
resulting output messages can be described by the same TL-scheme as
well. In this way, the interface provided by a smart contract to other
accounts and smart contracts may be formalized by means of a
TL-scheme.

In particular, (a subset of) get methods supported by a smart
contract can be described by such a formalized smart-contract
interface.

\nxsubpoint\label{sp:pub.int.smartc} \embt(Public interfaces of a
smart contract.)  Notice that a formalized smart-contract interface,
either in form of a TL-scheme (represented as a TL source file;
cf.~\ptref{sp:TL}) or in serialized form,\footnote{TL-schemes can be
  TL-serialized themselves;
  cf.\ \url{https://core.telegram.org/mtproto/TL-tl}.} can be
published---for example, in a special field in the smart-contract
account description, stored in the blockchain, or separately, if this
interface will be referred to many times. In the latter case a hash of
the supported public interface may be incorporated into the
smart-contract description instead of the interface description
itself.

An example of such a public interface is that of a DNS smart contract,
which is supposed to implement at least one standard get method for
looking up subdomains (cf.~\ptref{sp:dns.get}). A standard method for
registering new subdomains can be also included in the standard public
interface of DNS smart contracts.

\nxsubpoint\label{sp:ui.smartc} \embt(User interface of a smart
contract.)  The existence of a public interface for a smart contract
has other benefits, too. For example, a wallet client application may
download such an interface while examining a smart contract on the
request of a user, and display a list of public methods (i.e., of
available actions) supported by the smart contract, perhaps with some
human-readable comments if any are provided in the formal
interface. After the user selects one of these methods, a form may be
automatically generated according to the TL-scheme, where the user
will be prompted for all fields required by the chosen method and for
the desired amount of cryptocurrency (e.g., Grams) to be attached to
this request. Submitting this form will create a new blockchain
transaction containing the message just composed, sent from the user's
blockchain account.

In this way, the user will be able to interact with arbitrary smart
contracts from the wallet client application in a user-friendly way by
filling and submitting certain forms, provided these smart contracts
have published their interfaces.

\nxsubpoint\label{sp:ui.ton.serv} \embt(User interface of a
``ton-service''.)  It turns out that ``ton-services'' (i.e., services
residing in the TON Network and accepting queries through the ADNL and
RLDP protocols of~\ptref{sect:network}; cf.~\ptref{sp:pure.net.serv})
may also profit from having public interfaces, described by TL-schemes
(cf.~\ptref{sp:TL}). A client application, such as a light wallet or a
``ton-browser'', might prompt the user to select one of the methods
and to fill in a form with parameters defined by the interface,
similarly to what has just been discussed in~\ptref{sp:ui.smartc}. The
only difference is that the resulting TL-serialized message is not
submitted as a transaction in the blockchain; instead, it is sent as
an RPC query to the abstract address of the ``ton-service'' in
question, and the response to this query is parsed and displayed
according to the formal interface (i.e., a TL-scheme).

\nxsubpoint\label{sp:ui.ton.dns} \embt(Locating user interfaces via
TON DNS.)  The TON DNS record containing an abstract address of a
ton-service or a smart-contract account identifier might also contain
an optional field describing the public (user) interface of that
entity, or several supported interfaces. Then the client application
(be it a wallet, a ton-browser or a ton-proxy) will be able to
download the interface and interact with the entity in question (be it
a smart contract or a ton-service) in a uniform way.

\nxsubpoint \embt(Blurring the distinction between on-chain and off-chain
services.)  In this way, the distinction between on-chain, off-chain
and mixed services (cf.~\ptref{sp:on.off.chain}) is blurred for the
end user: she simply enters the domain name of the desired service
into the address line of her ton-browser or wallet, and the rest is
handled seamlessly by the client application.

\nxsubpoint\label{sp:telegram.integr} \embt(A light wallet and TON
entity explorer can be built into Telegram Messenger clients.)  An
interesting opportunity arises at this point. A light wallet and TON
entity explorer, implementing the above functionality, can be embedded
into the Telegram Messenger smartphone client application, thus
bringing the technology to more than 200 million people. Users would
be able to send hyperlinks to TON entities and resources by including
TON URIs (cf.~\ptref{sp:ton.uri}) in messages; such hyperlinks, if
selected, will be opened internally by the Telegram client application
of the receiving party, and interaction with the chosen entity will
begin.

\nxsubpoint \embt(``ton-sites'' as ton-services supporting an HTTP
interface.)  A {\em ton-site\/} is simply a ton-service that supports
an HTTP interface, perhaps along with some other interfaces. This
support may be announced in the corresponding TON DNS record.

\nxsubpoint \embt(Hyperlinks.)  Notice that the HTML pages returned by
ton-sites may contain {\em ton-hyperlinks}---that is, references to
other ton-sites, smart contracts and accounts by means of specially
crafted URI schemes (cf.~\ptref{sp:ton.uri})---containing either
abstract network addresses, account identifiers, or human-readable TON
DNS domains. Then a ``ton-browser'' might follow such a hyperlink when
the user selects it, detect the interface to be used, and display a
user interface form as outlined in \ptref{sp:ui.smartc}
and~\ptref{sp:ui.ton.serv}.

\nxsubpoint\label{sp:ton.uri} \embt(Hyperlink URLs may specify some
parameters.)  The hyperlink URLs may contain not only a (TON) DNS
domain or an abstract address of the service in question, but also the
name of the method to be invoked and some or all of its parameters. A
possible URI scheme for this might look as follows:
\begin{quote}
\texttt{ton://}\textit{<domain>}\texttt{/}\textit{<method>}\texttt{?}%
\textit{<field1>}\texttt{=}\textit{<value1>}\texttt{\&}%
\textit{<field2>}\texttt{=}\dots
\end{quote}
When the user selects such a link in a ton-browser, either the action
is performed immediately (especially if it is a get method of a smart
contract, invoked anonymously), or a partially filled form is
displayed, to be explicitly confirmed and submitted by the user (this
may be required for payment forms).

\nxsubpoint \embt(POST actions.)  A ton-site may embed into the HTML
pages it returns some usual-looking POST forms, with POST actions
referring either to ton-sites, ton-services or smart contracts by
means of suitable (TON) URLs. In that case, once the user fills and
submits that custom form, the corresponding action is taken, either
immediately or after an explicit confirmation.

\nxsubpoint\label{sp:ton.www} \embt(TON WWW.)  All of the above will
lead to the creation of a whole web of cross-referencing entities,
residing in the TON Network, which would be accessible to the end user
through a ton-browser, providing the user with a WWW-like browsing
experience. For end users, this will finally make blockchain
applications fundamentally similar to the web sites they are already
accustomed to.

\nxsubpoint \embt(Advantages of TON WWW.)  This ``TON WWW'' of
on-chain and off-chain services has some advantages over its
conventional counterpart. For example, payments are inherently
integrated in the system. User identity can be always presented to the
services (by means of automatically generated signatures on the
transactions and RPC requests generated), or hidden at will. Services
would not need to check and re-check user credentials; these
credentials can be published in the blockchain once and for all. User
network anonymity can be easily preserved by means of TON Proxy, and
all services will be effectively unblockable. Micropayments are also
possible and easy, because ton-browsers can be integrated with the TON
Payments system.

%%%%%%%%%%%%%%%%%%%%%%%%%%%%%%%%%%%%%%%%%%%%%%%%%
%
%
%                  PAYMENTS
%
%
%%%%%%%%%%%%%%%%%%%%%%%%%%%%%%%%%%%%%%%%%%%%%%%%%
\clearpage
\mysection{TON Payments}\label{sect:payments}

The last component of the TON Project we will briefly discuss in this
text is {\em TON Payments}, the platform for (micro)payment channels
and ``lightning network'' value transfers. It would enable ``instant''
payments, without the need to commit all transactions into the
blockchain, pay the associated transaction fees (e.g., for the gas
consumed), and wait five seconds until the block containing the
transactions in question is confirmed.

The overall overhead of such instant payments is so small that one can
use them for micropayments. For example, a TON file-storing service
might charge the user for every 128 KiB of downloaded data, or a paid
TON Proxy might require some tiny micropayment for every 128 KiB of
traffic relayed.

While {\em TON Payments\/} is likely to be released later than the
core components of the TON Project, some considerations need to be
made at the very beginning. For example, the TON Virtual Machine (TON
VM; cf.~\ptref{sp:tonvm}), used to execute the code of TON Blockchain
smart contracts, must support some special operations with Merkle
proofs. If such support is not present in the original design, adding
it at a later stage might become problematic
(cf.~\ptref{sp:genome.change.never}). We will see, however, that the
TON VM comes with natural support for ``smart'' payment channels
(cf.~\ptref{sp:ton.smart.pc.supp}) out of the box.

\mysubsection{Payment Channels}

We start with a discussion of point-to-point payment channels, and how they can be implemented in the TON Blockchain.

\nxsubpoint \embt(The idea of a payment channel.)  Suppose two
parties, $A$ and $B$, know that they will need to make a lot of
payments to each other in the future. Instead of committing each
payment as a transaction in the blockchain, they create a shared
``money pool'' (or perhaps a small private bank with exactly two
accounts), and contribute some funds to it: $A$ contributes $a$
coins, and $B$ contributes $b$ coins. This is achieved by creating a
special smart contract in the blockchain, and sending the money to it.

Before creating the ``money pool'', the two sides agree to a certain
protocol. They will keep track of the {\em state\/} of the pool---that
is, of their balances in the shared pool. Originally, the state is
$(a,b)$, meaning that $a$ coins actually belong to~$A$, and $b$ coins
belong to~$B$. Then, if $A$ wants to pay $d$ coins to $B$, they can
simply agree that the new state is $(a',b')=(a-d,b+d)$. Afterwards,
if, say, $B$ wants to pay $d'$ coins to $A$, the state will become
$(a'',b'')=(a'+d',b'-d')$, and so on.

All this updating of balances inside the pool is done completely
off-chain. When the two parties decide to withdraw their due funds
from the pool, they do so according to the final state of the
pool. This is achieved by sending a special message to the smart
contract, containing the agreed-upon final state $(a^*,b^*)$ along
with the signatures of both~$A$ and $B$. Then the smart contract sends
$a^*$ coins to $A$, $b^*$ coins to $B$ and self-destructs.

This smart contract, along with the network protocol used by $A$ and
$B$ to update the state of the pool, is a simple {\em payment channel
  between $A$ and~$B$.} According to the classification described
in~\ptref{sp:on.off.chain}, it is a {\em mixed\/} service: part of its
state resides in the blockchain (the smart contract), but most of its
state updates are performed off-chain (by the network protocol). If
everything goes well, the two parties will be able to perform as many
payments to each other as they want (with the only restriction being
that the ``capacity'' of the channel is not overrun---i.e., their
balances in the payment channel both remain non-negative), committing
only two transactions into the blockchain: one to open (create) the
payment channel (smart contract), and another to close (destroy) it.

\nxsubpoint \embt(Trustless payment channels.)  The previous example
was somewhat unrealistic, because it assumes that both parties are
willing to cooperate and will never cheat to gain some
advantage. Imagine, for example, that $A$ will choose not to sign the
final balance $(a',b')$ with $a'<a$. This would put $B$ in a difficult
situation.

To protect against such scenarios, one usually tries to develop {\em
  trustless\/} payment channel protocols, which do not require the
parties to trust each other, and make provisions for punishing any
party who would attempt to cheat.

This is usually achieved with the aid of signatures. The payment
channel smart contract knows the public keys of $A$ and $B$, and it
can check their signatures if needed. The payment channel protocol
requires the parties to sign the intermediate states and send the
signatures to each other. Then, if one of the parties cheats---for
instance, pretends that some state of the payment channel never
existed---its misbehavior can be proved by showing its signature on
that state. The payment channel smart contract acts as an ``on-chain
arbiter'', able to process complaints of the two parties about each
other, and punish the guilty party by confiscating all of its money
and awarding it to the other party.

\nxsubpoint\label{sp:simple.sync.pc} \embt(Simple bidirectional
synchronous trustless payment channel.)  Consider the following, more
realistic example: Let the state of the payment channel be described
by triple $(\delta_i,i,o_i)$, where $i$ is the sequence number of the
state (it is originally zero, and then it is increased by one when a
subsequent state appears), $\delta_i$ is the {\em channel imbalance\/}
(meaning that $A$ and $B$ own $a+\delta_i$ and $b-\delta_i$ coins,
respectively), and $o_i$ is the party allowed to generate the next
state (either $A$ or $B$). Each state must be signed both by $A$ and
$B$ before any further progress can be made.

Now, if $A$ wants to transfer $d$ coins to $B$ inside the payment
channel, and the current state is $S_i=(\delta_i,i,o_i)$ with $o_i=A$,
then it simply creates a new state $S_{i+1}=(\delta_i-d,i+1,o_{i+1})$,
signs it, and sends it to $B$ along with its signature. Then $B$
confirms it by signing and sending a copy of its signature to
$A$. After that, both parties have a copy of the new state with both
of their signatures, and a new transfer may occur.

If $A$ wants to transfer coins to $B$ in a state $S_i$ with $o_i=B$,
then it first asks $B$ to commit a subsequent state $S_{i+1}$ with the
same imbalance $\delta_{i+1}=\delta_i$, but with $o_{i+1}=A$. After
that, $A$ will be able to make its transfer.

When the two parties agree to close the payment channel, they both put
their special {\em final\/} signatures on the state $S_k$ they believe
to be final, and invoke the {\em clean\/} or {\em two-sided
finalization method\/} of the payment channel smart contract by sending
it the final state along with both final signatures.

If the other party does not agree to provide its final signature, or
simply if it stops responding, it is possible to close the channel
unilaterally. For this, the party wishing to do so will invoke the
{\em unilateral finalization\/} method, sending to the smart contract
its version of the final state, its final signature, and the most
recent state having a signature of the other party. After that, the
smart contract does not immediately act on the final state
received. Instead, it waits for a certain period of time (e.g., one
day) for the other party to present its version of the final
state. When the other party submits its version and it turns out to be
compatible with the already submitted version, the ``true'' final
state is computed by the smart contract and used to distribute the
money accordingly. If the other party fails to present its version of
the final state to the smart contract, then the money is redistributed
according to the only copy of the final state presented.

If one of the two parties cheats---for example, by signing two
different states as final, or by signing two different next
states $S_{i+1}$ and $S'_{i+1}$, or by signing an invalid new state
$S_{i+1}$ (e.g., with imbalance $\delta_{i+1}<-a$ or $>b$)---then the
other party may submit proof of this misbehavior to a third method of
the smart contract. The guilty party is punished immediately by losing
its share in the payment channel completely.

This simple payment channel protocol is {\em fair\/} in the sense that
any party can always get its due, with or without the cooperation of
the other party, and is likely to lose all of its funds committed to
the payment channel if it tries to cheat.

\nxsubpoint\label{sp:sync.pc.as.blockch} \embt(Synchronous payment
channel as a simple virtual blockchain with two validators.)  The
above example of a simple synchronous payment channel can be recast as
follows. Imagine that the sequence of states $S_0$, $S_1$, \dots,
$S_n$ is actually the sequence of blocks of a very simple
blockchain. Each block of this blockchain contains essentially only
the current state of the blockchain, and maybe a reference to the
previous block (i.e., its hash). Both parties $A$ and $B$ act as
validators for this blockchain, so every block must collect both of
their signatures. The state $S_i$ of the blockchain defines the
designated producer $o_i$ for the next block, so there is no race
between $A$ and $B$ for producing the next block. Producer $A$ is
allowed to create blocks that transfer funds from $A$ to $B$ (i.e.,
decrease the imbalance: $\delta_{i+1}\leq\delta_i$), and $B$ can only
transfer funds from $B$ to $A$ (i.e., increase $\delta$).

If the two validators agree on the final block (and the final state)
of the blockchain, it is finalized by collecting special ``final''
signatures of the two parties, and submitting them along with the
final block to the channel smart contract for processing and
re-distributing the money accordingly.

If a validator signs an invalid block, or creates a fork, or signs two
different final blocks, it can be punished by presenting a proof of
its misbehavior to the smart contract, which acts as an ``on-chain
arbiter'' for the two validators; then the offending party will lose
all its money kept in the payment channel, which is analogous to a
validator losing its stake.

\nxsubpoint\label{sp:async.pc} \embt(Asynchronous payment channel as a
virtual blockchain with two workchains.)  The synchronous payment
channel discussed in \ptref{sp:simple.sync.pc} has a certain
disadvantage: one cannot begin the next transaction (money transfer
inside the payment channel) before the previous one is confirmed by
the other party. This can be fixed by replacing the single virtual
blockchain discussed in~\ptref{sp:sync.pc.as.blockch} by a system of
two interacting virtual workchains (or rather shardchains).

The first of these workchains contains only transactions by $A$, and
its blocks can be generated only by~$A$; its states are
$S_i=(i,\phi_i,j,\psi_j)$, where $i$ is the block sequence number
(i.e., the count of transactions, or money transfers, performed by $A$
so far), $\phi_i$ is the total amount transferred from $A$ to $B$ so
far, $j$ is the sequence number of the most recent valid block in
$B$'s blockchain that $A$ is aware of, and $\psi_j$ is the amount of
money transferred from $B$ to $A$ in its $j$ transactions. A signature
of $B$ put onto its $j$-th block should also be a part of this
state. Hashes of the previous block of this workchain and of the
$j$-th block of the other workchain may be also included. Validity
conditions for $S_i$ include $\phi_i\geq 0$, $\phi_i\geq\phi_{i-1}$ if
$i>0$, $\psi_j\geq0$, and $-a\leq\psi_j-\phi_i\leq b$.

Similarly, the second workchain contains only transactions by $B$, and
its blocks are generated only by~$B$; its states are
$T_j=(j,\psi_j,i,\phi_i)$, with similar validity conditions.

Now, if $A$ wants to transfer some money to $B$, it simply creates a
new block in its workchain, signs it, and sends to $B$, without
waiting for confirmation.

The payment channel is finalized by $A$ signing (its version of) the
final state of its blockchain (with its special ``final signature''),
$B$ signing the final state of its blockchain, and presenting these
two final states to the clean finalization method of the payment
channel smart contract. Unilateral finalization is also possible, but
in that case the smart contract will have to wait for the other party
to present its version of the final state, at least for some grace
period.

\nxsubpoint \embt(Unidirectional payment channels.)  If only $A$ needs
to make payments to $B$ (e.g., $B$ is a service provider, and $A$ its
client), then a unilateral payment channel can be
created. Essentially, it is just the first workchain described
in~\ptref{sp:async.pc} without the second one. Conversely, one can say
that the asynchronous payment channel described in \ptref{sp:async.pc}
consists of two unidirectional payment channels, or ``half-channels'',
managed by the same smart contract.

\nxsubpoint\label{sp:pc.promises} \embt(More sophisticated payment
channels. Promises.)  We will see later in~\ptref{sp:ch.money.tr} that
the ``lightning network'' (cf.~\ptref{sect:lightning}), which enables
instant money transfers through chains of several payment channels,
requires higher degrees of sophistication from the payment channels
involved.

In particular, we want to be able to commit ``promises'', or
``conditional money transfers'': $A$ agrees to send $c$ coins to $B$,
but $B$ will get the money only if a certain condition is fulfilled,
for instance, if $B$ can present some string $u$ with $\Hash(u)=v$ for
a known value of $v$. Otherwise, $A$ can get the money back after a
certain period of time.

Such a promise could easily be implemented on-chain by a simple smart
contract. However, we want promises and other kinds of conditional
money transfers to be possible off-chain, in the payment channel,
because they considerably simplify money transfers along a chain of
payment channels existing in the ``lightning network''
(cf.~\ptref{sp:ch.money.tr}).

The ``payment channel as a simple blockchain'' picture outlined
in~\ptref{sp:sync.pc.as.blockch} and~\ptref{sp:async.pc} becomes
convenient here. Now we consider a more complicated virtual
blockchain, the state of which contains a set of such unfulfilled
``promises'', and the amount of funds locked in such promises. This
blockchain---or the two workchains in the asynchronous case---will
have to refer explicitly to the previous blocks by their
hashes. Nevertheless, the general mechanism remains the same.

\nxsubpoint\label{sp:sm.pc.chal} \embt(Challenges for the
sophisticated payment channel smart contracts.)  Notice that, while
the final state of a sophisticated payment channel is still small, and
the ``clean'' finalization is simple (if the two sides have agreed on
their amounts due, and both have signed their agreement, nothing else
remains to be done), the unilateral finalization method and the method
for punishing fraudulent behavior need to be more complex. Indeed, they
must be able to accept Merkle proofs of misbehavior, and to check
whether the more sophisticated transactions of the payment channel
blockchain have been processed correctly.

In other words, the payment channel smart contract must be able to
work with Merkle proofs, to check their ``hash validity'', and must
contain an implementation of $\evtrans$ and $\evblock$ functions
(cf.~\ptref{sp:blk.transf}) for the payment channel (virtual)
blockchain.

\nxsubpoint\label{sp:ton.smart.pc.supp} \embt(TON VM support for
``smart'' payment channels.)  The TON VM, used to run the code of TON
Blockchain smart contracts, is up to the challenge of executing the
smart contracts required for ``smart'', or sophisticated, payment
channels (cf.~\ptref{sp:sm.pc.chal}).

At this point the ``everything is a bag of cells'' paradigm
(cf.~\ptref{sp:everything.is.BoC}) becomes extremely convenient. Since
all blocks (including the blocks of the ephemeral payment channel
blockchain) are represented as bags of cells (and described by some
algebraic data types), and the same holds for messages and Merkle
proofs as well, a Merkle proof can easily be embedded into an inbound
message sent to the payment channel smart contract. The ``hash
condition'' of the Merkle proof will be checked automatically, and
when the smart contract accesses the ``Merkle proof'' presented, it
will work with it as if it were a value of the corresponding algebraic
data type---albeit incomplete, with some subtrees of the tree replaced
by special nodes containing the Merkle hash of the omitted
subtree. Then the smart contract will work with that value, which
might represent, for instance, a block of the payment channel
(virtual) blockchain along with its state, and will evaluate the
$\evblock$ function (cf.~\ptref{sp:blk.transf}) of that blockchain on
this block and the previous state. Then either the computation
finishes, and the final state can be compared with that asserted in
the block, or an ``absent node'' exception is thrown while attempting
to access an absent subtree, indicating that the Merkle proof is
invalid.

In this way, the implementation of the verification code for smart
payment channel blockchains turns out to be quite straightforward
using TON Blockchain smart contracts. One might say that {\em the TON
  Virtual Machine comes with built-in support for checking the
  validity of other simple blockchains.} The only limiting factor is
the size of the Merkle proof to be incorporated into the inbound
message to the smart contract (i.e., into the transaction).

\nxsubpoint\label{sp:pc.within.pc} \embt(Simple payment channel within
a smart payment channel.)  We would like to discuss the possibility of
creating a simple (synchronous or asynchronous) payment channel inside
an existing payment channel.

While this may seem somewhat convoluted, it is not much harder to
understand and implement than the ``promises'' discussed
in~\ptref{sp:pc.promises}. Essentially, instead of promising to pay
$c$ coins to the other party if a solution to some hash problem is
presented, $A$ promises to pay up to $c$ coins to $B$ according to the
final settlement of some other (virtual) payment channel
blockchain. Generally speaking, this other payment channel blockchain
need not even be between $A$ and $B$; it might involve some other
parties, say, $C$ and $D$, willing to commit $c$ and $d$ coins into
their simple payment channel, respectively. (This possibility is
exploited later in~\ptref{sp:virt.pc}.)

If the encompassing payment channel is asymmetric, two promises need
to be committed into the two workchains: $A$ will promise to pay
$-\delta$ coins to $B$ if the final settlement of the ``internal''
simple payment channel yields a negative final imbalance $\delta$ with
$0\leq-\delta\leq c$; and $B$ will have to promise to pay $\delta$ to
$A$ if $\delta$ is positive. On the other hand, if the encompassing
payment channel is symmetric, this can be done by committing a single
``simple payment channel creation'' transaction with parameters
$(c,d)$ into the single payment channel blockchain by~$A$ (which would
freeze $c$ coins belonging to~$A$), and then committing a special
``confirmation transaction'' by~$B$ (which would freeze $d$ coins
of~$B$).

We expect the internal payment channel to be extremely simple (e.g.,
the simple synchronous payment channel discussed
in~\ptref{sp:simple.sync.pc}), to minimize the size of Merkle proofs
to be submitted. The external payment channel will have to be
``smart'' in the sense described in~\ptref{sp:pc.promises}.

\mysubsection{Payment Channel Network, or ``Lightning
  Network''}\label{sect:lightning}

Now we are ready to discuss the ``lightning network'' of TON Payments
that enables instant money transfers between any two participating
nodes.

\nxsubpoint \embt(Limitations of payment channels.)  A payment channel
is useful for parties who expect a lot of money transfers between
them. However, if one needs to transfer money only once or twice to a
particular recipient, creating a payment channel with her would be
impractical. Among other things, this would imply freezing a
significant amount of money in the payment channel, and would require
at least two blockchain transactions anyway.

\nxsubpoint \embt(Payment channel networks, or ``lightning
networks''.)  Payment channel networks overcome the limitations of
payment channels by enabling money transfers along {\em chains} of
payment channels. If $A$ wants to transfer money to $E$, she does not
need to establish a payment channel with $E$. It would be sufficient
to have a chain of payment channels linking $A$ with $E$ through
several intermediate nodes---say, four payment channels: from $A$ to
$B$, from $B$ to $C$, from $C$ to $D$ and from $D$ to $E$.

\nxsubpoint \embt(Overview of payment channel networks.)  Recall that
a {\em payment channel network}, known also as a ``lightning
network'', consists of a collection of participating nodes, some of
which have established long-lived payment channels between them. We
will see in a moment that these payment channels will have to be
``smart'' in the sense of~\ptref{sp:pc.promises}. When a
participating node $A$ wants to transfer money to any other
participating node $E$, she tries to find a path linking $A$ to $E$
inside the payment channel network. When such a path is found, she
performs a ``chain money transfer'' along this path.

\nxsubpoint\label{sp:ch.money.tr} \embt(Chain money transfers.)
Suppose that there is a chain of payment channels from $A$ to $B$,
from $B$ to $C$, from $C$ to $D$, and from $D$ to $E$. Suppose,
further, that $A$ wants to transfer $x$ coins to $E$.

A simplistic approach would be to transfer $x$ coins to $B$ along
the existing payment channel, and ask him to forward the money further
to $C$. However, it is not evident why $B$ would not simply take the
money for himself. Therefore, one must employ a more sophisticated
approach, not requiring all parties involved to trust each other.

This can be achieved as follows. $A$ generates a large random number
$u$ and computes its hash $v=\Hash(u)$. Then she creates a promise to
pay $x$ coins to $B$ if a number $u$ with hash $v$ is presented
(cf.~\ptref{sp:pc.promises}), inside her payment channel
with~$B$. This promise contains $v$, but not $u$, which is still kept
secret.

After that, $B$ creates a similar promise to $C$ in their payment
channel. He is not afraid to give such a promise, because he is aware
of the existence of a similar promise given to him by $A$. If $C$ ever
presents a solution of the hash problem to collect $x$ coins promised
by $B$, then $B$ will immediately submit this solution to $A$ to
collect $x$ coins from $A$.

Then similar promises of $C$ to $D$ and of $D$ to $E$ are
created. When the promises are all in place, $A$ triggers the transfer
by communicating the solution $u$ to all parties involved---or just to
$E$.

Some minor details are omitted in this description. For example, these
promises must have different expiration times, and the amount promised
might slightly differ along the chain ($B$ might promise only
$x-\epsilon$ coins to $C$, where $\epsilon$ is a small pre-agreed
transit fee). We ignore such details for the time being, because they
are not too relevant for understanding how payment channels work and
how they can be implemented in TON.

\nxsubpoint\label{sp:virt.pc} \embt(Virtual payment channels inside a
chain of payment channels.)  Now suppose that $A$ and $E$ expect to
make a lot of payments to each other. They might create a new payment
channel between them in the blockchain, but this would still be quite
expensive, because some funds would be locked in this payment
channel. Another option would be to use chain money transfers
described in~\ptref{sp:ch.money.tr} for each payment. However, this
would involve a lot of network activity and a lot of transactions in
the virtual blockchains of all payment channels involved.

An alternative is to create a virtual payment channel inside the chain
linking $A$ to $E$ in the payment channel network. For this, $A$ and
$E$ create a (virtual) blockchain for their payments, as if they were
going to create a payment channel in the blockchain. However, instead
of creating a payment channel smart contract in the blockchain, they
ask all intermediate payment channels---those linking $A$ to $B$, $B$
to $C$, etc.---to create simple payment channels inside them, bound
to the virtual blockchain created by $A$ and $E$
(cf.~\ptref{sp:pc.within.pc}). In other words, now a promise to
transfer money according to the final settlement between $A$ and~$E$
exists inside every intermediate payment channel.

If the virtual payment channel is unidirectional, such promises can be
implemented quite easily, because the final imbalance $\delta$ is
going to be non-positive, so simple payment channels can be created
inside intermediate payment channels in the same order as described
in~\ptref{sp:ch.money.tr}. Their expiration times can also be set in
the same way.

If the virtual payment channel is bidirectional, the situation is
slightly more complicated. In that case, one should split the promise
to transfer $\delta$ coins according to the final settlement into two
half-promises, as explained in \ptref{sp:pc.within.pc}: to transfer
$\delta^-=\max(0,-\delta)$ coins in the forward direction, and to
transfer $\delta^+=\max(0,\delta)$ in the backward direction. These
half-promises can be created in the intermediate payment channels
independently, one chain of half-promises in the direction from $A$
to~$E$, and the other chain in the opposite direction.

\nxsubpoint\label{sp:lnet.find.path} \embt(Finding paths in the
lightning network.)  One point remains undiscussed so far: how will
$A$ and $E$ find a path connecting them in the payment network?  If
the payment network is not too large, an OSPF-like protocol can be
used: all nodes of the payment network create an overlay network
(cf.~\ptref{sp:net.within.net}), and then every node propagates all
available link (i.e., participating payment channel) information to
its neighbors by a gossip protocol. Ultimately, all nodes will have a
complete list of all payment channels participating in the payment
network, and will be able to find the shortest paths by
themselves---for example, by applying a version of Dijkstra's
algorithm modified to take into account the ``capacities'' of the
payment channels involved (i.e., the maximal amounts that can be
transferred along them). Once a candidate path is found, it can be
probed by a special ADNL datagram containing the full path, and asking
each intermediate node to confirm the existence of the payment channel
in question, and to forward this datagram further according to the
path. After that, a chain can be constructed, and a protocol for chain
transfers (cf.~\ptref{sp:ch.money.tr}), or for creating a virtual
payment channel inside a chain of payment channels
(cf.~\ptref{sp:virt.pc}), can be run.

\nxsubpoint \embt(Optimizations.)  Some optimizations might be done
here. For example, only transit nodes of the lightning network need to
participate in the OSPF-like protocol discussed
in~\ptref{sp:lnet.find.path}. Two ``leaf'' nodes wishing to connect
through the lightning network would communicate to each other the
lists of transit nodes they are connected to (i.e., with which they
have established payment channels participating in the payment
network). Then paths connecting transit nodes from one list to transit
nodes from the other list can be inspected as outlined above
in~\ptref{sp:lnet.find.path}.

\nxsubpoint \embt(Conclusion.)  We have outlined how the blockchain
and network technologies of the TON project are adequate to the task
of creating {\em TON Payments}, a platform for off-chain instant money
transfers and micropayments. This platform can be extremely useful
for services residing in the TON ecosystem, allowing them to easily
collect micropayments when and where required.

%%%%%%%%%%%%%%%%%%%%%%%%%%%%%%%%%%%%%%%%%%%%%%%%%
%
%
%                  CONCLUSION
%
%
%%%%%%%%%%%%%%%%%%%%%%%%%%%%%%%%%%%%%%%%%%%%%%%%%

\clearpage
\section*{Conclusion}
\markbothsame{\textsc{Conclusion}}
\addcontentsline{toc}{section}{Conclusion}

We have proposed a scalable multi-blockchain architecture capable of
supporting a massively popular cryptocurrency and decentralized
applications with user-friendly interfaces.

To achieve the necessary scalability, we proposed the {\em TON
  Blockchain}, a ``tightly-coupled'' multi-blockchain system
(cf.~\ptref{sp:blkch.interact}) with bottom-up approach to sharding
(cf.~\ptref{sp:shard.supp} and~\ptref{sp:ISP}). To further increase
potential performance, we introduced the 2-blockchain mechanism for
replacing invalid blocks (cf.~\ptref{sp:inv.sh.blk.corr}) and Instant
Hypercube Routing for faster communication between shards
(cf.~\ptref{sp:instant.hypercube}). A brief comparison of the TON
Blockchain to existing and proposed blockchain projects
(cf.~\ptref{sect:class.blkch} and~\ptref{sect:compare.blkch})
highlights the benefits of this approach for systems that seek to
handle millions of transactions per second.

The {\em TON Network}, described in Chapter~\ptref{sect:network},
covers the networking demands of the proposed multi-blockchain
infrastructure. This network component may also be used in combination
with the blockchain to create a wide spectrum of applications and
services, impossible using the blockchain alone
(cf.~\ptref{sp:blockchain.facebook}). These services, discussed in
Chapter~\ptref{sect:services}, include {\em TON DNS}, a service for
translating human-readable object identifiers into their addresses;
{\em TON Storage}, a distributed platform for storing arbitrary files;
{\em TON Proxy}, a service for anonymizing network access and
accessing TON-powered services; and {\em TON Payments\/}
(cf. Chapter~\ptref{sect:payments}), a platform for instant off-chain
money transfers across the TON ecosystem that applications may use for
micropayments.

The TON infrastructure allows for specialized light client wallet and
``ton-browser'' desktop and smartphone applications that enable a
browser-like experience for the end user (cf.~\ptref{sp:ton.www}),
making cryptocurrency payments and interaction with smart contracts
and other services on the TON Platform accessible to the mass
user. Such a light client can be integrated into the Telegram
Messenger client (cf.~\ptref{sp:telegram.integr}), thus eventually
bringing a wealth of blockchain-based applications to hundreds of
millions of users.

%%%%%%%%%%%%%%%%%%%%%%%%%%%%%%%%%%%%%%%%%%%%%%%%%
%
%
%                  BIBLIOGRAPHY
%
%
%%%%%%%%%%%%%%%%%%%%%%%%%%%%%%%%%%%%%%%%%%%%%%%%%

\clearpage
\markbothsame{\textsc{References}}

\begin{thebibliography}{2}

\bibitem{Birman}
  {\sc K.~Birman}, {\sl Reliable Distributed Systems: Technologies, Web Services and Applications}, Springer, 2005.

\bibitem{EthWP}
  {\sc V.~Buterin}, {\sl Ethereum: A next-generation smart contract and decentralized application platform}, \url{https://github.com/ethereum/wiki/wiki/White-Paper}, 2013.

\bibitem{BenOr}
  {\sc M.~Ben-Or, B.~Kelmer, T.~Rabin}, {\sl Asynchronous secure computations with optimal resilience}, in {\em Proceedings of the thirteenth annual ACM symposium on Principles of distributed computing}, p.~183--192. ACM, 1994.

\bibitem{PBFT}
  {\sc M.~Castro, B.~Liskov, et al.}, {\sl Practical byzantine fault tolerance}, {\it Proceedings of the Third Symposium on Operating Systems Design and Implementation\/} (1999), p.~173--186, available at \url{http://pmg.csail.mit.edu/papers/osdi99.pdf}.

\bibitem{EOSWP}
  {\sc EOS.IO}, {\sl EOS.IO technical white paper}, \url{https://github.com/EOSIO/Documentation/blob/master/TechnicalWhitePaper.md}, 2017.

\bibitem{Onion}
  {\sc D.~Goldschlag, M.~Reed, P.~Syverson}, {\sl Onion Routing for Anonymous and Private Internet Connections}, {\it Communications of the ACM}, {\bf 42}, num.~2 (1999), \url{http://www.onion-router.net/Publications/CACM-1999.pdf}.

\bibitem{Byzantine}
  {\sc L.~Lamport, R.~Shostak, M.~Pease}, {\sl The byzantine generals problem}, {\it ACM Transactions on Programming Languages and Systems}, {\bf 4/3} (1982), p.~382--401.

\bibitem{BitShWP}
  {\sc S.~Larimer}, {\sl The history of BitShares}, \url{https://docs.bitshares.org/bitshares/history.html}, 2013.

\bibitem{RaptorQ}
  {\sc M.~Luby, A.~Shokrollahi, et al.}, {\sl RaptorQ forward error correction scheme for object delivery}, IETF RFC 6330, \url{https://tools.ietf.org/html/rfc6330}, 2011.

\bibitem{Kademlia}
  {\sc P.~Maymounkov, D.~Mazi\`eres}, {\sl Kademlia: A peer-to-peer information system based on the XOR metric}, in {\em IPTPS '01 revised papers from the First International Workshop on Peer-to-Peer Systems}, p.~53--65, available at \url{http://pdos.csail.mit.edu/~petar/papers/maymounkov-kademlia-lncs.pdf}, 2002.

\bibitem{HoneyBadger}
  {\sc A.~Miller, Yu Xia, et al.}, {\sl The honey badger of BFT protocols}, Cryptology e-print archive 2016/99, \url{https://eprint.iacr.org/2016/199.pdf}, 2016.

\bibitem{BitcWP}
  {\sc S.~Nakamoto}, {\sl Bitcoin: A peer-to-peer electronic cash system}, \url{https://bitcoin.org/bitcoin.pdf}, 2008.

\bibitem{STGM}
  {\sc S.~Peyton Jones}, {\sl Implementing lazy functional languages on stock hardware: the Spineless Tagless G-machine}, {\it Journal of Functional Programming\/} {\bf 2} (2), p.~127--202, 1992.

\bibitem{Raptor}
  {\sc A.~Shokrollahi, M.~Luby}, {\sl Raptor Codes}, {\it IEEE Transactions on Information Theory\/} {\bf 6}, no.\ 3--4 (2006), p.~212--322.

\bibitem{DistrSys}
  {\sc M.~van Steen, A.~Tanenbaum}, {\sl Distributed Systems, 3rd ed.}, 2017.

\bibitem{HoTT}
  {\sc The Univalent Foundations Program}, {\sl Homotopy Type Theory: Univalent Foundations of Mathematics}, Institute for Advanced Study, 2013, available at \url{https://homotopytypetheory.org/book}.

\bibitem{PolkaWP}
  {\sc G.~Wood}, {\sl PolkaDot: vision for a heterogeneous multi-chain framework}, draft~1, \url{https://github.com/w3f/polkadot-white-paper/raw/master/PolkaDotPaper.pdf}, 2016.

\end{thebibliography}

%%%%%%%%%%%%%%%%%%%%%%%%%%%%%%%%%%%%%%%%%%%%%%%%%
%
%
%                  APPENDICES
%
%
%%%%%%%%%%%%%%%%%%%%%%%%%%%%%%%%%%%%%%%%%%%%%%%%%
\clearpage
\appendix
\myappendix{The TON Coin, or the Gram}\label{app:coins}

The principal cryptocurrency of the TON Blockchain, and in particular
of its masterchain and basic workchain, is the {\em TON Coin}, also
known as the {\em Gram\/} (GRM). It is used to make deposits required
to become a validator; transaction fees, gas payments (i.e.,
smart-contract message processing fees) and persistent storage
payments are also usually collected in Grams.

\nxpoint \embt(Subdivision and terminology.)  A {\em Gram\/} is
subdivided into one billion ($10^9$) smaller units, called {\em
  nanograms}, {\em ngrams} or simply {\em nanos}. All transfers and
account balances are expressed as non-negative integer multiples of
nanos. Other units include:
\begin{itemize}
\item A {\em nano}, {\em ngram} or {\em nanogram} is the smallest
  unit, equal to $10^{-9}$ Grams.
\item A {\em micro\/} or {\em microgram\/} equals one thousand
  ($10^3$) nanos.
\item A {\em milli\/} is one million ($10^6$) nanos, or one thousandth
  part ($10^{-3}$) of a Gram.
\item A {\em Gram\/} equals one billion ($10^9$) nanos.
\item A {\em kilogram}, or {\em kGram}, equals one thousand ($10^3$)
  Grams.
\item A {\em megagram}, or {\em MGram}, equals one million ($10^6$)
  Grams, or $10^{15}$ nanos.
\item Finally, a {\em gigagram}, or {\em GGram}, equals one billion
  ($10^9$) Grams, or $10^{18}$ nanos.
\end{itemize}

There will be no need for larger units, because the initial supply of
Grams will be limited to five billion ($5\cdot10^9$) Grams (i.e., 5
Gigagrams).

\nxpoint \embt(Smaller units for expressing gas prices.)  If the
necessity for smaller units arises, ``specks'' equal to $2^{-16}$
nanograms will be used. For example, gas prices may be indicated in
specks. However, the actual fee to be paid, computed as the product of
the gas price and the amount of gas consumed, will be always rounded
down to the nearest multiple of $2^{16}$ specks and expressed as an
integer number of nanos.

\nxpoint \embt(Original supply, mining rewards and inflation.)  The
total supply of Grams is originally limited to $5$ Gigagrams (i.e.,
five billion Grams or $5\cdot10^{18}$ nanos).

This supply will increase very slowly, as rewards to validators for
mining new masterchain and shardchain blocks accumulate. These rewards
would amount to approximately $20\%$ (the exact number may be adjusted
in future) of the validator's stake per year, provided the validator
diligently performs its duties, signs all blocks, never goes offline
and never signs invalid blocks. In this way, the validators will have
enough profit to invest into better and faster hardware needed to
process the ever growing quantity of users' transactions.

We expect that at most $10\%$\footnote{The maximum total amount of
  validator stakes is a configurable parameter of the blockchain, so
  this restriction can be enforced by the protocol if necessary.} of
the total supply of Grams, on average, will be bound in validator
stakes at any given moment. This will produce an inflation rate of
$2\%$ per year, and as a result, will double the total supply of Grams
(to ten Gigagrams) in 35 years. Essentially, this inflation represents
a payment made by all members of the community to the validators for
keeping the system up and running.

On the other hand, if a validator is caught misbehaving, a part or all
of its stake will be taken away as a punishment, and a larger portion
of it will subsequently be ``burned'', decreasing the total supply of
Grams. This would lead to deflation. A smaller portion of the fine may
be redistributed to the validator or the ``fisherman'' who committed a
proof of the guilty validator's misbehavior.

\nxpoint\label{sp:gram.price} \embt(Original price of Grams.)  The
price of the first Gram to be sold will equal approximately
$\$0.1$ (USD). Every subsequent Gram to be sold (by the TON Reserve,
controlled by the TON Foundation) will be priced one billionth higher
than the previous one. In this way, the $n$-th Gram to be put into
circulation will be sold at approximately
\begin{equation}\label{eq:gram.price}
  p(n)\approx 0.1\cdot (1+10^{-9})^n\quad\text{USD},
\end{equation}
or an approximately equivalent (because of quickly changing market
exchange rates) amount of other (crypto)currencies, such as BTC or
ETH.

\nxsubpoint\label{sp:exp.priced} \embt(Exponentially priced
cryptocurrencies.)  We say that the Gram is an {\em exponentially
  priced cryptocurrency}, meaning that the price of the $n$-th Gram to
be put into circulation is approximately $p(n)$ given by the formula
\begin{equation}
  p(n)=p_0\cdot e^{\alpha n}
\end{equation}
with specific values $p_0=0.1$ USD and $\alpha=10^{-9}$.

More precisely, a small fraction $dn$ of a new coin is worth
$p(n)\,dn$ dollars, once $n$ coins are put into circulation. (Here $n$
is not necessarily an integer.)

Other important parameters of such a cryptocurrency include $n$, the
total number of coins in circulation, and $N\geq n$, the total number
of coins that can exist. For the Gram, $N=5\cdot 10^9$.

\nxsubpoint \embt(Total price of first $n$ coins.)  The total price
$T(n)=\int_0^n p(n)\,dn\approx p(0)+p(1)+\cdots+p(n-1)$ of the first
$n$ coins of an exponentially priced cryptocurrency (e.g., the Gram)
to be put into circulation can be computed by
\begin{equation}
  T(n)=p_0\cdot\alpha^{-1}(e^{\alpha n}-1)\quad.
\end{equation}

\nxsubpoint \embt(Total price of next $\Delta n$ coins.)  The total
price $T(n+\Delta n)-T(n)$ of $\Delta n$ coins put into circulation
after $n$ previously existing coins can be computed by
\begin{equation}\label{eq:T.m.n}
  T(n+\Delta n)-T(n)=p_0\cdot\alpha^{-1}(e^{\alpha(n+\Delta n)}-e^{\alpha n})
  =p(n)\cdot\alpha^{-1}(e^{\alpha\,\Delta n}-1)\quad.
\end{equation}

\nxsubpoint \embt(Buying next coins with total value $T$.)  Suppose
that $n$ coins have already been put into circulation, and that one
wants to spend $T$ (dollars) on buying new coins. The quantity of
newly-obtained coins $\Delta n$ can be computed by putting $T(n+\Delta
n)-T(n)=T$ into \eqref{eq:T.m.n}, yielding
\begin{equation}\label{eq:new.coins}
  \Delta n=\alpha^{-1}\log\left(1+\frac{T\cdot\alpha}{p(n)}\right)\quad.
\end{equation}
Of course, if $T\lll p(n)\alpha^{-1}$, then $\Delta n\approx T/p(n)$.

\nxsubpoint \embt(Market price of Grams.)  Of course, if the free
market price falls below $p(n):=0.1\cdot (1+10^{-9})^n$, once $n$
Grams are put into circulation, nobody would buy new Grams from the
TON Reserve; they would choose to buy their Grams on the free market
instead, without increasing the total quantity of Grams in
circulation. On the other hand, the market price of a Gram cannot
become much higher than $p(n)$, otherwise it would make sense to
obtain new Grams from the TON Reserve. This means that the market
price of Grams would not be subject to sudden spikes (and drops); this
is important because stakes (validator deposits) are frozen for at
least one month, and gas prices cannot change too fast either. So, the
overall economic stability of the system requires some mechanism that
would prevent the exchange rate of the Gram from changing too
drastically, such as the one described above.

\nxsubpoint \embt(Buying back the Grams.)  If the market price of the
Gram falls below $0.5\cdot p(n)$, when there are a total of $n$ Grams
in circulation (i.e., not kept on a special account controlled by the
TON Reserve), the TON Reserve reserves the right to buy some Grams
back and decrease $n$, the total quantity of Grams in
circulation. This may be required to prevent sudden falls of the
Gram exchange rate.

\nxsubpoint \embt(Selling new Grams at a higher price.)  The TON
Reserve will sell only up to one half (i.e., $2.5\cdot10^9$ Grams) of
the total supply of Grams according to the price
formula~\eqref{eq:gram.price}.  It reserves the right not to sell any
of the remaining Grams at all, or to sell them at a higher price than
$p(n)$, but never at a lower price (taking into account the uncertainty
of quickly changing exchange rates). The rationale here is that once
at least half of all Grams have been sold, the total value of the Gram
market will be sufficiently high, and it will be more difficult for outside forces to manipulate the exchange rate than it may be at the very
beginning of the Gram's deployment.

\nxpoint\label{sp:unalloc.gr} \embt(Using unallocated Grams.)  The TON
Reserve will use the bulk of ``unallocated'' Grams (approximately
$5\cdot10^9-n$ Grams)---i.e., those residing in the special account of
the TON Reserve and some other accounts explicitly linked to it---only
as validator stakes (because the TON Foundation itself will likely
have to provide most of the validators during the first deployment
phase of the TON Blockchain), and for voting in the masterchain for or
against proposals concerning changes in the ``configurable
parameters'' and other protocol changes, in the way determined by the
TON Foundation (i.e., its creators---the development team). This also
means that the TON Foundation will have a majority of votes during the
first deployment phase of the TON Blockchain, which may be useful if a
lot of parameters end up needing to be adjusted, or if the need arises
for hard or soft forks. Later, when less than half of all Grams remain
under control of the TON Foundation, the system will become more
democratic. Hopefully it will have become more mature by then, without
the need to adjust parameters too frequently.

\nxsubpoint\label{sp:dev.grams} \embt(Some unallocated Grams will be
given to developers.)  A predefined (relatively small) quantity of
``unallocated'' Grams (e.g., 200 Megagrams, equal to 4\% of the total
supply) will be transferred during the deployment of the TON
Blockchain to a special account controlled by the TON Foundation, and
then some ``rewards'' may be paid from this account to the developers
of the open source TON software, with a minimum two-year vesting
period.

\nxsubpoint\label{sp:TON.own.grams} \embt(The TON Foundation needs
Grams for operational purposes.)  Recall that the TON Foundation will
receive the fiat and cryptocurrency obtained by selling Grams from the
TON Reserve, and will use them for the development and deployment of
the TON Project. For instance, the original set of validators, as well
as an initial set of TON Storage and TON Proxy nodes may be installed
by the TON Foundation.

While this is necessary for the quick start of the project, the
ultimate goal is to make the project as decentralized as possible. To
this end, the TON Foundation may need to encourage installation of
third-party validators and TON Storage and TON Proxy nodes---for
example, by paying them for storing old blocks of the TON Blockchain
or proxying network traffic of a selected subset of services. Such
payments will be made in Grams; therefore, the TON Foundation will
need a significant amount of Grams for operational purposes.

\nxsubpoint \embt(Taking a pre-arranged amount from the Reserve.) The
TON Foundation will transfer to its account a small part of the TON
Reserve---say, 10\% of all coins (i.e.\ 500 Megagrams) after the end
of the initial sale of Grams---to be used for its own purposes as
outlined in~\ptref{sp:TON.own.grams}. This is best done simultaneously
with the transfer of the funds intended for TON developers, as
mentioned in~\ptref{sp:dev.grams}.

After the transfers to the TON Foundation and the TON developers, the
TON Reserve price $p(n)$ of the Gram will immediately rise by a
certain amount, known in advance. For example, if 10\% of all coins
are transferred for the purposes of the TON Foundation, and 4\% are
transferred for the encouragement of the developers, then the total
quantity $n$ of coins in circulation will immediately increase by
$\Delta n=7\cdot10^8$, with the price of the Gram multiplying by
$e^{\alpha\,\Delta n}=e^{0.7}\approx 2$ (i.e, doubling).

The remainding ``unallocated'' Grams will be used by the TON Reserve
as explained above in~\ptref{sp:unalloc.gr}. If the TON Foundation
needs any more Grams thereafter, it will simply convert into Grams
some of the funds it had previously obtained during the sale of the
coins, either on the free market or by buying Grams from the TON
Reserve.  To prevent excessive centralization, the TON Foundation will
never endeavour to have more than 10\% of the total amount of Grams
(i.e., 500 Megagrams) on its account.

\nxpoint\label{sp:bulk.sales} \embt(Bulk sales of Grams.)  When a lot
of people simultaneously want to buy large amounts of Grams from the
TON Reserve, it makes sense not to process their orders immediately,
because this would lead to results very dependent on the timing of
specific orders and their processing sequence.

Instead, orders for buying Grams may be collected during some
pre-defined period of time (e.g., a day or a month) and then processed
all together at once. If $k$ orders with $i$-th order worth $T_i$
dollars arrive, then the total amount $T=T_1+T_2+\cdots+T_k$ is used
to buy $\Delta n$ new coins according to \eqref{eq:new.coins}, and the
sender of the $i$-th order is allotted $\Delta n\cdot T_i/T$ of these
coins. In this way, all buyers obtain their Grams at the same average
price of $T/\Delta n$ USD per Gram.

After that, a new round of collecting orders for buying new Grams
begins.

When the total value of Gram buying orders becomes low enough, this
system of ``bulk sales'' may be replaced with a system of immediate
sales of Grams from the TON Reserve according to
formula~\eqref{eq:new.coins}.

The ``bulk sales'' mechanism will probably be used extensively during
the initial phase of collecting investments in the TON Project.

\end{document}
